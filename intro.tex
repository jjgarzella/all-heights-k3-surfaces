Given a variety \(X\) over a field of characteristic \(p\),
Artin and Mazur in \cite{artin-mazur-1977-height}
define the height \(h(X) \in \mathbb{N} \cup \{\infty\}\) 
(sometimes written \(h\) if \(X\) is obvious from the context)
to be the height 
of a certain formal group \(\Phi\) associated to \(X\).
The height is in some sense a measure of the
``arithmetic complexity'' of a variety,
with higher heights indicating ``more complexity''.
For example, consider the case when \(X\) is a 
K3 surface (i.e. a surface for which \(\omega_{X} \isom \mathcal{O}_{X} \) 
and \(H^{1}(X,\mathcal{O}_{X} ) = 0\)).
In this case, \(h\) determines the \textit{Newton polygon}
of \(X\), which in turn gives partial information about the 
point counts of \(X\) over finite fields.
In this case, the Newton polygon equals the Hodge polygon
(i.e. the variety is ordinary) if and only if \(h=1\),
and the Newton polygon is supersingular if and only if 
\(h = \infty\).

Given a field \(k\) of characteristic \(p\), one may ask

\begin{quest}
    For which values \(h\) does there exist a variety
    \(X\) such that \(h(X) = h\)?
\end{quest}

In the case that \(X\) is a K3 surface, it is known from the 
original work of Artin and Mazur
that either \(1 \leq h(X) \leq 10\) or \(h = \infty\).
One expects that such $h$ are taken on by K3 surfaces
in characteristic $p$.
For example, in \cite{artin-1974-k3-surfaces},
Artin shows\footnote{
    Artin showed this conditionally, dependent on flat
    duality for surfaces, which was later proved
    by Milne in \cite{milne-1976-flat-duality}.
}
that this holds over an algebraically
closed field \(k\) of characteristic \(p\) in the
case that \(p = 3 \mod 4\).
In \cite{taelman-2016-k3-given-l-function}, 
Taelman shows that
all possible \(h\) are realized
over some sufficiently big finite field \(\mathbb{F}_{q}\). 
By base change, this implies the result for extensions
thereof (e.g. algebraically closed fields).
However, Taelman's argument is not constructive,
and there is no known bound on how big \(q\) must be to guarantee
existence of surfaces of any height.

More concretely, it follows from the work of
Kedlaya and Sutherland in 
\cite{kedlaya-sutherland-2016-census-k3-f2}
that there exist quartic K3 surfaces
(i.e. quartic surfaces in \(\mathbb{P}^{3}\))
of all possible heights over \(\mathbb{F}_{2}\).
More recently,
Kawakami, Takamatsu, and Yoshikawa in \cite{kty-2022-fedder}
have given examples of quartic K3 surfaces
of all possible heights over \(\mathbb{F}_{3}\).

Kawakami, Takamatsu, and Yoshikawa consider, 
instead of the Artin-Mazur height, a different quantity called the
\textit{quasi-\(F\)-split height} 
(see Section 2 for a definition),
which is known to be the same as the Artin-Mazur height
for Calabi-Yau varieties, and thus K3 surfaces.
The theory of the quasi-\(F\)-split height has its
roots in the theory of \(F\)-singularites and commutative
algebra/birational geometry in characteristic \(p\).
In particular, one theorem fundamental
to the development of the theory of \(F\)-singularities
is Fedder's criterion (Theorem \ref{thm:fedder:criterion}), 
which gives a very concrete and computable way to check
whether or not a variety has height \(1\).
The main theorem of Kawakami, Takamatsu, and Yoshikawa
(\cite[Theorem~A]{kty-2022-fedder})
is a generalization of Fedder's criterion
to higher heights which gives a 
computable way to check whether a variety has height \(h\).
Moreover, they provide a simpler version
(\cite[Theorem~C]{kty-2022-fedder})
in the case when \(X\) is a Calabi-Yau hypersurface
(and thus also for quartic K3 surfaces).
The forthcoming work 
\cite{fgmq-2025-witt-vectors-macaulay2} 
provides an implementation of Theorem A.
However, this algorithm (and the ones used to compute
the examples
in \cite{kty-2022-fedder}) is not especially practical
as it requires multiplying many large polynomials
(\cite{takamatsu-2024-algorithm}, 
\cite{fgmq-2025-witt-vectors-macaulay2}),
limiting the computation to
very low characteristic.

In this work, we push the method of Kawakami,
Takamatsu, and Yoshikawa to the limits of readily
available hardware, 
and produce examples of quartic K3 surfaces \(X\) with all
possible quasi-\(F\)-split heights \(h\) 
over \(\mathbb{F}_{5}\) and \(\mathbb{F}_{7}\).
Our main insight is that one of the main
bottlenecks of the problem can be broken down 
into many repeated matrix-vector
multiplications for a square matrix of size 
\(\binom{4p-1}{3}\).
Because of this, we can use Nvidia's CUBLAS library 
\cite{nvidia-2024-cublas}
to perform the matrix multiplications
on the GPU, making such computations
nearly instantaneous.
To create this matrix, one must calculate the matrix of a
certain linear operator.
We provide a few novel algorithms which accomplish this task,
including some that can be implemented on the GPU.
The remaining algorithm is bottlenecked by the operation
of raising a polynomial to a large power, 
so we implement this on the GPU as well using 
a Fast Fourier Transform (FFT) approach.
Our algorithms and implementations lead to a high throughput
of heights of K3 surfaces: 
about 1400 surfaces per second over \(\mathbb{F}_{5}\) and
about 180 surfaces per second over \(\mathbb{F}_{7}\).
Our implementations can also handle \(\mathbb{F}_{11}\) 
and \(\mathbb{F}_{13}\), though they are slower than known methods
of calculating the Newton polygon (e.g. 
\cite{chk-2019-toric-controlled-reduction}).

All of our algorithms are implemented in Julia 
\cite{julia-2017}, and
make use of the OSCAR computer algebra system 
\cite{OSCAR-book}.
Code from this work is open source and available online 
in various
Julia packages: 
MMPSingularities.jl \cite{mmpsingularities-jl},
GPUFiniteFieldMatrices.jl \cite{gpuffmatrices-jl}, 
GPUPolynomials.jl \cite{gpupolynomials-jl},
and CudaNTTs.jl \cite{cudantts-jl}.

In Section 2, we give background on the quasi-\(F\)-split height
and Fedder's criterion.
In Section 3, we describe the naive implementation of
\cite[Theorem~C]{kty-2022-fedder} and describe
our modification.
In Section 4, we describe our algorithms for calculating the matrix
of the key linear operator which we term ``multiplying then splitting''.
In Section 5, we describe our GPU implementation of the Number Theoretic
Transform, which is a finite field variant of the FFT.
In Section 6, we describe the considerations we need to keep in mind
to use CUBLAS.
In Section 7, we describe the computation of surfaces of all possible
heights over \(\mathbb{F}_{5}\) and \(\mathbb{F}_{7}\).

Throughout the paper, we do various time tests to compare various
approaches for each computational step.
All timing tests were performed with 
an Intel i5-8400 CPU and a Nvidia GeForce RTX 3070 GPU.

\subsection{Acknowledgements}

The authors thank the maintainers and support staff 
of the UCSD research cluster for use of their devices,
and Wilson Cheung for much helpful tech support.
Furthermore, they thank the authors of 
\cite{kty-2022-fedder} and \cite{fgmq-2025-witt-vectors-macaulay2}
for providing preliminary implementations of 
their algorithms of Theorem C and Theorem A of 
\cite{kty-2022-fedder}.
The second author thanks Jakub Witaszek, Kiran S. Kedlaya,
Steve Huang, and Karl Schwede for helpful discussions.
The second author was partially supported by the 
National Science Foundation Graduate Research
Fellowship Program under Grant No. 2038238, and a fellowship
from the Sloan Foundation.

