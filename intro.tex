Given an algebraic variety \(X\) over a field of characteristic \(p\),
Artin and Mazur in \cite{artin-mazur-1977-height}
define the height \(h(X) \in \mathbb{N} \cup \{\infty\}\) 
(sometimes written \(h\) if \(X\) is obvious from the context)
to be the height 
of a certain formal group associated to \(X\).
The height is in some sense a measure of the
``arithmetic complexity'' of a variety,
with larger heights indicating ``more complexity.''
For example, consider the case when \(X\) is a 
K3 surface (i.e. a surface for which \(\omega_{X} \isom \mathcal{O}_{X} \) 
and \(H^{1}(X,\mathcal{O}_{X} ) = 0\)).
In this case, \(h\) determines the \textit{Newton polygon}
of \(X\), which in turn gives partial information about the 
point counts of \(X\) over finite fields,
i.e. \(\#X(\mathbb{F}_{q})\) for \(q = p^{r}, r \in \mathbb{N}\).
One may construct another polygon, the \textit{Hodge polygon},
from the Hodge-theoretic data associated to \(X\).
For K3 surfaces (and in fact for any surfaces), 
the Newton polygon equals the Hodge polygon
(i.e. the variety is ordinary) if and only if \(h=1\),
and the Newton polygon is \textit{supersingular} if and only if 
\(h = \infty\).

Given a field \(k\) of characteristic \(p\), one may ask

\begin{quest}
    For which values \(h\) does there exist a variety
    \(X\) such that \(h(X) = h\)?
\end{quest}

In the case that \(X\) is a K3 surface, it is known from the 
original work of Artin and Mazur
that either \(1 \leq h(X) \leq 10\) or \(h = \infty\).
One expects that all such $h$ are realized by K3 surfaces
in characteristic $p$.
For example, in \cite{artin-1974-k3-surfaces},
Artin shows\footnote{
    Artin showed this conditionally, dependent on flat
    duality for surfaces, which was later proved
    by Milne in \cite{milne-1976-flat-duality}.
}
that this holds over an algebraically
closed field \(k\) of characteristic \(p\) in the
case that \(p \equiv 3 \mathmod 4\).
In \cite{taelman-2016-k3-given-l-function}, 
Taelman shows that
all possible \(h\) are realized
over some sufficiently big finite field \(\mathbb{F}_{q}\). 
By base change, this implies the result for extensions
thereof (e.g. algebraically closed fields).
However, Taelman's argument is not constructive,
and there is no known bound on how big \(q\) must be to guarantee
existence of surfaces of any height.

More concretely, it follows from the work of
Kedlaya and Sutherland in 
\cite{kedlaya-sutherland-2016-census-k3-f2}
that there exist quartic K3 surfaces
(i.e. quartic surfaces in \(\mathbb{P}^{3}\))
of all possible heights over \(\mathbb{F}_{2}\).
More recently,
Kawakami, Takamatsu, and Yoshikawa in \cite{kty-2022-fedder}
have given examples of quartic K3 surfaces
of all possible heights over \(\mathbb{F}_{3}\).

Kawakami, Takamatsu, and Yoshikawa consider, 
instead of the Artin--Mazur height, another quantity called the
\textit{quasi-\(F\)-split height} 
(see Section 2 for a definition),
which is known to be equal to the Artin--Mazur height
for Calabi-Yau varieties, and thus K3 surfaces.
The theory of the quasi-\(F\)-split height has its
roots in the theory of \(F\)-singularities and commutative
algebra/birational geometry in characteristic \(p\).
In particular, one theorem fundamental
to the development of the theory of \(F\)-singularities
is Fedder's criterion (Theorem \ref{thm:fedder:criterion}), 
which gives a very concrete and computable way to check
whether or not a variety has height \(1\).
The main theorem of Kawakami, Takamatsu, and Yoshikawa
(\cite[Theorem~A]{kty-2022-fedder})
is a generalization of Fedder's criterion
to higher heights which gives a 
computable way to check whether a variety has height \(h\).
Moreover, they provide a simpler version
(\cite[Theorem~C]{kty-2022-fedder})
in the case when \(X\) is a Calabi-Yau hypersurface
(and thus also for quartic K3 surfaces).
The forthcoming work 
\cite{fgmqt-2025-witt-vectors-macaulay2} 
provides an implementation of Theorem A.
However, the algorithm used to compute the examples in
\cite{kty-2022-fedder}
is not especially practical
as it requires multiplying many large polynomials
(\cite{takamatsu-2024-algorithm}, 
\cite{fgmqt-2025-witt-vectors-macaulay2}),
limiting the computation to
very low characteristic.



In this work, we use computational techniques to push the method of Kawakami,
Takamatsu, and Yoshikawa to the limits of readily
available hardware.
With our algorithmic improvements, we produce examples of quartic K3 surfaces \(X\) with all
possible heights \(h\) 
over \(\mathbb{F}_{5}\) and \(\mathbb{F}_{7}\).
Our techniques are also applicable over \(\mathbb{F}_{11}\) and 
\(\mathbb{F}_{13}\), where
we produce examples of heights \(1 \leq h \leq 6\) and 
\(1 \leq h \leq 5\), respectively.

\subsection{On computational bottlenecks}

We briefly overview the computational obstacles to fast computation
of \(h\).
As seen in Section \ref{sec:preliminaries},
the algorithm presented in \cite{kty-2022-fedder} boils down to the 
following elementary computations.

\begin{enumerate}[(1)]
    \item The computation of a quantity 
        called \(\Delta_{1}\) (see Section \ref{subsec:compute:delta1}
        particularly Definition \ref{defn:delta1}),
        which involves
        \begin{enumerate}[(a)]
            \item Raising polynomials to powers
            \item Extracting the cross terms from 
                the result of the previous bullet point
                (See Proposition \ref{prop:delta1:formula}
                for the formal meaning of this operation).
        \end{enumerate}
        The coefficients in these steps must be over 
        \(\mathbb{Z} / p^{2} \mathbb{Z}\).
    \item A loop, which is guaranteed to run \(h-1\) times, in which we must
        \begin{enumerate}[(a)]
            \item perform a large polynomial multiplication 
                for which \(\Delta_{1}\) is one of the summands
            \item apply an operation which we call ``splitting'', 
                following the literature on the
                quasi-\(F\)-split height.
                This is an operation on polynomials which involves
                discarding many terms and lowering the exponents of
                the remaining terms.
                See Algorithm \ref{alg:naive:u} for a precise description.
        \end{enumerate}
        The coefficients in these steps live over \(\mathbb{F}_{p}\).
\end{enumerate}

In practice, the computation is bottlenecked by steps 
(1a) and (2a), as (1b) and (2b) are linear-time in the number of
terms of the polynomial.
Our implementation, which can be found in 
\cite{mmpsingularities-jl}, uses OSCAR, and thus uses FLINT
for polynomial multiplication and powering operations.
However, for \(p = 5\), the algorithm
takes [TIME] seconds for (1) and [TIME] seconds for (2),
while for \(p = 7\) it takes [TIME] seconds for (1) and
[TIME] seconds for (2).

Our main insight is that each iteration
of (2) is a linear operator, and that we can speed
up the computation by pre-computing its matrix.
We call this matrix the ``matrix of multiply then split'',
and once we do this, replace each iteration of the loop 
with a (dense\footnote{
    Our algorithm could be made sparse, but we do not need it here,
    since as we will see below, matrix-vector multiplication
    is not our bottleneck.
    Thus, we put this off for future work.
}) matrix-vector multiplication.
This matrix-vector multiplication can be carried out using FLINT
or Julia's built-in matrix multiplication, both of which call the same
underlying BLAS library.
We provide algorithms for calculating the matrix of multiply
then split in Section \ref{sec:alg:momts}.
This essentially eliminates (2) as a bottleneck.
In our implementation of this version, step (2) takes
 seconds for \(p = 5\) and [TIME] seconds for \(p = 7\).
Meanwhile, calculating the matrix of multiply then split
takes [TIME] seconds for \(p=5\) and 
[TIME] seconds for \(p = 7\).

To address the bottleneck of (1),
we have no clever trick. 
Instead, we implement polynomial powering via Fast Fourier Transform (FFT) using
Graphics Processing Unit (GPU) hardware,
allowing us to run the computation in a massively parallel way.
Our GPU implementation of (1) takes 0.017 seconds for \(p = 5\) and 0.16 seconds for \(p = 7\).
In order to avoid the memory overhead of communication between the Central Processing Unit (CPU) and
the GPU, we also provide a GPU implementation of calculating the matrix
of multiply then split. 
We also use Nvidia's CUBLAS library \cite{nvidia-2024-cublas} to
do the matrix-vector multiplication.
We emphasize that these are only necessary because of the communication overhead, and that matrix-vector
multiplication on the CPU is already fast enough to overcome bottleneck (2).

In our tests, we were able to achieve a high throughput of examples,
about 1400 surfaces per second over \(\mathbb{F}_{5}\) and
about 180 surfaces per second over \(\mathbb{F}_{7}\).
This is roughly three orders of magnitude better than the state of the art
(see Section ).
Our implementations can also compute the height \(h\) for examples
over
\(\mathbb{F}_{11}\) and \(\mathbb{F}_{13}\),
where they are about 1.5 orders of magnitude better than the state of the art.
Over these two primes, our algorithms are roughly two orders of magnitude
better than other methods for calculating \(h\),
but are not fast enough to compute examples of all possible heights.
All of our algorithms are implemented in Julia 
\cite{julia-2017}, and
make use of the OSCAR computer algebra system 
\cite{OSCAR-book}.
Code from this work is open source and available online 
in various
Julia packages: 
MMPSingularities.jl \cite{mmpsingularities-jl},
GPUFiniteFieldMatrices.jl \cite{gpuffmatrices-jl}, 
GPUPolynomials.jl \cite{gpupolynomials-jl},
and CudaNTTs.jl \cite{cudantts-jl}.

% One possibility to further speed up the the computation
% would be to compute the matrix of multiply then split
% in a lazy/sparse way, computing each row only when it
% is needed for (2).
% However, as we are still bottlenecked by (1), 
% it is unlikely to affect the overall computation much.

% Our algorithm, which is Theorem C of \cite{kty-2022-fedder}, 
% is not presented as an algorithm. 
% In Section \ref{sec:preliminaries}, we give a more thorough overview.

% Our main insight is that one of the main
% bottlenecks of the problem can be broken down 
% into many repeated matrix-vector
% multiplications for a square matrix of size 
% \(\binom{4p-1}{3}\).
% Because of this, we can use Nvidia's CUBLAS library 
% \cite{nvidia-2024-cublas}
% to perform the matrix multiplications
% on a GPU, making such computations
% nearly instantaneous.
% To create this matrix, one must calculate the matrix of a
% certain linear operator.
% We provide a few novel algorithms which accomplish this task,
% including some that can be implemented on the GPU.
% The remaining algorithm is bottlenecked by the operation
% of raising a polynomial to a large power, 
% so we implement this on the GPU as well using 
% a Fast Fourier Transform (FFT) approach.
% Our algorithms and implementations lead to a high throughput
% of heights of K3 surfaces: 
% about 1400 surfaces per second over \(\mathbb{F}_{5}\) and
% about 180 surfaces per second over \(\mathbb{F}_{7}\).
% Our implementations can also handle \(\mathbb{F}_{11}\) 
% and \(\mathbb{F}_{13}\). 
% For \(p = 11\), our method is comparable to the state of the art
% for computing Newton polygons (e.g. 
% \cite{chk-2019-toric-controlled-reduction}),
% though it is noticeably slower 
% for \(p = 13\) (see section \ref{sec:heights:surfaces}).

\subsection{Outline of this paper}


In Section 2, we give background on the quasi-\(F\)-split height
and Fedder's criterion.
In Section 3, we describe the naive implementation of
\cite[Theorem~C]{kty-2022-fedder} and describe
our modification.
In Section 4, we describe our algorithms for calculating the matrix
% We say "multiply then split" everywhere else
of the key linear operator which we term ``multiply then split''.
In Section 5, we describe our GPU implementation of the Number Theoretic
Transform, which is a finite field variant of the FFT.
In Section 6, we describe the considerations we need to keep in mind
to use CUBLAS.
In Section 7, we describe the computation of surfaces of all possible
heights over \(\mathbb{F}_{5}\) and \(\mathbb{F}_{7}\).

Throughout the paper, we do various timing experiments to compare 
approaches for each computational step.
All timing tests were performed with 
an Intel i5-8400 CPU and a Nvidia GeForce RTX 3070 GPU.

\subsection{Future Work}

In the future, we hope to implement other algorithms, such as Theorem A of \cite{kty-2022-fedder}
and other variants of the quasi-\(F\)-split height.
We also hope that our speedups will inspire others in computer algebra to consider the possibilities
for optimization using GPU acceleration methods.

\subsection{Acknowledgements}

The authors thank the maintainers and support staff 
of the UCSD research cluster for use of their devices,
and Wilson Cheung for much helpful tech support.
Furthermore, they thank the authors of 
\cite{kty-2022-fedder} 
and \cite{fgmqt-2025-witt-vectors-macaulay2}
for providing preliminary implementations of 
Theorem C and Theorem A
of \cite{kty-2022-fedder}.
The second author thanks Jakub Witaszek, Kiran S. Kedlaya,
Steve Huang, and Karl Schwede for helpful discussions.
The authors are also very grateful to 
Tommy Hofmann, Steve Huang, and 
Michael Monagan for giving feedback on an early draft of
this paper.

The second author was partially supported by the 
National Science Foundation Graduate Research
Fellowship Program under Grant No. 2038238, and a fellowship
from the Sloan Foundation.

