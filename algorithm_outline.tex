
\subsection{The Witt Vectors}

We begin by defining the Witt Vectors. 
Since the theory of Witt vectors is a vast and active topic of research,
we only recall what is required for the Fedder's criterion algorithm.
For more (REFERENCES HERE ETC ETC)

\begin{defn}
	The \(n\)-th Witt Polynomial \(\omega_{n}\) is defined as
	\[
		\omega_{n}(X_{0}, \ldots, X_{n}) = X_{0}^{p^{n}} + pX_{1}^{p^{n-1}} + \ldots + p^{n}X_{n}
	\] 
\end{defn}

Now let \(R\) be a ring of characteristic \(p\).
We define the map \(\Phi\) to be the map 
\[
	\prod_{n \in \mathbb{N}}^{} R 
	\xrightarrow{(\omega_{n})_{n}} 
	\prod_{n \in \mathbb{N}}^{} R  
\] 
defined as \(\omega_{n}\) for the \(n\)-th component.

\begin{lem}
	There exist integer polynomials 
	\(S_n(X_{0}, \ldots, X_{n}, Y_{0}, \ldots, Y_{n})\) 
	with the property that 
	\[
		\Phi((S_{n})_{n \in \mathbb{N}}) =
		\Phi((X_{n})_{n \in \mathbb{N}}) 
		+ \Phi((Y_{n})_{n \in \mathbb{N}})
	.\] 
	Likewise, there exist integer polynomials 
	\(P_{n}(X_{0}, \ldots, X_{n}, Y_{0}, \ldots, Y_{n})\)
	such that
	\[
		\Phi((P_{n})_{n \in \mathbb{N}}) =
		\Phi((X_{n})_{n \in \mathbb{N}}) 
		\cdot \Phi((Y_{n})_{n \in \mathbb{N}})
	.\] 
\end{lem}

\begin{proof}
	See Rabinoff for example for a proof.
\end{proof}

We now define the ring of Witt vectors 
\(W(R)\) to be 
\(\prod_{n \in \mathbb{N}}^{} R \) 
as a set, with the ring structure defined
by 
\[
	(a_{n})_{n \in \mathbb{N}} + 
	(b_{n})_{n \in \mathbb{N}} =
	(S_{n}(a_{0}, \ldots, a_{n}, b_{0}, \ldots, b_{n}))_{n \in \mathbb{N}}
\] 
and likewise multiplication is defined
using the \(P_{n}\).
The lemma then shows that \(\Phi\) is a homomorphism 
\[
	W(R) \xrightarrow{} \prod_{n \in \mathbb{N}}^{} R 
.\] 

The main example, which also provides the fundamental motivation
for the Witt vectors, is the case when \(R = \mathbb{F}_{p}\).
It is well known that \(W(\mathbb{F}_{p}) = \mathbb{Z}_{p}\), 
giving an alternative construction of the \(p\)-adic numbers.
However, we caution the reader that if one takes a
naive \(p\)-adic expansion 
\(\sum_{n = 0}^{\infty} c_{n}p^{n} \in \mathbb{Z}_{p}\)
with \(c_{n} \in \{0, \ldots, p-1\}\), this does not correspond
to the Witt vector \((c_{0}, c_{1}, c_{2}, \ldots)\).
In fact, the aforementioned sum corresponds to 
\((c_{0}, c_{1}^{p}, c_{2}^{p^{2}}, \ldots)\), see
(e.g. cite Kim section 2, perhaps rabinoff). 
This motivates the following, which we recall without proof:

\begin{defn}
	There exists a homomorphism \(F \colon W(R) \xrightarrow{} W(R)\),
	called the \textit{Frobenius},
	defined by
	\[
	 (c_{0}, c_{1}, \ldots) \mapsto (c_{0}^{p}, c_{1}^{p}, \ldots)
	\] 
\end{defn}

\begin{defn}
	There exists a homomorphism \(V \colon W(R) \xrightarrow{} W(R)\),
	called the \textit{Vershibung},
	defined by
	\[
		(c_{0}, c_{1}, c_{2}, \ldots) \mapsto  (0, c_{0}, c_{1}, \ldots)
	\] 
\end{defn}

\begin{lem}
	The composition \(F \circ V = V \circ F\) is the multiplication by
	\(p\) map on \(W(R)\).
\end{lem}

\begin{proof}
	Cite Rabinoff
\end{proof}

Thus, we see that \(W(R) / pW(R) \isom R\).
This is sometimes called the first \textit{truncated}
Witte vectors.
We also have higher truncated variants.

\begin{defn}
	The \(n\)-truncated Witt vectors
	\(W_{n}(R)\) are defined as
	\(W(R) / p^{n}W(R)\).
\end{defn}

By the above lemma, this translates to an actual truncation
in the coordinates of the Witt vector.


* motivate with Z_p example
* warn that the coordinates are different
* state the definitions of frobenius and vershibung
* quote that FV = p
* deduce that that W(R) / pW(R) = R
* thus define truncated Witt vectors

* polite definition of the Witt vectors
  * also truncated Witt vectors
* references
  * rabinoff, Hazewinkel ch 17, kim, Schnieder, Kedlaya 
  * bunch of theory papers

Comment on the description of the polynomials S_0 and S_1,
and how they are computationally related to 
raising to powers of p

\subsection{Splittings of Frobenius}

* the u map
* colon ideals
* fedder's criterion

\subsection{An algorithmic description of Fedder's criterion}

We now describe how the Witt Vectors can be used to calculate the 
quasi-F-split height / Artin-Mazur height of a Calabi-Yau 
hypersurface, following \cite{quasifedder}.
We will describe the algorithm in more detail, while proofs can be
found in \cite{quasifedder}.

\subsubsection{\(\Delta_{1}\)}

For the following discussion, 
let \(k\) be a perfect field of characteristic \(p\) and 
let \(S := k[x_{1}, \ldots, x_{n}\).
Let \(f = \sum_{I}^{} a_{I}\mathbf{x}^{I}\) be a polynomial in \(S\).

\begin{defn}
	Let \(\Delta_{1}(f)\)
	be defined by the following equation in \(W_{2}(S)\) :
	\[
		(0, \Delta_{1}(f)) = (f,0) - \sum_{I}^{} (a_{I}\mathbf{x}^{I}, 0) 
	\] 
\end{defn}

The following proposition demonstrates how we can calculate 
\(\Delta_{1}(f)\) algorithmically.

\begin{prop}
	\label{prop:delta1:formula}
	Let \([f]\) be a lift of \(f\) to the integers,
	i.e.
	\(f = \sum_{I}^{} [a_{I}] \mathbf{x}^{I}\).
	If \(k = \mathbb{F}_{p}\),
	we can compute \(\Delta_{1}(f)\) by 
	taking the reduction of
	\[
		\frac{[f]^{p} - \sum_{I}^{} ([a_{I}]x^{I})^{p} }{p}
	\] 
	mod \(p\).
\end{prop}

\begin{proof}
	Unwind the definition of addition in the Witt Vectors
\end{proof}

\begin{rmk}
	If \(f\) is a homogeneous polynomial of degree \(d\), 
	then \(\Delta_{1}(f)\) is a polynomial of degree \(pd\).
\end{rmk}

\subsection{The algorithm}

Let \(f\) be a homogeneous polynomial of degree \(n\) 
in \(S\). 
Then \(Z(f)\) is a Calabi-Yau hypersurface.

Let \(u\) be the generator of \(\Hom(S,F_{\star}S)\) from the 
previous section.
Define \(\theta\) by  (INSERT HERE BASED ON PREV SECTION NOTATION)

\begin{alg}
	\label{alg:qfs:outline}

	Inputs: \(b \in \mathbb{N}\) chosen bound, \(f\) a homogeneous
	polynomial of degree \(n\) in \(S\).
	\begin{enumerate}[(1)]
		\item Compute \(f^{p-1}\)
		\item If \(f^{p-1} \notin \mathfrak{m}^{p}\), return \(1\). 
		\item Compute \(\Delta_{1}(f^{p-1})\) via Proposition \ref{prop:delta1:formula}.
		\item Set \(n = 2\) and \(g = f^{p-1}\)
		\item Loop
			\begin{enumerate}[(1)]
				\item \(g := \theta(g)\) (using the output of Step 3)
				\item If \(g \notin \mathfrak{m}^{p},\) return \(n\).
				\item If \(n < b\), return \(\infty\)	
			\end{enumerate}
	\end{enumerate}
\end{alg}

\begin{thm}
	[\cite{quasifedder}, Theorem C]
	Assume that
	\(Z(f)\) has quasi-F-split height \(h < b\),
	then Algorithm \ref{alg:qfs:outline} terminates
	and returns \(h\).
\end{thm}

\begin{proof}
	This is just rephrasing \cite{quasifedder}, Theorem C.
\end{proof}

In the case of Calabi-Yau hypersurfaces, we have bounds on the height by
(INSERT CITATION OF THE RIGHT VAN DER GEER AND KATSURA PAPER, THERE ARE FOUR OPTIONS), 
so we can deterministically recover the height.  

\subsubsection{Our improvements}

A naive implmentation of Algorithm \ref{alg:qfs:outline}
is provided in MMPSingularities.jl.
The bottleneck ends up being polynomial multiplication, 
in two places:
raising \(f^{p-1}\) to the \(p\)-th power in the integers,
and multiplying \(g\) by \(\Delta_{1}(f^{p-1})\) 
in \(\theta\).

For the first, we implement a gpu-based FFT that is
further optimized for homogeneous inputs.
For the second, we observe that \(f^{p-1}\) has degree
\(n(p-1)\). 
Furthermore, we observe that by Remark \ref{} and 
(REF FSPLIT SECTION), \(\theta\) is a linear map
from the vectors space of 
homogenous polynomials of degree \(n(p-1)\) 
to itself.
Thus, if we can efficiently compute the matrix of \(\theta\),
we can repeatedly apply matrix-vector multiplication,
which is much faster.


