
\subsection{Witt vectors}

We begin by defining the ring of Witt vectors. 
Since the theory of Witt vectors is a vast and active topic of research,
we recall the bare minimum 
required for the quasi-\(F\)-split Fedder's criterion algorithm.
For a more complete introduction to the ring of Witt vectors with proofs, 
see \cite{rabinoff-2014-witt-vec}.
For an intuitive introduction or derivation of the ring of Witt vectors, see
\cite{kim-2017-witt-vec}.
There are many other perspectives on Witt vectors. 
For example:
\cite[Chapter~17]{hazewinkel-1978-formal-groups} covers the ring of Witt vectors
and its relationship with formal groups; 
\cite{kedlaya-2021-prismatic} gives a categorical perspective on the ring of Witt vectors
that relates to lifts of the Frobenius; and
\cite[Chapter~1]{schneider-2017-galois-rep-phi-gamma} defines a generalization
known as the ring of \textit{ramified Witt vectors} in detail.

\begin{defn}
	The \textit{\(n\)-th Witt Polynomial} \(\omega_{n}\) is defined as
	\[
		\omega_{n}(X_{0}, \ldots, X_{n}) = X_{0}^{p^{n}} + pX_{1}^{p^{n-1}} + \ldots + p^{n}X_{n}
        \in \mathbb{Z}[X_{0}, \ldots, X_{n}]
	\] 
    
\end{defn}

\noindent Now let \(R\) be a ring of characteristic \(p\).
We define the map \(\Phi\) to be the map 
% \[
% 	\prod_{n \in \mathbb{N}}^{} R 
% 	\xrightarrow{(\omega_{n})_{n}} 
% 	\prod_{n \in \mathbb{N}}^{} R  
% \] 
\begin{align*}
    \Phi : \prod_{n \in \mathbb{N}}^{R} R  
    &\xlongrightarrow{(\omega_{n})_{n}} \prod_{n \in \mathbb{N}}^{R} R \\
    (r_{0}, r_{1}, \ldots, r_{n}, \ldots) &\longmapsto 
    (\omega_{0}(r_{0}), \omega_{1}(r_{0}, r_{1}), 
    \ldots, \omega_{n}(r_{0}, \ldots, r_{n}), \ldots)
.\end{align*}

defined as \(\omega_{n}\) for the \(n\)-th component.
That is, \(\omega_{n}\) is considered as a 
map (of sets) \(R^{n} \xrightarrow{} R\) by
evaluation.

\begin{lem}
	There exist integer polynomials 
	\(S_n(X_{0}, \ldots, X_{n}, Y_{0}, \ldots, Y_{n})\) 
	with the property that 
	\[
		\Phi((S_{n})_{n \in \mathbb{N}}) =
		\Phi((X_{n})_{n \in \mathbb{N}}) 
		+ \Phi((Y_{n})_{n \in \mathbb{N}})
	.\] 
	Likewise, there exist integer polynomials 
	\(P_{n}(X_{0}, \ldots, X_{n}, Y_{0}, \ldots, Y_{n})\)
	such that
	\[
		\Phi((P_{n})_{n \in \mathbb{N}}) =
		\Phi((X_{n})_{n \in \mathbb{N}}) 
		\cdot \Phi((Y_{n})_{n \in \mathbb{N}})
	.\] 
\end{lem}

\begin{proof}
	For example, see \cite[Theorem~2.6]{rabinoff-2014-witt-vec} and the 
	surrounding discussion.
\end{proof}

We now define the ring of Witt vectors 
\(W(R)\) to be 
\(\prod_{n \in \mathbb{N}}^{} R \) 
as a set, with the ring structure defined
by 
\[
	(a_{n})_{n \in \mathbb{N}} + 
	(b_{n})_{n \in \mathbb{N}} =
	(S_{n}(a_{0}, \ldots, a_{n}, b_{0}, \ldots, b_{n}))_{n \in \mathbb{N}}
\] 
and likewise, multiplication is defined
using the \(P_{n}\).
The lemma then shows that \(\Phi\) is a homomorphism 
\[
	W(R) \xrightarrow{} \prod_{n \in \mathbb{N}}^{} R 
.\] 
The fact that the polynomials \(S_{n}\) and \(P_{n}\) do not depend on the base
ring \(R\) means that the construction is functorial; that is,
for a map of rings \(R \xrightarrow{} R^{\prime} \), we get a 
map \(W(R) \xrightarrow{} W(R^{\prime})\).

The main example, which also provides the fundamental motivation
for Witt vectors, is the case when \(R = \mathbb{F}_{p}\).
It is well known that \(W(\mathbb{F}_{p}) = \mathbb{Z}_{p}\), 
giving an alternative construction of the \(p\)-adic numbers.

\begin{warn}
	If one takes a
    naive \(p\)-adic expansion 
    \(\sum_{n = 0}^{\infty} c_{n}p^{n} \in \mathbb{Z}_{p}\)
    with \(c_{n} \in \{0, \ldots, p-1\}\), this does not correspond
    to the Witt vector \((c_{0}, c_{1}, c_{2}, \ldots)\).
    In fact, the aforementioned sum corresponds to 
    \((c_{0}, c_{1}^{p}, c_{2}^{p^{2}}, \ldots)\). See
	\cite[Section~2]{kim-2017-witt-vec}. 
    This motivates the following: 
\end{warn}

\begin{defn}
	There exists a homomorphism \(F \colon W(R) \xrightarrow{} W(R)\),
	called the \textit{Frobenius},
	defined by
	\[
	 (c_{0}, c_{1}, \ldots) \mapsto (c_{0}^{p}, c_{1}^{p}, \ldots)
	\] 
    (this is induced by the Frobenius on \(R\) by functoriality).
\end{defn}

\begin{defn}
	There exists a homomorphism \(V \colon W(R) \xrightarrow{} W(R)\),
	called the \textit{Verschiebung},
	defined by
	\[
		(c_{0}, c_{1}, c_{2}, \ldots) \mapsto  (0, c_{0}, c_{1}, \ldots)
	.\] 
\end{defn}

We see that \(W(R) / VW(R) \isom R\).
This is sometimes called the first \textit{truncated}
Witt vectors.
We also have higher truncated variants.

\begin{defn}
	The \(n\)-th truncated Witt vectors
	\(W_{n}(R)\) are defined as
	\(W(R) / V^{n}W(R)\).
\end{defn}

We may regard the \(n\)-th truncated Witt Vectors as consisting
of elements \((c_{0}, \ldots, c_{n-1})\) with addition
and multiplication by the \(S_{i}\) and \(P_{i}\).
Thus, one has \(\lim_{n} W_{n}(R) = W(R)\).

\begin{lem}
    \label{lem:frob:versch}
	The composition \(F \circ V = V \circ F\) is the multiplication by
	\(p\) map on \(W(R)\).
\end{lem}

\begin{proof}
	\cite[Proposition~5]{kim-2017-witt-vec}
\end{proof}

\begin{lem}
    \label{lem:perfect:witt}
    Let \(R\) be a \textit{perfect} ring,
    that is, a ring for which \(F\) is an isomorphism.
    Then \(W_{n}(R) = W(R) / p^{n}W(R)\).
\end{lem}

\begin{proof}
    Since \(F\) is an isomorphism, this follows from 
    Lemma \ref{lem:frob:versch}
\end{proof}

\begin{rmk}
    If we take \(R = \mathbb{F}_{p}\), then
    Lemma \ref{lem:perfect:witt}
    shows that \(W_{n}(\mathbb{F}_{p}) = \mathbb{Z} / p^{n}\mathbb{Z}\).
    One common intuition for \(W_{n}(R)\)
    for more general \(R\) is an analogy:
    \(\mathbb{Z} / p^{n}\mathbb{Z}\) 
    is to \(\mathbb{F}_{p}\) 
    as \(W_{n}(R)\) is to \(R\).
    For rings that are not perfect,
    this intuition is less precise but still
    somewhat useful.
\end{rmk}

\begin{nota}
    We have a map (of sets) 
    \(R \xrightarrow{} W(R)\) 
    given by 
    \(c \mapsto (c, 0, \ldots)\).
    The element \((c, 0, \ldots)\) is denoted
    \([c]\).
    The map \(c \mapsto [c]\) is multiplicative
    but not additive (see \cite[Section~1]{kim-2017-witt-vec}).
    We have analogous maps \(R \xrightarrow{} W_{n}(R)\).
    By abuse of notation, we denote 
    the image of \(c\) by \([c]\).
\end{nota}


\begin{rmk}
	\label{rmk:polyraise:w2}
    Our computations will end up primarily involving
    \(W_{2}(R)\)\footnote{
    This comes from the delta formula,
    \cite[Theorem~D]{kty-2022-fedder}},
    where addition is governed by the polynomials
	\(S_{0}(X_{0}, Y_{0}) = X_{0} + Y_{0}\)
	and
	\[
		S_{1}(X_{0}, X_{1}, Y_{0}, Y_{1})
		= X_{1} + Y_{1} + 
		\frac{(X_{0} + Y_{0})^{p} - X_{0}^{p} - Y_{0}^{p}}{p}
	.\] 
	Thus we see that addition in \(W_{2}(R)\) involves raising 
	elements in \(R\) (i.e. the first component) to the \(p\)-th
	power in a lift of \(R\) to characteristic 0.
    Thus, the bottleneck for most computations with Witt vectors,
    including the present work, tends to be raising
    integer polynonmials to powers.
\end{rmk}

\subsection{Splittings of Frobenius}
\label{subsec:split:frob}

Let \(R\) be a ring of characteristic \(p\). 
We have the Frobenius morphism 
\(F \colon R \xrightarrow{} R\), 
defined as \(F(x) = x^p\).
We describe a few alternative perspectives on
the Frobenius which will be useful later.

\begin{rmk}
	\label{rmk:frob:perspectives}
    \hfill
    \begin{enumerate}[(1)]
    	\item Let \(R\) be reduced. 
    		Then we may view the Frobenius as the inclusion
    		\(R \ins R^{1 / p}\), where \(R^{1 / p}\) 
    		is the ring of formal \(p\)-th roots of elements
    		of \(R\).
    	\item Similarly, again assuming that \(R\) is reduced
    		we may view the Frobenius as the inclusion
    		\(R^{p} \ins R\), 
    	\item More generally, we define \(F_{\star}R\) to
			be the \(R\)-algebra with ring structure 
			the same as \(R\), with module structure
			\(r \cdot x = F(r)x = r^{p}x\).
			Then we view the Frobenius as a map
			\(R \xrightarrow{} F_{\star}R\).
			In this description, \(F\) is an \(R\)-module 
			homomorphism as well. 
			The module \(F_{\star}R\) corresponds to the
			pushforward construction from geometry
			(i.e. pushforward of quasi-coherent
			sheaves on \(\Spec R\)).
			Even more generally, for any \(R\)-module
			\(M\) we denote \(F_{\star}M\) to be
			the analogously defined pushforward by
			Frobenius.
    \end{enumerate}
\end{rmk}

\begin{defn}
	We say that \(R\) is \textit{\(F\)-split} if the map \(F\) is
	split as a map of 
	\(R\)-modules \(R \xrightarrow{} F_{\star}R\).
\end{defn}

We will be chiefly concerned with \textit{hypersurfaces},
so we specialize to this case now.
For what follows, we assume that \(k\) is a field
of characteristic \(p > 0\) which is 
\textit{\(F\)-finite}; that is, the Frobenius
map is module-finite.

\begin{defn}
	Let \(S = k[x_{1}, \ldots, x_{n}]\).
	Then we say that \(f \in S\) is \(F\)-split if 
	\(S / (f)\) is.
\end{defn}

A fundamental fact about \(F\)-splitness is that there
exists a very concrete criterion for whether or 
not a polynomial (hypersurface) \(f\) is \(F\)-split.
First, we introduce some notation.
If \(I = (x_{1}, \ldots, x_{n})\) is a finitely generated 
ideal of some ring \(R\), 
then \(I^{[m]}\)
is defined to be \((x_{1}^{m}, \ldots, x_{n}^{m})\).

\begin{thm}
	[Fedder's Criterion]
    \label{thm:fedder:criterion}
	Let \(f \in S = k[x_{1}, \ldots, x_{n}]\).
	Let \(\mathfrak{m} = (x_{1}, \ldots, x_{n})\) 
	be the ideal generated by the variables.
	Then \(f\) is \(F\)-split if and only if 
	\(f^{p-1} \notin \mathfrak{m}^{[p]}\).
\end{thm}

\begin{proof}
	See \cite[Theorem~2.5]{ma-polstra-2021-F-sing-comm-alg}.
\end{proof}

Of particular interest is the case when \(f\) is 
homogeneous of degree \(n\), which geometrically
corresponds to a \textit{Calabi-Yau} hypersurface.
In this case, we have

\begin{thm}
	\label{thm:fsplit:ordinary}
	If \(f \in S\) 
	homogeneous of degree \(n\) 
	is \(F\)-split,
	then the Artin--Mazur height 
	of \(Z(f) = \Proj (S / (f))\) 
	is 1. That is, \(Z(f)\) is weakly ordinary.
\end{thm}

Recently, in \cite{yobuko-2019-qfs-calabi-yau}
Yobuko introduced the notion of 
quasi-\(F\)-splitness, which generalizes \(F\)-splitness.

\begin{defn}
	The ring \(R\) is \textit{\(n\)-quasi-\(F\)-split} if there exists
	a map \(\phi \colon W_{n}(R) \xrightarrow{} R\) such that
	\[
	\begin{tikzcd}
		W_{n}(R) \arrow{r}{F} \arrow{d}[swap]{} &
		F_{\star}W_{n}(R) \arrow{ld}{\phi} \\
	R 
	\end{tikzcd}
	,\]
	where the vertical map is the first Witt vector truncation.
\end{defn}

We further define the \textit{quasi-\(F\)-split height} of \(R\) as the smallest \(n\) 
for which \(R\) is \(n\)-quasi-\(F\)-split.
As above, the quasi-\(F\)-split height of 
\(f \in S = k[x_{1}, \ldots, x_{n}]\) is that
of \(R = S / (f)\).

\begin{rmk}
	Both \(F\)-splitness and quasi-\(F\)-splitness have 
	various geometric
	variants which are more general then the 
	ring-theoretic/affine versions given here. 
	These are covered extensively in the literature, for example
	see \cite{kttwyy-2022-qfs-birat}.
\end{rmk}

The quasi-\(F\)-split height also gives a generalization
of Theorem \ref{thm:fsplit:ordinary}:

\begin{thm}
	If \(f \in S\) is homogeneous of degree \(n\),
	then the Artin--Mazur height of \(Z(f)\)
	is equal to the quasi-\(F\)-split height
	of \(f\).
\end{thm}

\begin{proof}
	This is a special case of 
	\cite[Theorem~4.5]{yobuko-2019-qfs-calabi-yau}.
\end{proof}

\section{Fedder's criterion for quasi-\(F\)-splitness: an algorithmic perspective}

We now describe how the ring of Witt vectors can be used to calculate the 
quasi-\(F\)-split height (equivalently, Artin--Mazur height) of a Calabi-Yau 
hypersurface, using the version of Fedder's criterion in \cite{kty-2022-fedder}.
We will describe the algorithm in more detail. Proofs can be
found in \cite{kty-2022-fedder}.

\subsection{The computation of \(\Delta_{1}\)}

For the following discussion, 
let \(k\) be a perfect field of characteristic \(p\) and 
let \(S := k[x_{1}, \ldots, x_{n}]\).
Let \(f = \sum_{I}^{} a_{I}\mathbf{x}^{I}\) be a polynomial in \(S\).

\begin{defn}
	Let \(\Delta_{1}(f)\)
	be defined by the following equation in \(W_{2}(S)\):
	\[
		(0, \Delta_{1}(f)) = (f,0) - \sum_{I}^{} (a_{I}\mathbf{x}^{I}, 0) 
	.\] 
\end{defn}

The following proposition demonstrates how we can calculate 
\(\Delta_{1}(f)\) algorithmically.

\begin{prop}
	\label{prop:delta1:formula}
	Let \(\tilde{f}\) be a lift of \(f\) to \(W(k)\),
	i.e.
	\(\tilde{f} = \sum_{I}^{} [a_{I}] \mathbf{x}^{I}\).
	If \(k = \mathbb{F}_{p}\),
	we can compute \(\Delta_{1}(f)\) by 
	taking the reduction of
	\[
		\frac{\tilde{f}^{p} - \sum_{I}^{} ([a_{I}]x^{I})^{p} }{p}
	\] 
	mod \(p\).
\end{prop}

\begin{proof}
	Iteratively apply the formula of the first
	Witt polynomial \(S_{1}\) 
	to the monomials of \(\tilde{f}\).
\end{proof}

\begin{rmk}
	If \(f\) is a homogeneous polynomial of degree \(d\), 
	then \(\Delta_{1}(f)\) is a polynomial of degree \(pd\).
\end{rmk}

\begin{rmk}
    \label{rmk:lift:roi}
    Note that if \(k = \mathbb{F}_{p}\), we can always choose a lift
    of \(f\) with coefficients in \(\mathbb{Z}\), 
    not just \(W(k) = \mathbb{Z}_{p}\).
    In practice, we must choose a lift to the integers
    so we can represent it computationally.

    A similar fact holds for \(k = \mathbb{F}_{q}\) with 
    \(q \neq p\), replacing \(\mathbb{Z}\) with
    \(\mathbb{Z}[\zeta_{q-1}]\).
    This could be used to calculate the quasi-\(F\)-split 
    heights for surfaces not defined over \(\mathbb{F}_{p}\),
    though we have not attempted this.
    More generally, if \(q = p^{e}\), one may take
    any abelian extension \(K\) of degree \(e\) over \(\mathbb{Q}\) such that
    \(p\) is an inert prime in \(K\).
    Then, there is a unique prime \(\mathfrak{p}\) lying over \(p\) 
    such that \(\mathcal{O}_{K} / \mathfrak{p} \isom k\).
    Furthermore, \(\mathcal{O}_{K} \ins W(k)\)
    and so we may lift \(f\) to \(\mathcal{O}_{K}\) 
    and do arithmetic there. 
    Doing arithmetic in a degree \(e\) extension will have
    better complexity then in the cyclotomic extension, which
    has degree \(\phi(q) = \phi(p^{e}) = p^{e-1}(p-1)\), where
    \(\phi\) is the Euler totient function.
    We do not implement the calculation of quasi-\(F\)-split
    heights for \(q \neq p\).
\end{rmk}



Proposition \ref{prop:delta1:formula} gives a natural algorithm
for calculating the term \(\Delta_{1}(f)\): 


\begin{algorithm}[H]
\label{alg:calc:delta1}
\caption{Calculation of \(\Delta_{1}(f)\) }
\KwInput{$f \in \mathbb{F}_p[x_1, \dots , x_n]$}
\KwOutput{$\Delta_1(f)$}

$\tilde{f} \gets \texttt{lift}(f)$\;
$D \gets \tilde{f}^p$\;

\For{$t \in \textnormal{\texttt{terms}} (\tilde{f})$}{
	$D \gets D - t^p$
}

$D \gets D / p$

\Return{D}
\end{algorithm}

\subsection{Splittings of Frobenius from a computational perspective}

Let \(S = k[x_{1}, \ldots, x_{n}]\) as before. 
We will use perspective (3) from
Remark \ref{rmk:frob:perspectives}; 
recall that we identify 
\(S\) with the target of Frobenius and
\(S^{p}\) with the source.
We see 
by counting degrees
that we have a generating set for \(S\) as an
\(S^{p}\)-module given by
all monomials
\(x_{1}^{i_{1}}\cdots x_{n}^{i_{n}}\)
where \(0 \leq i_{j} \leq p-1\) for all \(j\).
Moreover, since \(S\) is a polynomial ring, 
there are no (module-theoretic) relations
and
\(S\) is the free \(S^{p}\)-module generated by 
these monomials, i.e.  \[
S = \bigoplus_{1 \leq j \leq n,~ 0 \leq i_{j} \leq p-1}^{} x_{1}^{i_{1}}\cdots x_{n}^{i_{n}} S^{p}
.\] 
Then the projection of \(S\) to any of the direct sum components
is an element of \(\Hom(S,S^{p})\), which is a
splitting of Frobenius.
Let \(u\) be the projection onto the component of
\(x_{1}^{p-1}, \ldots, x_{n}^{p-1}\).

The splitting \(u\) plays an important role in \(F\)-singularity
theory, see for example 
\cite[Claim~2.6]{ma-polstra-2021-F-sing-comm-alg}.
For our purposes, we are only concerned with computing \(u\) 
for a polynomial in \(S\). 
Given \(f \in S\), we will first compute
\(u(f) \in S^{p}\), and then use the identification
\(S^{p} \isom S\) by taking \(p\)-th roots of 
exponents.

\SetKwComment{Comment}{// }{}

\begin{algorithm}[H]
\caption{Splitting of Frobenius}
\label{alg:naive:u}
\KwInput{$f \in S$}
\KwOutput{$u(f) \in S$}
\Comment{Discard terms of $f$ with exponents not congruent to $(p-1, \dots, p-1) \mod p$}
\For{$t \in \textnormal{\texttt{terms}} (f)$}{
	\Comment{See Def \ref{def:poly:nota} for exps() notation}
	\If{$\textnormal{\texttt{exps}} (t) \not\equiv (p-1, \dots, p-1) \mod p$}{
		$f \gets f - t$
	}
}

$r \gets 0$\;
\For{$t \in \textnormal{\texttt{terms}} (f)$}{
	$\textnormal{\texttt{exps}}(t) \gets \textnormal{\texttt{exps}}(t) - (p-1, \dots, p-1)$\;
	\Comment{This division is exact because of the previous step}
	$\textnormal{\texttt{exps}}(t) \gets \textnormal{\texttt{exps}}(t) / p$\;
	$r \gets r + t$
}

\Return $r$
\end{algorithm}

\subsection{The naive algorithm}

Let \(f\) be a homogeneous polynomial of degree \(n\) 
in \(S\), 
so that \(Z(f)\) is a Calabi-Yau hypersurface.
Recall from Subsection \ref{subsec:split:frob} that
\(\mathfrak{m}^{[p]} = (x_{1}^{p}, \ldots, x_{n}^{p})\)
is the Frobenius power of the maximal ideal.
Following \cite{kty-2022-fedder}, we have the following
algorithm to calculate the quasi-\(F\)-split height.

\begin{algorithm}[H]
% I think consistency of the term overrides title capitalization.
\caption{quasi-\(F\)-split height: naive algorithm}
\label{alg:qfs:naive}
\KwInput{
	Chosen bound $b \in \mathbb{N}$, \\
	~~~~~~~~~~~~~Homogeneous polynomial $f \in S$ of degree $n$
}
\KwOutput{$h(f)$}
$g \gets f^{p - 1}$\;
\If{$g \notin \mathfrak{m}^{[p]}$}{
	\Return $1$\;
}
$\Delta \gets \Delta_1(f^{p-1})$\;
$h \gets 2$\;
\While{true}{
	\If{$b < h$}{
		\Return $\infty$\;
	}
	$g \gets u(\Delta g)$\;
	\If{$g \notin \mathfrak{m}^{[p]}$}{
		\Return $h$\;
	}
	$h \gets h + 1$\;
}
\end{algorithm}

\begin{thm}
	[\cite{kty-2022-fedder}, Theorem C]
	Assume that
	\(Z(f)\) has quasi-\(F\)-split height \(h < b\).
	Then Algorithm \ref{alg:qfs:naive} terminates
	and returns \(h\).
\end{thm}

\begin{proof}
	This is just rephrasing \cite[Theorem~C]{kty-2022-fedder}.
\end{proof}

In the case of Calabi-Yau hypersurfaces, we have bounds on the height by
\cite{van-der-geer-katsura-2003-calabi-yau},
so we can deterministically recover the height.  
See also \cite[Theorem~0.1]{artin-1974-k3-surfaces},
for the case of K3 surfaces.
For a K3 surface, the height (if finite) is bounded by 10.

\begin{ex}
    Let \(p=3\), and let
    \(f = x^4 + y^4 + z^4 + w^4 \in S = k[x,y,z,w]\).
    Raising \(f\) to the \(p-1 = 2\) in \(\mathbb{F}_{3}\), we get
    \[
    g = x^8 + 2x^4y^4 + 2x^4z^4 + 2x^4w^4 + y^8 + 2y^4z^4 + 2y^4w^4 + z^8 + 2z^4w^4 + w^8
    .\] 
    The only monomial of degree \(8\) that is not contained
    in \(\mathfrak{m}^{[p]} = (x^{3}, y^{3}, x^{3}, w^{3})\) is \(x^{2}y^{2}z^{2}w^{2}\).
    Call this monomial \(c\).
    Since this coefficient of \(c\) in \(g\) is zero, we conclude
    by the classical Fedder's criterion (Theorem \ref{thm:fedder:criterion})
    that \(g\) is not \(F\)-split.
    Next, we must calculate \(\Delta_{1}(g)\).
    To do this, we take a lift of \(g\) by regarding the coefficients
    as being in \(\mathbb{Z}\) instead of \(\mathbb{F}_{3}\).
    We raise \(g^3\) in the integers, and subract by every term of \(g\) 
    cubed.

    Then, we multiply \(g\) by \(\Delta_{1}(g)\), getting a polynomial of degree 24.
    We apply the splitting of Frobenius, as described in Algorithm \ref{alg:naive:u}, 
    obtaining another polynomial of degree \(8\). 
    We can now check the term \(x^2y^2z^2w^2\) again, and it will again be zero. 
    For this choice of $f$, all of the exponents for all quantities 
    must be multiples of four\footnote{
    In fact, this implies that the degree \(8\) polynomial is zero.
    However, this phenomenon is quite rare.
    }
    We then continue, multiplying, applying the splitting, checking the coefficient, 
    until we conclude that the height is greater than 10. 
    Now, by Hodge theory, if the height is greater than 10, it must be infinity. 
    So we conclude that the height is infinity.
\end{ex}

\subsection{The key idea: finding the matrix of the linear operator ``multiply then split''}

An implementation of Algorithm \ref{alg:qfs:naive}
is provided in MMPSingularities.jl.
The bottleneck ends up being polynomial multiplication, 
in two places:
\begin{enumerate}[(1)]
    \item raising (a lift of) \(g = f^{p-1}\) to the \(p\)-th power 
        (in line 4 of Algorithm \ref{alg:calc:delta1})
    \item multiplying \(g\) by \(\Delta_{1}(f^{p-1})\) 
        (in line 13 of Algorithm \ref{alg:qfs:naive})
\end{enumerate}

\noindent For a quartic K3 surface of characteristic \(5\),
for example, each step takes about 1 second 
using FLINT \cite{flint-2023-flint}. 

We now explain how to overcome the second
bottleneck. Since \(Z(f)\) is Calabi-Yau
(i.e. \(\deg f = n\)), we have that \(f^{p-1}\) has degree
\(n(p-1)\). 
Furthermore, by Proposition \ref{prop:delta1:formula}
the degree of \(\Delta_{1}(f^{p-1})\) 
is \(np(p - 1)\).
Thus, \(\Delta_{1}(f^{p-1})g\) has degree
\(n(p^{2} - 1)\); however, the effect 
of \(u\) on any polynomial is subtracting \(p-1\)
from the exponents of the terms and dividing by \(p\) 
(see Algorithm \ref{alg:naive:u}).
Thus, the ``multiply then split'' map 
\(g \mapsto u(\Delta_{1}(f^{p-1}) g)\) 
is a linear map from 
the space of homogeneous polynomials of degree \(n(p-1)\) 
to itself.
As a consequence of this observation,
if we can efficiently compute the matrix of 
\(g \mapsto u(\Delta_{1}(f^{p-1})g)\),
we can repeatedly apply matrix-vector multiplication.

Furthermore, when \(g\) is written as a vector 
in the basis of homogeneous monomials of degree
\(n(p-1)\), we can test if \(g \notin \mathfrak{m}^{[p]}\) 
in an especially simple way: the only 
monomial that is not in \(\mathfrak{m}^{[p]}\) is
\(x_{1}^{p-1}\cdots x_{n}^{p-1}\), 
see for example \cite{kty-2022-fedder}.
Thus, we can check if a single element of the vector representing
\(g\) is nonzero.

The algorithm for Fedder's criterion then becomes:

\begin{algorithm}[H]
\caption{Quasi-\(F\)-split height: matrix-based algorithm}
\label{alg:qfs:matrix}
\KwInput{
	Chosen bound $b \in \mathbb{N}$, \\
	~~~~~~~~~~~~~Homogeneous polynomial $f \in S$ of degree $n$
}
\KwOutput{$h(f)$}
$g \gets f^{p - 1}$\;
\If{$g \notin \mathfrak{m}^{[p]}$}{
	\Return $1$\;
}
$\Delta \gets \Delta_1(f^{p - 1})$\;
$M \gets $ the matrix of $g^{\prime} \mapsto u(\Delta g^{\prime})$\;
$h \gets 2$\;
$g_v \gets $ the representation of $g$ as a vector\;
$i \gets $ the index of the monomial $x_{1}^{p - 1} \dots x_{n}^{p - 1}$\;
\While{true}{
	\If{$b < h$}{
		\Return $\infty$\;
	}
	$g_v \gets Mg_v$\;
	\If{$g_v[i] \neq 0$}{
		\Return $h$\;
	}
	$h \gets h + 1$
}
\end{algorithm}

Thus, we have reduced our problem (algorithmically, at least)
to finding the matrix of the ``mulitply then split'' operation.
