
\subsection{The Witt Vectors}

TODO: at a polite introduction to the Witt Vectors here

* the Witt vectors
* the truncated Witt vectors

\subsection{Splittings of Frobenius}

* the u map
* colon ideals
* fedder's criterion

\subsection{An algorithmic description of Fedder's criterion}

We now describe how the Witt Vectors can be used to calculate the 
quasi-F-split height / Artin-Mazur height of a Calabi-Yau 
hypersurface, following \cite{quasifedder}.
We will describe the algorithm in more detail, while proofs can be
found in \cite{quasifedder}.

\subsubsection{\(\Delta_{1}\)}

For the following discussion, 
let \(k\) be a perfect field of characteristic \(p\) and 
let \(S := k[x_{1}, \ldots, x_{n}\).
Let \(f = \sum_{I}^{} a_{I}\mathbf{x}^{I}\) be a polynomial in \(S\).

\begin{defn}
	Let \(\Delta_{1}(f)\)
	be defined by the following equation in \(W_{2}(S)\) :
	\[
		(0, \Delta_{1}(f)) = (f,0) - \sum_{I}^{} (a_{I}\mathbf{x}^{I}, 0) 
	\] 
\end{defn}

The following proposition demonstrates how we can calculate 
\(\Delta_{1}(f)\) algorithmically.

\begin{prop}
	\label{prop:delta1:formula}
	Let \([f]\) be a lift of \(f\) to the integers,
	i.e.
	\(f = \sum_{I}^{} [a_{I}] \mathbf{x}^{I}\).
	If \(k = \mathbb{F}_{p}\),
	we can compute \(\Delta_{1}(f)\) by 
	taking the reduction of
	\[
		\frac{[f]^{p} - \sum_{I}^{} ([a_{I}]x^{I})^{p} }{p}
	\] 
	mod \(p\).
\end{prop}

\begin{proof}
	Unwind the definition of addition in the Witt Vectors
\end{proof}

\begin{rmk}
	If \(f\) is a homogeneous polynomial of degree \(d\), 
	then \(\Delta_{1}(f)\) is a polynomial of degree \(pd\).
\end{rmk}

\subsection{The algorithm}

Let \(f\) be a homogeneous polynomial of degree \(n\) 
in \(S\). 
Then \(Z(f)\) is a Calabi-Yau hypersurface.

Let \(u\) be the generator of \(\Hom(S,F_{\star}S)\) from the 
previous section.
Define \(\theta\) by  (INSERT HERE BASED ON PREV SECTION NOTATION)

\begin{alg}
	\label{alg:qfs:outline}

	Inputs: \(b \in \mathbb{N}\) chosen bound, \(f\) a homogeneous
	polynomial of degree \(n\) in \(S\).
	\begin{enumerate}[(1)]
		\item Compute \(f^{p-1}\)
		\item If \(f^{p-1} \notin \mathfrak{m}^{p}\), return \(1\). 
		\item Compute \(\Delta_{1}(f^{p-1})\) via Proposition \ref{prop:delta1:formula}.
		\item Set \(n = 2\) and \(g = f^{p-1}\)
		\item Loop
			\begin{enumerate}[(1)]
				\item \(g := \theta(g)\) (using the output of Step 3)
				\item If \(g \notin \mathfrak{m}^{p},\) return \(n\).
				\item If \(n < b\), return \(\infty\)	
			\end{enumerate}
	\end{enumerate}
\end{alg}

\begin{theorem}
	[\cite{quasifedder}, Theorem C]
	Assume that
	\(Z(f)\) has quasi-F-split height \(h < b\),
	then Algorithm \ref{alg:qfs:outline} terminates
	and returns \(h\).
\end{theorem}

\begin{proof}
	This is just rephrasing \cite{quasifedder}, Theorem C.
\end{proof}

In the case of Calabi-Yau hypersurfaces, we have bounds on the height by
(INSERT CITATION OF THE RIGHT VAN DER GEER AND KATSURA PAPER, THERE ARE FOUR OPTIONS), 
so we can deterministically recover the height.  

\subsubsection{Our improvements}

A naive implmentation of Algorithm \ref{alg:qfs:outline}
is provided in MMPSingularities.jl.
The bottleneck ends up being polynomial multiplication, 
in two places:
raising \(f^{p-1}\) to the \(p\)-th power in the integers,
and multiplying \(g\) by \(\Delta_{1}(f^{p-1})\) 
in \(\theta\).

For the first, we implement a gpu-based FFT that is
further optimized for homogeneous inputs.
For the second, we observe that \(f^{p-1}\) has degree
\(n(p-1)\). 
Furthermore, we observe that by Remark \ref{} and 
(REF FSPLIT SECTION), \(\theta\) is a linear map
from the vectors space of 
homogenous polynomials of degree \(n(p-1)\) 
to itself.
Thus, if we can efficiently compute the matrix of \(\theta\),
we can repeatedly apply matrix-vector multiplication,
which is much faster.


