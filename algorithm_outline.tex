
\subsection{The Witt Vectors}

We begin by defining the Witt vectors. 
Since the theory of Witt vectors is a vast and active topic of research,
we only recall the bare minimum 
required for the quasi-F-split Fedder's criterion algorithm.
For a more complete introduction to the Witt vectors with proofs, 
see \cite{rabinoff-2014-witt-vec}.
For an intuitive introduction/derivation of the Witt vectors, see
\cite{kim-2017-witt-vec}.
There are many other perspectives on Witt vectors. 
For example:
\cite[Chapter~17]{hazewinkel-1978-formal-groups} covers the Witt vectors
and its relationship with formal groups; 
\cite{kedlaya-2021-prismatic} gives a categorical perspecitve on Witt vectors
that relates to lifts of the Frobenius; and
\cite[Chapter~1]{schneider-2017-galois-rep-phi-gamma} defines a generalization
known as the \textit{ramified Witt vectors} in detail.

\begin{defn}
	The \(n\)-th Witt Polynomial \(\omega_{n}\) is defined as
	\[
		\omega_{n}(X_{0}, \ldots, X_{n}) = X_{0}^{p^{n}} + pX_{1}^{p^{n-1}} + \ldots + p^{n}X_{n}
	\] 
\end{defn}

Now let \(R\) be a ring of characteristic \(p\).
We define the map \(\Phi\) to be the map 
\[
	\prod_{n \in \mathbb{N}}^{} R 
	\xrightarrow{(\omega_{n})_{n}} 
	\prod_{n \in \mathbb{N}}^{} R  
\] 
defined as \(\omega_{n}\) for the \(n\)-th component.

\begin{lem}
	There exist integer polynomials 
	\(S_n(X_{0}, \ldots, X_{n}, Y_{0}, \ldots, Y_{n})\) 
	with the property that 
	\[
		\Phi((S_{n})_{n \in \mathbb{N}}) =
		\Phi((X_{n})_{n \in \mathbb{N}}) 
		+ \Phi((Y_{n})_{n \in \mathbb{N}})
	.\] 
	Likewise, there exist integer polynomials 
	\(P_{n}(X_{0}, \ldots, X_{n}, Y_{0}, \ldots, Y_{n})\)
	such that
	\[
		\Phi((P_{n})_{n \in \mathbb{N}}) =
		\Phi((X_{n})_{n \in \mathbb{N}}) 
		\cdot \Phi((Y_{n})_{n \in \mathbb{N}})
	.\] 
\end{lem}

\begin{proof}
	\cite[Theorem~2.6]{rabinoff-2014-witt-vec} and the 
	surrounding discussion, for example.
\end{proof}

We now define the ring of Witt vectors 
\(W(R)\) to be 
\(\prod_{n \in \mathbb{N}}^{} R \) 
as a set, with the ring structure defined
by 
\[
	(a_{n})_{n \in \mathbb{N}} + 
	(b_{n})_{n \in \mathbb{N}} =
	(S_{n}(a_{0}, \ldots, a_{n}, b_{0}, \ldots, b_{n}))_{n \in \mathbb{N}}
\] 
and likewise multiplication is defined
using the \(P_{n}\).
The lemma then shows that \(\Phi\) is a homomorphism 
\[
	W(R) \xrightarrow{} \prod_{n \in \mathbb{N}}^{} R 
.\] 
Moreover, the fact that the polynomials do not depend on the base
ring \(R\) means that the construction is functorial; that is,
for a map of rings \(R \xrightarrow{} R^{\prime} \), we get a 
map \(W(R) \xrightarrow{} W(R^{\prime})\).

The main example, which also provides the fundamental motivation
for the Witt vectors, is the case when \(R = \mathbb{F}_{p}\).
It is well known that \(W(\mathbb{F}_{p}) = \mathbb{Z}_{p}\), 
giving an alternative construction of the \(p\)-adic numbers.

\begin{warn}
    %However, we caution the reader that 
	If one takes a
    naive \(p\)-adic expansion 
    \(\sum_{n = 0}^{\infty} c_{n}p^{n} \in \mathbb{Z}_{p}\)
    with \(c_{n} \in \{0, \ldots, p-1\}\), this does not correspond
    to the Witt vector \((c_{0}, c_{1}, c_{2}, \ldots)\).
    In fact, the aforementioned sum corresponds to 
    \((c_{0}, c_{1}^{p}, c_{2}^{p^{2}}, \ldots)\), see
	e.g. \cite[Section~2]{kim-2017-witt-vec}. 
    This motivates the following: 
\end{warn}


\begin{defn}
	There exists a homomorphism \(F \colon W(R) \xrightarrow{} W(R)\),
	called the \textit{Frobenius},
	defined by
	\[
	 (c_{0}, c_{1}, \ldots) \mapsto (c_{0}^{p}, c_{1}^{p}, \ldots)
	\] 
	(this is induced by the Frobenius on \(R\) by functoriality).
\end{defn}

\begin{defn}
	There exists a homomorphism \(V \colon W(R) \xrightarrow{} W(R)\),
	called the \textit{Verschiebung},
	defined by
	\[
		(c_{0}, c_{1}, c_{2}, \ldots) \mapsto  (0, c_{0}, c_{1}, \ldots)
	\] 
\end{defn}

\begin{lem}
	The composition \(F \circ V = V \circ F\) is the multiplication by
	\(p\) map on \(W(R)\).
\end{lem}

\begin{proof}
	\cite[Proposition~5]{kim-2017-witt-vec}
\end{proof}

Thus, we see that \(W(R) / pW(R) \isom R\).
This is sometimes called the first \textit{truncated}
Witt vectors.
We also have higher truncated variants.

\begin{defn}
	The \(n\)-truncated Witt vectors
	\(W_{n}(R)\) are defined as
	\(W(R) / p^{n}W(R)\).
\end{defn}

By the above lemma, this translates to an actual truncation
in the coordinates of the Witt vectors.

\begin{rmk}
	\label{rmk:polyraise:w2}
    Our computations will end up primarily involving
    \(W_{2}(R)\)\footnote{
    This comes from the delta formula,
    \cite[Theorem~D]{kty-2022-fedder}},
    where addition is governed by the polynomials
	\(S_{0}(X_{0}, Y_{0}) = X_{0} + Y_{0}\)
	and
	\[
		S_{1}(X_{0}, X_{1}, Y_{0}, Y_{1})
		= X_{1} + Y_{1} + 
		\frac{(X_{0} + Y_{0})^{p} - X_{0}^{p} - Y_{0}^{p}}{p}
	.\] 
	Thus we see that addition in \(W_{2}(R)\) involves raising 
	things in \(R\) (i.e. the first component) to the \(p\)-th
	power in the integers.
	%This is the first place that raising polynomials
	%to the \(p\)-th power comes into the picture 
	%algorithmically.
\end{rmk}




\subsection{Splittings of Frobenius}

Let \(R\) be a ring of characteristic \(p\). 
We have the Frobenius morphism 
\(F \colon R \xrightarrow{} R\), 
defined as \(F(x) = x^p\).
We describe a few alternative perspectives on
the Frobenius which will be useful later.

\begin{rmk}
	\label{rmk:frob:perspectives}

    \begin{enumerate}[(1)]
    	\item Let \(R\) be reduced. 
    		Then we may view the Frobenius as the inclusion
    		\(R \ins R^{1 / p}\), where \(R^{1 / p}\) 
    		is the ring of formal \(p\)-th roots of elements
    		of \(R\).
    	\item Similarly, again assuming that \(R\) is reduced
    		we may view the Frobenius as the inclusion
    		\(R^{p} \ins R\), 
    	\item More generally, we define \(F_{\star}R\) to
			be the \(R\)-algebra with ring structure 
			the same as \(R\), with module structure
			\(r \cdot x = F(r)x = r^{p}x\).
			Then we view the Frobenius as a map
			\(R \xrightarrow{} F_{\star}R\).
			In this description, \(F\) is a module 
			homomorphism as well. 
			The module \(F_{\star}R\) corresponds to the
			pushforward construction from geometry
			(i.e. pushforward of quasi-coherent
			sheaves on \(\Spec R\)).
			Even more generally, for any \(R\)-module
			\(M\) we denote \(F_{\star}M\) to be
			the analogously defined pushforward by
			Frobenius.
    \end{enumerate}
\end{rmk}

\begin{defn}
	We say that \(R\) is \(F\)-split if the map \(F\) is
	split as a map of 
	\(R\)-modules \(R \xrightarrow{} F_{\star}R\).
\end{defn}

We will be cheifly concerned with \textit{hypersurfaces},
so we specialize to this case now.

\begin{defn}
	Let \(S = k[x_{1}, \ldots, x_{n}]\).
	Then \(f \in S\) is \(F\)-split if 
	\(S / (f)\) is.
\end{defn}

A fundamental fact about \(F\)-splitness is that there
exists a very concrete criterion for whether or 
not a polynomial (hypersurface) \(f\) is \(F\)-split.
First, though, we introduce some notation.
If \(I = (x_{1}, \ldots, x_{n})\) is a finitely generated 
ideal of some ring \(R\), 
then \(I^{[m]}\)
is defined to be \((x_{1}^{m}, \ldots, x_{n}^{m})\)

\begin{thm}
	[Fedder's Criterion]
	Let \(f \in S = k[x_{1}, \ldots, x_{n}]\).
	Let \(\mathfrak{m} = (x_{1}, \ldots, x_{n})\) 
	be the ideal generated by the variables.
	Then \(f\) is \(F\)-split if and only if 
	\(f^{p-1} \notin \mathfrak{m}^{[p]}\)
\end{thm}

\begin{proof}
	See \cite[Theorem~2.5]{ma-polstra-2021-F-sing-comm-alg}
\end{proof}

%\begin{rmk}
%	Fedder's criterion is the second place where raising
%	homogeneous polynomials to powers enters the picture
%	algorithmically.
%	This time, unlike Remark \ref{rmk:polyraise:w2},
%	the raising to a power happens in characteristic \(p\).
%\end{rmk}

Particular interest is in the case when \(f\) is 
homogeneous of degree \(n\), which geometrically
corresponds to a \textit{Calabi-Yau} hypersurface.
In this case, we have

\begin{thm}
	\label{thm:fsplit:ordinary}
	If \(f \in S\) 
	homogeneous of degree \(n\) 
	is \(F\)-split,
	then the Artin-Mazur height 
	of \(Z(f) = \Proj (S / (f))\) 
	is 1, i.e. \(Z(f)\) is weakly ordinary.
\end{thm}

Recently, in \cite{yobuko-2019-qfs-calabi-yau}
Yobuko introduced the notion of 
quasi-\(F\)-splitness, which generalizes \(F\)-splitness.

\begin{defn}
	The ring \(R\) is \(n\)-quasi-\(F\)-split if there exists
	a map \(\phi \colon W_{n}(R) \xrightarrow{} R\) such that
	\[
	\begin{tikzcd}
		W(R) \arrow{r}{F} \arrow{d}[swap]{} &
		F_{\star}W(R) \arrow{ld}{\phi} \\
	R 
	\end{tikzcd}
	,\]
	where the vertical map is the \(1\)-st Witt vector truncation.

	The quasi-\(F\)-split height of \(R\) is the smallest \(n\) 
	for which \(R\) is \(n\)-quasi-\(F\)-split.

	As above, the quasi-\(F\)-split height of 
	\(f \in S = k[x_{1}, \ldots, x_{n}]\) is that
	of \(R = S / (f)\).
\end{defn}

\begin{rmk}
	Both \(F\)-splitness and quasi-\(F\)-splitness have 
	various geometric
	variants which are more general then the 
	ring-theoretic/affine versions given here. 
	These are covered extensively in the literature, for example
	see \cite{kttwyy-2022-qfs-birat}.
\end{rmk}

The quasi-\(F\)-split height also gives a generalization
of Theorem \ref{thm:fsplit:ordinary}:

\begin{thm}
	If \(f \in S\) is homogenous of degree \(n\),
	then the Artin-Mazur height of \(Z(f)\)
	is equal to the quasi-\(F\)-split height
	of \(f\).
\end{thm}

\begin{proof}
	This is a special case of 
	\cite[Theorem~4.5]{yobuko-2019-qfs-calabi-yau}.
\end{proof}

\section{Fedder's criterion for quasi-F-splitness: an algorithmic perspective}

We now describe how the Witt Vectors can be used to calculate the 
quasi-F-split height / Artin-Mazur height of a Calabi-Yau 
hypersurface, using the version of Fedder's criterion in \cite{kty-2022-fedder}.
We will describe the algorithm in more detail, proofs can be
found in \cite{kty-2022-fedder}.

\subsection{The computation of \(\Delta_{1}\)}

For the following discussion, 
let \(k\) be a perfect field of characteristic \(p\) and 
let \(S := k[x_{1}, \ldots, x_{n}]\).
Let \(f = \sum_{I}^{} a_{I}\mathbf{x}^{I}\) be a polynomial in \(S\).

\begin{defn}
	Let \(\Delta_{1}(f)\)
	be defined by the following equation in \(W_{2}(S)\) :
	\[
		(0, \Delta_{1}(f)) = (f,0) - \sum_{I}^{} (a_{I}\mathbf{x}^{I}, 0) 
	\] 
\end{defn}

The following proposition demonstrates how we can calculate 
\(\Delta_{1}(f)\) algorithmically.

\begin{prop}
	\label{prop:delta1:formula}
	Let \([f]\) be a lift of \(f\) to the integers,
	i.e.
	\(f = \sum_{I}^{} [a_{I}] \mathbf{x}^{I}\).
	If \(k = \mathbb{F}_{p}\),
	we can compute \(\Delta_{1}(f)\) by 
	taking the reduction of
	\[
		\frac{[f]^{p} - \sum_{I}^{} ([a_{I}]x^{I})^{p} }{p}
	\] 
	mod \(p\).
\end{prop}

\begin{proof}
	Iteratively apply the formula of the first
	Witt polynomial \(S_{1}\) 
	to the monomials of \(f\).
\end{proof}

\begin{rmk}
	If \(f\) is a homogeneous polynomial of degree \(d\), 
	then \(\Delta_{1}(f)\) is a polynomial of degree \(pd\).
\end{rmk}

Proposition \ref{prop:delta1:formula} gives an obvious algorithm
for calculating the term \(\Delta_{1}(f)\): 

\begin{algorithm}[H]
\caption{Calculation of \(\Delta_{1}(f)\) }
	\label{alg:calc:delta1}
    \begin{algorithmic}[1]
		\State \textbf{Input:} \(f \in \mathbb{F}_{p}[x_{1}, \ldots, x_{n}]\) 
		\State \(\tilde{f} \gets \text{lift}(f)\) 
		\State \(D \gets \tilde{f}^{p}\) 
		\For{\(t \in \text{terms}(\tilde{f})\) }
		    \State \(D \gets D - t^{p}\) 
		\EndFor
		\State \(D \gets D / p\)
		\State \Return \(D \% p\) 
	\end{algorithmic}
\end{algorithm}

\subsection{Splittings of Frobenius from a computational perspective}

Let \(S = k[x_{1}, \ldots, x_{n}]\) as before. 
We will use perspective (3) from
Remark \ref{rmk:frob:perspectives}; 
recall that we idenfity 
\(S\) with the target of frobenius and
\(S^{p}\) with the source.
We see 
by counting degrees
that we have a generating set for \(S\) as an
\(S^{p}\)-module given by
all monomials
\(x_{1}^{i_{1}}\ldots x_{n}^{i_{n}}\)
where \(0 \leq i_{j} \leq p-1\) for all \(j\).
Moreover, since \(S\) is a polynomial ring, 
there are no (module-theoretic) relations
and
\(S\) is a the free \(S^{p}\)-module generated by 
these monomials, i.e.  \[
S = \bigoplus_{1 \leq j \leq n, 0 \leq i_{j} \leq p-1}^{} x_{1}^{i_{1}}\ldots x_{n}^{i_{n}} S^{p}
.\] 
Then the projection to any of the direct sum components
is an element of \(\Hom(R,R^{p})\) is in fact a
splitting of Frobenius.
Let \(u\) be the projection onto the component of
\(x_{1}^{p-1}, \ldots, x_{n}^{p-1}\).

The splitting \(u\) plays an important role in \(F\)-singularity
theory, see for example 
\cite[Claim~2.6]{ma-polstra-2021-F-sing-comm-alg}.
For our purposes, we only care about computing \(u\) 
for a polynomial in \(S\). 
Given \(f \in S\), we will first compute
\(u(f) \in S^{p}\), and then use the identifiaction
\(S^{p} \isom S\) by taking \(p\)-th roots of 
exponents.

\begin{algorithm}[H]
\caption{Splitting of Frobenius}
\label{alg:naive:u}
\begin{algorithmic}[1]
\State \textbf{Input:} \(f\) 
\State \textbf{Output:} \(u(f)\) considered as an element of \(S\).
\State Discard all terms of \(f\) whose exponents are not congruent to
	\((p-1, \ldots, p-1)\) mod p.
\State result \(\gets 0\)	
\For{\(t \in \text{terms}(f)\) }
    \State subtract \(p-1\) from the exponent of 
	    \(t\) with respect to all variables.
    \Comment{All exponents in \(t\) are now divisible by \(p\) }
    \State Divide all exponents of \(t\) by \(p\).
    \State result \(\gets \text{result} + t\)
\EndFor 
\State \Return result
\end{algorithmic}
\end{algorithm}


\subsection{The naive algorithm}

Let \(f\) be a homogeneous polynomial of degree \(n\) 
in \(S\), 
so that \(Z(f)\) is a Calabi-Yau hypersurface.
Following \cite{kty-2022-fedder}, we have the following
algorithm to calculate the quasi-F-split height.

\begin{algorithm}[H]
\caption{Quasi-\(F\)-Split Height: naive algorithm}
\label{alg:qfs:naive}
\begin{algorithmic}[1]
\State Inputs: \(b \in \mathbb{N}\) chosen bound, \(f\) a homogeneous
	polynomial of degree \(n\) in \(S\).
\State \(g \gets f^{p-1}\) 
\If{\(g \not\in \mathfrak{m}^{p}\) }
    \State \Return \(1\) 
\EndIf
\State \(\Delta \gets \Delta_{1}(f^{p-1})\) \Comment{Use Algorithm \ref{alg:calc:delta1}}
\State \(n \gets 2\) 
\While{true}
    \If{\(b < n\)}
        \State \Return \(\infty\)
    \EndIf
    \State \(g \gets u(\Delta * g)\) \Comment{Use Algorithm \ref{alg:naive:u}}
    \If{\(g \notin \mathfrak{m}^{p}\)}
        \State \Return \(n\) 
    \EndIf
    \State \(n \gets n + 1\)
\EndWhile
\end{algorithmic}
\end{algorithm}

%\begin{alg}
%	\label{alg:qfs:outline}
%
%	\begin{enumerate}[(1)]
%		\item Compute \(f^{p-1}\)
%		\item If \(f^{p-1} \notin \mathfrak{m}^{p}\), return \(1\). 
%		\item Compute \(\Delta_{1}(f^{p-1})\) via Proposition \ref{prop:delta1:formula}.
%		\item Set \(n = 2\) and \(g = f^{p-1}\)
%		\item Loop
%			\begin{enumerate}[(1)]
%				\item \(g := \theta(g)\) (using the output of Step 3)
%				\item If \(g \notin \mathfrak{m}^{p},\) return \(n\).
%				\item If \(n < b\), return \(\infty\)	
%			\end{enumerate}
%	\end{enumerate}
%\end{alg}

\begin{thm}
	[\cite{kty-2022-fedder}, Theorem C]
	Assume that
	\(Z(f)\) has quasi-F-split height \(h < b\),
	then Algorithm \ref{alg:qfs:naive} terminates
	and returns \(h\).
\end{thm}

\begin{proof}
	This is just rephrasing \cite{kty-2022-fedder}, Theorem C.
\end{proof}

In the case of Calabi-Yau hypersurfaces, we have bounds on the height by
\cite{van-der-geer-katsura-2003-calabi-yau},
so we can deterministically recover the height.  
See also \cite[Theorem~0.1]{artin-1974-k3-surfaces},
for the case of K3 surfaces.
For a K3 surface, the height (if finite) is bounded by 10.

\subsection{The key idea: finding the matrix of the linear operator ``multiply then split''}

A naive implmentation of Algorithm \ref{alg:qfs:outline}
is provided in MMPSingularities.jl.
The bottleneck ends up being polynomial multiplication, 
in two places:
\begin{enumerate}[(1)]
    \item raising \(f^{p-1}\) to the \(p\)-th power in the integers
        (in the computation of \(\Delta_{1}\))
    \item multiplying \(g\) by \(\Delta_{1}(f^{p-1})\) 
        (in line 9 of Algorithm \ref{alg:qfs:naive})
\end{enumerate}

For a quartic K3 surface of characteristic \(5\),
for example, each multiplication take about 1 second 
using FLINT. 

%With that, we may define \(\theta\) as it appears
%in \cite{kty-2022-fedder}:
%
%\begin{defn}
%	[See the introduction of \cite{kty-2022-fedder}]
%	The map \(\theta = \theta_{f}\) is defined by 
%	\(\theta_{f}(g) = u(\Delta_{1}(f^{p-1}) \cdot g)\).
%	We generally omit the dependence on \(f\) from
%	the notation.
%\end{defn}

We now explain how to overcome the second
bottleneck. We observe that 
since \(Z(f)\) is Calabi-Yau
(i.e. \(\deg f = n\)), we have that \(f^{p-1}\) has degree
\(n(p-1)\). 
Furthermore, we observe by Proposition \ref{prop:delta1:formula}
that the degree of \(\Delta_{1}(f^{p-1})\) 
is \(np(p - 1)\).
Thus, \(\delta_{1}(f^{p-1})g\) has degree
\(n(p^{2} - 1)\); however, the effect 
of \(u\) on any polynomial is subtracting \(p-1\)
from and dividing by \(p\) 
(see Algorithm \ref{alg:naive:u}).
Thus, the map 
\(g \mapsto u(\Delta_{1}(f^{p-1}) g)\) 
is an (evidently linear) map from 
the space of homogenous polynomials of degree \(n(p-1)\) 
to itself.
As a consequence of this observation,
if we can efficiently compute the matrix of 
\(g \mapsto u(\Delta_{1}(f^{p-1})g)\)
we can repeatedly apply matrix-vector multiplication.

Furthermore, when \(g\) is written as a vector 
in the basis of homogeneous monomials of degree
\(n*(p-1)\), we can test if \(g \notin \mathfrak{m}^{[p]}\) 
in an especially simple way: the only 
monomial that is not in \(\mathfrak{m}^{[p]}\) is
\(x_{1}^{p-1}\cdots x_{n}^{p-1}\), 
see for example \cite{kty-2022-fedder}.
Thus, we can check if a single element of the vector representing
\(g\) is nonzero.

The algorithm for Fedder's criterion then becomes:

\begin{algorithm}[H]
\caption{Quasi-\(F\)-Split Height: matrix-based algorithm}
\label{alg:qfs:matrix}
\begin{algorithmic}[1]
\State Inputs: \(b \in \mathbb{N}\) chosen bound, \(f\) a homogeneous
	polynomial of degree \(n\) in \(S\).
\State \(g \gets f^{p-1}\) 
\If{\(g \not\in \mathfrak{m}^{p}\) }
    \State \Return \(1\) 
\EndIf
\State \(\Delta \gets \Delta_{1}(f^{p-1})\) \Comment{Use Algorithm \ref{alg:calc:delta1}}
\State \(M \gets\) the matrix of  \(g^{\prime} \mapsto u(\Delta * g^{\prime})\)
\State \(n \gets 2\) 
\State \(g_{v} \gets\) the representation of \(g\) as a vector
\State \(i \gets \) the index of the monomial \(x_{1}^{p-1}\cdots x_{n}^{p-1}\)
\While{true}
    \If{\(b < n\)}
        \State \Return \(\infty\)
    \EndIf
    \State \(g_{v} \gets M * g_{v}\) 
    \If{\(g[i] \neq 0\)}
        \State \Return \(n\) 
    \EndIf
    \State \(n \gets n + 1\)
\EndWhile
\end{algorithmic}
\end{algorithm}

Thus, we have reduced our problem (algorithmically, at least)
to finding the matrix of a ``mulitply then split'' operation.
In the following section, we give three such algorithms.
One is a naive algorithm, which is only 
reasonably fast if used with GPU hardware. 
The second algorithm is suitable for use on CPUs,
though isn't easily massively parallel.
Finally, the third algorithm 
outperforms the first two on the CPU and the GPU,
and is suitable for use on a cluster. 
However, it has the cost of needing to pregenerate some
data, which takes about 5 seconds.

In practice, Algorithm \ref{alg:qfs:matrix} is bottlenecked
by the polynomial multiplication in the calculation of 
\(\Delta_{1}(f^{p-1})\), as noted above. 
In order to get enough performance for our computations,
we implement GPU-accelerated versions of
matrix multiplication mod p and polynomial multiplication
over \(\mathbb{Z}\).

