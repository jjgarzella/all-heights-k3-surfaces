\documentclass[a4paper]{article}

%% Language and font encodings
\usepackage[english]{babel}
\usepackage[utf8x]{inputenc}
\usepackage[T1]{fontenc}

%% Sets page size and margins
\usepackage[a4paper,top=3cm,bottom=2cm,left=3cm,right=3cm,marginparwidth=1.75cm]{geometry}

%% Useful packages
\usepackage{mathptmx,amsmath,colonequals}
\usepackage{amssymb,gensymb,mathrsfs}
\usepackage{amscd,amsthm}
\usepackage{mathtools}
\usepackage{graphicx}
\usepackage[colorinlistoftodos]{todonotes}
\usepackage[colorlinks=true, allcolors=blue]{hyperref}
\usepackage[shortlabels]{enumitem}
\usepackage{extarrows}
\usepackage{tikz,tikz-cd}
\usepackage[ruled,linesnumbered]{algorithm2e}
\usepackage{listings}
\usepackage{nicefrac}
\usepackage{array}
\usepackage{booktabs}
% I think this is the solution to the pagebreaking
% algorithms
\usepackage{float}

% algorithm2e stuff
\SetKwInput{KwInput}{Input}
\SetKwInput{KwNotation}{Notation}
\SetKwInput{KwOutput}{Output}
\DontPrintSemicolon
\newcommand{\code}[1]{\textnormal{\texttt{#1}}}

\newtheorem{thm}{Theorem}[section]
\newtheorem{lem}[thm]{Lemma}
\newtheorem{cor}[thm]{Corollary}
\newtheorem{prop}[thm]{Proposition}
\newtheorem{conj}[thm]{Conjecture}
\newtheorem{alg}[thm]{Algorithm}
\newtheorem{warn}[thm]{Warning}
\newtheorem{cxt}[thm]{Context}
\newtheorem{heur}[thm]{Heuristic}
\newtheorem{nota}[thm]{Notation}
\newtheorem{quest}[thm]{Question}
\theoremstyle{definition}
\newtheorem{defn}[thm]{Definition}
\newtheorem{ex}[thm]{Example}
\newtheorem{xca}[thm]{Exercise}
\theoremstyle{remark}
\newtheorem{rmk}[thm]{Remark}

\newcommand{\arxiv}[1]{\href{http://arxiv.org/abs/#1}{{\tt arXiv:#1}}}

% convenient renaming
\newcommand{\isom}{\cong}
\newcommand{\ins}{\subset}
\newcommand{\dual}{\vee}
\newcommand{\cross}{\times}

% general text operators
\newcommand{\Image}{\operatorname{Im}}
\newcommand{\Coim}{\operatorname{Coim}}
\newcommand{\Ker}{\operatorname{Ker}}
\newcommand{\Coker}{\operatorname{Coker}}
\newcommand{\colim}{\operatorname{colim}}

% arrows
\newcommand{\surj}{\twoheadrightarrow}
\newcommand{\inj}{\xhookrightarrow{}}

% context
\newcommand{\basefield}{K}
\newcommand{\basering}{R}

% algebra
\newcommand{\category}[1]{\textsf{#1}}
\newcommand{\Hom}{\operatorname{Hom}}
\newcommand{\Ext}{\operatorname{Ext}}
\newcommand{\Tor}{\operatorname{Tor}}
% TODO: put a \newcommand for the characteristic

% commutative algebra
%\newcommand{\depth}{\operatorname{depth}}
%\newcommand{\height}{\operatorname{height}}
%\newcommand{\pd}{\operatorname{pd}}
%\newcommand{\injd}{\operatorname{injd}}

% algebraic geometry
\newcommand{\Spec}{\operatorname{Spec}}
\newcommand{\Spa}{\operatorname{Spa}}
\newcommand{\Proj}{\operatorname{Proj}}

\newcommand{\triv}{\texttt{TRIV}}
\newcommand{\merge}{\texttt{MERGE}}
\newcommand{\wicsalg}{\texttt{WICS}}

% custom ops here
\newcommand{\exps}{\operatorname{exps}}
\newcommand{\wics}{\operatorname{wics}}
\newcommand{\coeff}{\operatorname{coeff}}
\newcommand{\length}{\operatorname{length}}
\newcommand{\indexof}{\operatorname{indexof}}


\newcommand{\modp}{\; (\text{mod} \; p)}
\newcommand{\mathmod}{ \text{ mod }}


% custom commands here
\newcommand\bsfrac[2]{\reflectbox{\nicefrac{\reflectbox{$#1$}}{\reflectbox{$#2$}}}}

\title{K3 surfaces of any Artin--Mazur height over \(\mathbb{F}_{5}\) and 
    \(\mathbb{F}_{7}\) via 
quasi-\(F\)-split singularities and GPU acceleration}
\author{Ryan Batubara, Jack J. Garzella, Alex Pan}

\begin{document}

\maketitle

\begin{abstract}
    \noindent We develop a fast algorithm to calculate the Artin--Mazur height (equivalently, the
    quasi-\(F\)-split height) of a Calabi-Yau hypersurface, building on the work in
    \cite{kty-2022-fedder}. 
    We provide an implementation of our approach, and use it to show that there are quartic
    K3 surfaces of any Artin--Mazur height over \(\mathbb{F}_{5}\) and \(\mathbb{F}_7\).
\end{abstract}


\section{Introduction}




Throughout the paper, we do various timing tests to compare various
approaches for each computational step.
All timign tests were done with 
an Intel i5-8400 CPU and a Nvidia GeForce RTX 3070 GPU.


\section{Preliminaries: the quasi-\(F\)-split height and Fedder's criterion}
\label{sec:preliminaries}


\subsection{The Witt Vectors}

TODO: at a polite introduction to the Witt Vectors here

* the Witt vectors
* the truncated Witt vectors

\subsection{Splittings of Frobenius}

* the u map
* colon ideals
* fedder's criterion

\subsection{An algorithmic description of Fedder's criterion}

We now describe how the Witt Vectors can be used to calculate the 
quasi-F-split height / Artin-Mazur height of a Calabi-Yau 
hypersurface, following \cite{quasifedder}.
We will describe the algorithm in more detail, while proofs can be
found in \cite{quasifedder}.

\subsubsection{\(\Delta_{1}\)}

For the following discussion, 
let \(k\) be a perfect field of characteristic \(p\) and 
let \(S := k[x_{1}, \ldots, x_{n}\).
Let \(f = \sum_{I}^{} a_{I}\mathbf{x}^{I}\) be a polynomial in \(S\).

\begin{defn}
	Let \(\Delta_{1}(f)\)
	be defined by the following equation in \(W_{2}(S)\) :
	\[
		(0, \Delta_{1}(f)) = (f,0) - \sum_{I}^{} (a_{I}\mathbf{x}^{I}, 0) 
	\] 
\end{defn}

The following proposition demonstrates how we can calculate 
\(\Delta_{1}(f)\) algorithmically.

\begin{prop}
	\label{prop:delta1:formula}
	Let \([f]\) be a lift of \(f\) to the integers,
	i.e.
	\(f = \sum_{I}^{} [a_{I}] \mathbf{x}^{I}\).
	If \(k = \mathbb{F}_{p}\),
	we can compute \(\Delta_{1}(f)\) by 
	taking the reduction of
	\[
		\frac{[f]^{p} - \sum_{I}^{} ([a_{I}]x^{I})^{p} }{p}
	\] 
	mod \(p\).
\end{prop}

\begin{proof}
	Unwind the definition of addition in the Witt Vectors
\end{proof}

\begin{rmk}
	If \(f\) is a homogeneous polynomial of degree \(d\), 
	then \(\Delta_{1}(f)\) is a polynomial of degree \(pd\).
\end{rmk}

\subsection{The algorithm}

Let \(f\) be a homogeneous polynomial of degree \(n\) 
in \(S\). 
Then \(Z(f)\) is a Calabi-Yau hypersurface.

Let \(u\) be the generator of \(\Hom(S,F_{\star}S)\) from the 
previous section.
Define \(\theta\) by  (INSERT HERE BASED ON PREV SECTION NOTATION)

\begin{alg}
	\label{alg:qfs:outline}

	Inputs: \(b \in \mathbb{N}\) chosen bound, \(f\) a homogeneous
	polynomial of degree \(n\) in \(S\).
	\begin{enumerate}[(1)]
		\item Compute \(f^{p-1}\)
		\item If \(f^{p-1} \notin \mathfrak{m}^{p}\), return \(1\). 
		\item Compute \(\Delta_{1}(f^{p-1})\) via Proposition \ref{prop:delta1:formula}.
		\item Set \(n = 2\) and \(g = f^{p-1}\)
		\item Loop
			\begin{enumerate}[(1)]
				\item \(g := \theta(g)\) (using the output of Step 3)
				\item If \(g \notin \mathfrak{m}^{p},\) return \(n\).
				\item If \(n < b\), return \(\infty\)	
			\end{enumerate}
	\end{enumerate}
\end{alg}

\begin{theorem}
	[\cite{quasifedder}, Theorem C]
	Assume that
	\(Z(f)\) has quasi-F-split height \(h < b\),
	then Algorithm \ref{alg:qfs:outline} terminates
	and returns \(h\).
\end{theorem}

\begin{proof}
	This is just rephrasing \cite{quasifedder}, Theorem C.
\end{proof}

In the case of Calabi-Yau hypersurfaces, we have bounds on the height by
(INSERT CITATION OF THE RIGHT VAN DER GEER AND KATSURA PAPER, THERE ARE FOUR OPTIONS), 
so we can deterministically recover the height.  

\subsubsection{Our improvements}

A naive implmentation of Algorithm \ref{alg:qfs:outline}
is provided in MMPSingularities.jl.
The bottleneck ends up being polynomial multiplication, 
in two places:
raising \(f^{p-1}\) to the \(p\)-th power in the integers,
and multiplying \(g\) by \(\Delta_{1}(f^{p-1})\) 
in \(\theta\).

For the first, we implement a gpu-based FFT that is
further optimized for homogeneous inputs.
For the second, we observe that \(f^{p-1}\) has degree
\(n(p-1)\). 
Furthermore, we observe that by Remark \ref{} and 
(REF FSPLIT SECTION), \(\theta\) is a linear map
from the vectors space of 
homogenous polynomials of degree \(n(p-1)\) 
to itself.
Thus, if we can efficiently compute the matrix of \(\theta\),
we can repeatedly apply matrix-vector multiplication,
which is much faster.




\section{Fast quasi-\(F\)-split Fedder's criterion algorithms}
\label{sec:alg:momts}


Principles: 

* naive gpu algorithm is fast enough
* naive gpu algorithm can be faster than
a ``clever'' cpu algorithm with better complexity

%* comment on the general problem of multiply then split
%* TODO: rephrase everything in terms of the general problem, without 
%reference to delta_1

\begin{cxt}
	\label{cxt:mult:split}
	Let \(\Delta \in S = k[x_{1}, \ldots, x_{n}]\)
	be a homogeneous
	polynomial of degree \(D\).
	In algorithms, we will sometimes conflate \(\Delta\) 
	with a list containing all its terms.
	For \(d\) an integer,
	and let \(S_{d}\) denote the homogeneous polynomials of degree \(d\).
	The dimension of \(S_{d}\) is then \(\binom{n+d-1}{n-1}\).
	Thus, the number of terms of \(\Delta\) is bounded 
	above by \(\binom{n+D-1}{n-1}\).
	We let \(M_{d}\) denote a list (in lexographical order)
	of the monomial basis of \(S_{d}\).
	Let \(d^{\prime} \colonequals \frac{d+D-n-1}{p}\).
	Then 
	We wish to have an algorithm that calculates the matrix of
	the linear operator
	\begin{align*}
		\theta : S_{d} &\longrightarrow S_{d^{\prime}} \\
		g &\longmapsto u(\Delta g)
	.\end{align*}
\end{cxt}

\begin{rmk}
	Note that if  \(d^{\prime} = \frac{d+D-p(n-1)}{p}\)
	is not an integer, \(\theta\) is the zero map
	and ther is nothing more to do.
	
	If \(\Delta = \Delta_{1}(f^{p-1})\) and \(d = n(p-1)\),
	then \(d^{\prime} = d\) and \(\theta\) is
	represented by
	a square matrix.
\end{rmk}

\begin{defn}
	Let \(m\) be a monomial in \(S\). 
	Then we say that \(m\) \textbf{matches}
	with another monomial \(m^{\prime}\) 
	if the monomial \(mm^{\prime}\) is
	not killed by \(u\).
\end{defn}

In the case of calculating the quasi-\(F\)-split height of a Calabi-Yau
hypersurface,
we have \(d = n(p-1)\) and \(D = np(p-1)\).

Notes for complexity calculations:

TODO: change the algorithms to reflect that the input and output space
might not be the same vector space in general.

\subsection{The Naive Algorithm}

\begin{algorithm}[H]
    \caption{Matrix of $\theta$: Trivial Algorithm}
    \label{alg:matrix:trivial}
    \begin{algorithmic}[1]
    \State \textbf{Input}: $\Delta, M_{d}, M_{d^{\prime}}, p$
    \State \textbf{Output}: $length(M_{d^{\prime}}) \times length(M_{d})$ 
	matrix representing $\theta$
    \State mat $\gets zeros(length(M_{d^{\prime}}), length(M_{d}))$
    \For{$m \in M_{d}$}
	    \State $c \gets indexof(m, M _{d})$
        \For{$\delta \in \Delta$}
            \If{$(exponents(\delta) .+ exponents(m)) .\% p =[p-1, \dots, p-1]$} 
			\State res $\gets (exponents(\delta) + exponents(m) - (p-1, \ldots, p-1)) / p$
				\State $r \gets indexof(\text{res},M_{d^{\prime}})$
                \State $\text{mat}[r, c] += coeff(\delta)$
            \EndIf
        \EndFor
    \EndFor
    \State \Return mat
    \end{algorithmic}
\end{algorithm}

In this algorithm, we iterate through the monomials $m \in M_{d}$, each corresponding to a column in the
resulting matrix. For each monomial, we search for terms that match $\delta \in \Delta$, apply $u$ to their product $\delta u$, and 
get the lexographical index of the result to find which row to add to.

Since $length(M_{d}) = \binom{n + d - 1}{n - 1}$ and 
$length(\Delta) = \binom{n + D - 1}{n - 1}$, the matrix can be generated in 
$O(\binom{n + d - 1}{n - 1}\binom{n + D - 1}{n + d - 1})$ operations. 
This can be
seen from the nested loops, assuming that integer arithmetic operations are constant time.
In the case of the quasi-\(F\)-split height of a K3 surface, 
this becomes
\(O(\binom{np - 1}{n-1}\binom{n(p^2 - p + 1)}{n-1})\).

In practice, the majority of the combinations of terms of $\Delta$ and $M_{d}$ don't match. This means in
Algorithm \ref{alg:matrix:trivial}, much of our runtime is wasted on checking for whether terms match. 
However, we emphasize that this algorithm, if implemented in parallel on the GPU, is indeed
fast enough to not be a bottleneck in practice.

\subsection{Modified Merge-based Algorithm}

To find the matrix of \(\theta\), mathematically
speaking we must first multiply by \(\Delta_{1}(f)\) 
and then apply the map \(u\).
Our first observation is that naively 
multiplying and then applying \(u\) 
as in algorithm \ref{alg:naive:u}
does a lot of unnecessary work.
In particular, any term in the 
product \(\Delta_{1}(f)a\) which 
is not congruent to 
\((p-1, \ldots, p-1) \mod p\)
is not needed. 
Even if \(a\) is a monomial, giving a linear
algoritm for multiplication, using naive
multiplication stores a lot of unnecessary terms
in memory.
Here, we give an algorithm that calculates
the matrix of \(\theta\) by only
using such terms.
We start with an elementary fact.

\begin{lem}
	\label{lem:tuples:modp}
	Let \(\underbar{a} = (a_{1}, \ldots, a_{n}),
	\underbar{b} = (b_{1}, \ldots, b_{n})\)
	be in \(\{0, \ldots, p-1\}^{n} \ins \mathbb{Z}^{n}\).

	Then \(\underbar{a} + \underbar{b} = 
	(p-1, \ldots, p-1) \mod p\)
	if and only if 
	\(\underbar{a} + \underbar{b} = 
	(p-1, \ldots, p-1)\).
\end{lem}

\begin{proof}
	Each coordinate has \(a_{i} + b_{i} \leq 2(p-1) = 2p-2\).
\end{proof}


\begin{cor}
	Let \(\underbar{a}\) be as above.
	Let \(\underbar{x}^{\underbar{a}}\)
	denote \(x_{1}^{a_{1}} \ldots x_{n}^{a_{n}}\).
	Then there exists a unique
	match \(m(\underbar{a}) \in \{0, \ldots, p-1\}^{n}\) 
\end{cor}

\begin{proof}
	\(m(\underbar{a}) = 
	(p-1, \ldots, p-1) - \underbar{a}\).
	The claim follows from 
	\ref{lem:tuples:modp}.
\end{proof}

Essentially, this corollary says that
when we consider arbitrary monomials \(M\),
when we consider its exponent vector 
mod \(p\), it has a unique
match mod \(p\).
However, we have even more:

\begin{cor}
	\label{cor:match:order}
	Let \(\leq_{lex}\) denote the 
	lexographical ordering.
	If \(\underbar{a} \leq_{lex} \underbar{b}\),
	then 
	\(m(\underbar{b}) \leq_{lex} m(\underbar{a})\)
\end{cor}

\begin{proof}
	Follows from the definition of lexographical
	ordering.
\end{proof}

The previous two corollaries justify
the following algorithm.

\begin{algorithm}
\caption{Multiply than split: merge-based algorithm}
\label{alg:theta:merge}
\begin{algorithmic}[1]
\State \textbf{Inputs:} \(\Delta, M_{d}, M_{d^{\prime}}, p\)
\State mat \(\gets zeros(length(M), length(M))\) 
\State \(L \gets exponents(\Delta) .\% p\) 
\State \(R \gets exponents(M_{d}) .\% p\)
\State sort both  \(L\) and \(R\) 
\State \(\Delta^{\prime}, M^{\prime} \gets \)permute \(\Delta\) and \(M\) by the sort permutations as \(L\) and \(R\).
\State \(l \gets 1, r \gets legnth(R)\)
\While{\(1 \leq R \&\& l \leq length(L)\)}
	\State cmp \(\gets L[l] + R[r] - (p-1, \ldots, p-1)\) 
	\If{cmp < 0}
	    \State \(l \gets l + 1\) 
	\ElsIf{0 < cmp}
	    \State \(r \gets r - 1\)
    \ElsIf{0 == cmp}
        \State matchr \(\gets R[r]\) 
        \State numLmatches \(\gets\) the number of adjacent entries in \(L\) which are equal to \(L[l]\) 
        \While{\(R[r] == \) matchr \(\&\& 1 \leq r\)}
		    \For{\(ll \in (l, l+1, \ldots, l+\text{numLmatches})\)}
                \State res \(\gets (exponents(\Delta^{\prime}[ll]) .+ exponents(M^{\prime}[r]) .- (p-1, \ldots, p-1)) ./ p\) 
                \State coeff \(\gets coeff(\Delta^{\prime}[ll])\)
                \State \(col \gets indexof(M^{\prime}[r],M_{d})\)
                \State \(row \gets indexof(\text{res},M_{d^{\prime}})\)
                \State \(\text{mat}[row,col] += coeff\) 
            \EndFor
            \State \(r \gets r - 1\)
        \EndWhile     
        \State \(l \gets l + \text{numLmatches} - 1\) 
    \EndIf
\EndWhile
\State \Return mat
\end{algorithmic}
\end{algorithm}

%\begin{alg}
%
%\begin{lstlisting}
%
%delta1 = terms of delta1
%mons = all homogeneous monomials
%nMons = length(mons)
%
%answer := zeros(nMons,nMons)
%L := delta1 .% p
%R := M .% p
%
%sort both L and R (mod p)
%sort delta1 and M according to the same permutation
%
%l := 1, r := length(R)
%
%while 1 <= r && l <= length(R)
%    cmp := L[l] + R[r] - (p-1, ..., p-1)
%    switch cmp
%    case cmp < 0
%        l += 1
%    case 0 < cmp
%        r -= 1
%    case cmp == 0
%        matchr := R[r]
%        numLmatches := (the number of 
%            adjacent entries in L that are equal to L[l])
%        while R[r] == matchr
%            for ll in l:l+numLmatches
%
%                Extract delta1[ll] and its 
%                corresponding coefficient, as well as M[r]
%
%                Apply u to the product of the resulting monomials
%
%                Add this term to the correct entry of answer
%            end
%        end
%    end	
%end		
%
%\end{lstlisting}
%\end{alg}

Note that the inner while loops account for the fact that
after modding by \(p\), one does not expect 
either array to have unique keys.
The algorithm is justified by the previous corollaries.

In practice, instead of sorting the arrays in place, 
we use a method like julia's ``sortperm''
%TODO: code font
to get the sort permutation.

We also may use the kronecker substitution to make
the comparisons and additions use machine integers
rather than vectors.
In practice, this speeds the algorithm up a lot.

* complexity analysis
n = length L, m = length R
we need n + m + n log n + m log m operations.
Thus, the algorithm is big o of max(n log n, m log m),
and in practice the \(\Delta_{1}\) length
usually dominates.
TODO: state the complexity results in terms of the size of f

%In what follows, when we put a dot (\(.\))
%before an operation, that denotes
%componentwise application
%of that operation, similar to Julia syntax.

\subsection{Algorithm based on Generating weak integer compositions}

	Let \(M\) be a lexographically-ordered array containing the terms $m$ of the monomial basis of
	homogeneous polynomials of degree
	\(n*(p-1)\). Let $M_m$ denote the index of $m$ in $M$. Let $coeff(t)$ and $exps(t)$ denote the coefficient of monomial $t$ and exponent vector of $t$ respectively.

\begin{defn}
    $generateMatchingMonomials(\delta, p, M)$ produces a set of monomials $m \in M$ that match with $\delta$
\end{defn}

This function and its implementation are best explained with an example.

\begin{ex}
    Let $p = 5$, and $\delta$ be a monomial with degree sequence $[21, 16, 21, 22]$, where $a$ is some arbitrary coefficient. This is a term that may
    result from the computation of $\Delta_1(f^{p - 1})$ for a quartic K3 surface $f$ over $\mathbb{F}_5$. In this case, $M$ is an array
    of all possible $4$-variate monomials with homogeneous degree $16$. From [21, 16, 21, 22] we add [3, 3, 3, 2] to obtain [24, 19, 24, 24], which
    has all elements congruent to $p-1 \mod p$. Because our monomials are homogeneous of degree $16$, and
    we have only added $3 + 3 + 3 + 2 = 11$ to the degree sequence of $\delta$, we still have $16 - 11 = 5$ more to add.
    To maintain congruence to $p-1 \mod p$, we must add all $5$ to one select variable. With 4 variables to choose from, doing this gives the
    matching monomials $\{[8, 3, 3, 2], [3, 8, 3, 2], [3, 3, 8, 2], [3, 3, 3, 7]\}$.
\end{ex}

\begin{thm}
    generateMatchingMonomials() correctly generates all matching monomials (Should I give a full algorithm for the implementation and why each step works, or can I just remove this Theorem and leave it to the reader? The proof is just a bunch of number crunching which I have the components of commented out at the bottom of this file)
\end{thm}

With this, we have a faster version of our trivial algorithm:

\begin{algorithm}[H]
    \caption{Matrix of $\theta$: WICS Algorithm}
    \label{alg:matrix:WICS}
    \begin{algorithmic}[1]
    \State \textbf{Input}: $\Delta, M_{d}, M_{d^{\prime}}, p$
    \State \textbf{Output}: $length(M_{d^{\prime}}) \times length(M_{d})$ matrix representing $\theta$
    \State mat $\gets zeros(length(M_{d^{\prime}}), length(M_{d}))$
    \For{$\delta \in \Delta$}
        \State mons $\gets generateMatchingMonomials(\delta, p, M_{d})$
        \For{$m \in mons$}
		    \State $c \gets indexof(m,M_{d})$
			\State res $\gets (exponents(\delta) + exponents(m) - (p-1, \ldots, p-1)) / p$
			\State $r \gets indexof(\text{res},M_{d^{\prime}})$
            \State $\text{mat}[r, c] += coeff(\delta)$
        \EndFor
    \EndFor
    \State \Return mat
    \end{algorithmic}
\end{algorithm}

Again, letting $x = len(M)$ and $y = len(\Delta)$, this algorithm generates the matrix in $O(y)$ operations. The number of matching monomials each term $\delta$ has is bounded above by $\binom{\lfloor \frac{n(p - 1)}{p} \rfloor + n - 1}{n}$ (I have a proof but requires me to explain details of generateMatchingMonomials()), which is effectively constant for all $n$ and $p$ computable in the near future.

TODO: state the complexity stuff in terms of the size of f


%\begin{lstlisting}
%	
%	
%	\(answer \colonequals []\)
%	\(L \colonequals \Delta .\% p\) 
%	and \(R \colonequals M .\% p\).
%
%	Sort both \(L\) and \(R\) mod \(p\).
%	Sort \(\Delta\) and \(M\) according to the 
%	same permutation.
%
%	Let \(\ell \colonequals 1, r \colonequals \text{length}(R)\)
%
%	while \(1 \leq r \&\& l \leq \text{length}(L)\)
%
%	--- \(cmp \colonequals L[\ell] + R[r] - (p-1, \ldots, p-1)\)
%
%	--- switch \(cmp\)
%
%	--- --- case  \(cmp < 0\) 
%
%	--- --- --- \(l += 1\) 
%
%	--- --- case \(0 < cmp\) 
%
%	--- --- --- \(r -= 1\) 
%
%	--- --- case \(cmp == 0\) 
%
%	--- --- --- \(matchr \colonequals R[r]\) 
%
%    --- --- --- \(numLmatches \colonequals \) the number of
%	adjacent entrices in \(L\) that are equal to \(L[\ell]\)
%
%	--- --- --- while \(R[r] == matchr\) 
%
%	--- --- --- --- for \(\ell\ell\) in \(\ell\) to \(\ell\) + numLmatches
%
%	--- --- --- --- --- Extract \(\Delta[\ell\ell]\) 
%	and its corresponding coeffient,
%	as well as \(M[r]\)
%
%	--- --- --- --- --- Apply \(u\) to the product
%	of the resulting monomials
%
%	--- --- --- --- --- add this term to \(answer\)
%
%	--- --- --- ---  end for
%
%	--- --- --- --- \(r += 1\) 
%
%	--- --- --- end while
%
%	--- --- --- \(l += numLmatches\) 
%	
%	--- end switch
%
%	end while
%
%\end{lstlisting}



%* picture


% Given $d$ and $p$, we want to add to the coordinates of $d$ until we get some vector $d' \equiv (p - 1, \dots, p - 1) \mod p$. Because $d$ is a term of $\Delta_{1}(f ^ {p - 1})$, the sum of the coordinates of $d$ is $n \cdot (p - 1) \cdot p \equiv 0 \mod p$. Since $d' \equiv (p - 1, \dots, p - 1) \mod p$, the sum of the coordinates of $d'$ is congruent to $n \cdot (p - 1) \equiv -n \mod p$. This means that we will always add $-n \mod p$ to the coordinates of $d$ to get $d'$, and since $m \in M$ has homogeneous degree $n \cdot (p - 1) \equiv -n \mod p$, we have some multiple of $p$ to add to the coordinates of $d'$ to get a term of $\Delta_{1}(f ^ {p - 1})$. Let $m' = d' - d$: this is the foundation of the "relevant" monomial (sum of coordinates of $m'$ is congruent to $-n \mod p$). To actually get an element $m \in M$, we need to add a multiple of $p$ to each coordinate (probably needs a lemma). Since we need to add to $sum(d') \equiv -n \mod p$ from $sum(m') \equiv -n \mod p$, we arrive at the problem of adding $\frac{n * (p - 1) - c}{p}$ $p$'s to $m'$, which can be solved using weak integer compositions.










\section{Polynomial powering}

\subsection{Kronecker Substitution and Homogeneity}
In multivariate polynomial arithmetic, it is commonplace to map a multivariate polynomial to a univariate 
polynomial. This is because a comparison between two degree sequences $(d_1, \dots , d_n)$ of monomials 
$x_1^{d_1} \dots x_n^{d_n}$ is $n$ times slower than comparing two integers. One method of doing this is the Kronecker 
substitution:

\begin{defn}
    Let $M$ be a non-inclusive upper bound on the degree of any variable of the polynomial
    $f \in R[x_1, \dots, x_n]$. Then, the Kronecker Substitution produces a univariate polynomial $g \in R[z]$ by substituting powers of $z$ in 
    the evaluation of $f$: [cite this]{https://arxiv.org/pdf/1401.6694}

    \begin{center}
        $g(z) = f(z, z^M, z^{M^2}, \dots, z^{M^{n-1}})$
    \end{center}
\end{defn}

\noindent Additionally, because our polynomials are homogeneous, we can simply ignore one variable in all of our 
operations between monomials of the same degree, since the variable can be recovered by subtracting the 
known total degree of the monomial from the homogeneous degree.

\subsubsection{FFT}

For our algorithms of choice, we chose Monagan's powfps algorithm for computing $g = f^{p-1} \; (1)$, and the 
Fast Fourier Transform for computing $g^p \; (3)$. Monagan's powfps algorithm already computes $(1)$ in time 
on the order of milliseconds, so we focus more on the bottleneck, $(3)$. Even though the FFT isn't suited for 
sparse problems, the massive performance boost from GPU parallelization allows it to run faster than other 
algorithms, as discussed in section 3.3.

To determine the size of our FFT, we need to compute the largest possible degree that the Kronecker 
Substitution can map a monomial of $g^p$ to. Let $f \in R[x_1, \dots, x_n]$ be homogeneous with degree $d$. Then, 
$(f^{p-1})^p$ is homogeneous with degree $D = dp(p - 1)$. Then, by Definition 3.1, we can select an upper bound 
of $M = D + 1$, and the maximum degree the Kronecker Substitution will be able to map to is $D*M^{n - 1}$. But 
if we ignore one variable of $f$ due to the homogeneous property, we get a maximum degree of $D*M^{n - 2}$. 
Because we use a radix-2 FFT, we find the next power of 2 after $D*M^{n - 2} + 1$, the $+1$ coming from a 
$k$-length FFT representing a $k-1$ degree polynomial.

Below is a table of FFT lengths needed to compute $g^p$:


\begin{center}
    \textbf{Required FFT lengths to compute $g^p$}
    \begin{tabular}{c|c|c|c|c}
    $\bsfrac{\,p}{n}$ & 5 & 7 & 11 & 13 \\ \hline
    4 & 1,048,576 & 8,388,608 & 134,217,728 & 268,435,456 \\
    5 & 134,217,728 & 2,147,483,648 & 137,438,953,472 & 549,755,813,888
    \end{tabular}
\end{center}    

To make computing $\Delta_1(f^{p-1})$ even faster, we look for ways to reduce the FFT size. One such way is by imposing restrictions on the space of possible monomials. Consider the restriction where we remove all terms containing only one variable raised to the $n$-th power from the basis of monomials of $f$.

[Explanation on how the restriction affects kroneckersubstitution]

Below is a table of FFT lenghts needed to compute the restricted $g^p$:


\begin{center}
    \textbf{Required FFT lengths to compute restricted $g^p$}
    \begin{tabular}{c|c|c|c|c}
    $\bsfrac{\,p}{n}$ & 5 & 7 & 11 & 13 \\ \hline
    4 & 262,144 & 2,097,152 & 67,108,864 & 134,217,728 \\
    5 & 67,108,864 & 1,073,741,824 & 68,719,476,736 & 274,877,906,944
    \end{tabular}
\end{center}


\subsection{Finding the Maximum Coefficient}
In order to determine which moduli to perform our FFT in, we first need to obtain an upper bound on all of 
the coefficients of $g ^ p$.

To do this, we generate the maximum possible polynomial and throw it into OSCAR and raise it to the p and call it a day

\subsection{Other algorithms}
Many other algorithms are described in [CITE]. In this section we touch on them, and discuss why they
weren't used.

\subsubsection{RMUL and RSQR}
GPU parallelized versions even slower than CPU fps algorithm in OSCAR. (will benchmark later)

\subsubsection{BINA and BINB}
Also even slower than CPU fps algorithm (will also benchmark later)

\subsubsection{MNE}
Note that a basic implementation of MNE actually outperforms FPS for computing $g = f^{p - 1}$. However, 
since this step is already instantaneous compared to computing $g ^ p$, we use FPS because it doesn't require 
the hassle of pregeneration.

The memory consumption of MNE scales terribly with the number of input terms, meaning it isn't exactly suited for computing $g ^ p$. In fact, for the $(n, p) = (4, 5)$ problem, 1294 yottabytes are required to store the multinomial lookup table. This is 100 million times more than the entire internet. So, we cannot use this method.

% In order to generate a multinomial coefficient lookup table for $g^p$, we need to compute the most number of terms $g^p$ can have. Letting $n$ be the number of variables of $g$, this turns out to be the number of ways to distribute $n(p - 1)$ balls into $n$ bins, or $T = \binom{n(p - 1) + n - 1}{n - 1}$. 

\subsubsection{SUMS and FPS}
FPS, the more optimized version of SUMS, performs better than FFT at < 10 starting terms. However,
parallelization faces diminishing returns per thread added, to the point where the merge algorithm needed
will take longer than the CPU version. (in theory, I've been too lazy to actually implement it with GPU 
heap)

\section{Matrix multiplication mod \(p\)}

% cut section 5.1, replace with maybe one sentence about what we need to do
% section 5.2, probably shortened in light of edits made to the polynomial multiplication section
% section 5.2, paragraph 2, should mention the implied bit (I (JJ) can explain this if it isn’t clear)
% Everywhere that we say “mod” in this section should become “reduce” or “reduce mod p”
% Table: 
% needs a caption, should be “Table 1”
% I think the table is too big. For the paper, can we shave off the UInt8/Int8 rows and see how it looks?
% The “Largest Int” column should go from scientific notation to algebraic expressions (i.e. 2^32 - 1)
% The numerical columns should not use scientific notation, instead we can use \cdot !0^ (read: times ten to the).
% The table should have a heading that says “Maximum number of additions before overflow” or something to this effect.
% There should be a vertical line between “Largest Int” and the numerical columns
% Everywhere there’s a -1 it should be a blank or a dash
% The figure about max number of operations can go away. The table gives enough info
% the current text of section 5.3 should shrink to one paragraph that cites the places we used to learn the algorithm
% then we should put in the stuff about what we did with modding (er, reducing) here
% section 5.4 can be one sentence at the end of the paragraph that section 5.3 will become, it can just say that we implement a heuristic algorithm in the package
% section 5.5 can move to “computations”

\subsection{Maximum Operations under mod p}

We start by defining notation. Let $A$ be $m \times n$ and $B$ $n \times k$, and $N$ the modulus. One problem with matrix multiplication is the magnitude of each entry in the result $C$, which is at most $n L^2$, where $L$ is the largest entry in $A$ and $B$. When dealing matrices of size $m,n,k \geq 5000$, and large $L$, $n L^2$ may be larger than the numerical datatype used, and thus larger datatypes are necessary. 

In matrix multiplication mod $N$, the maximum size of any element in $C$ is $n (N-1)^2$, which is often smaller than the integer limit of Float32, the most common datatype used in CUDA programming. We take a moment to remark that by IEEE standards, integer multiplication (that only uses the mantissa) is guaranteed to be exact even in Float32 types. As such, it is often preferrable to use Float32 or Float64 datatypes since these are often more optimized than their Int counterparts due to [CITE]. EXPLAIN THE IMPLIED BIT.

Thus, we wish to find the maximum number of operations $o$ before our datatype overflows, namely one less than the value $o'$ such that $o' \cdot (N^2 - 2N - 1)$ is larger than the integer limit $I$. Thus
\begin{align*}
    o = \left\lfloor \frac{I}{(N^2 - 2N - 1)} \right\rfloor - 1
\end{align*}

We list common values of $o$ for the standard Julia datatypes and a selection of values for $N$:

\begin{center}
\begin{tabular}{llrrrrr}
\hline
\textbf{Datatype} & \multicolumn{1}{l}{\textbf{Largest Int}} & \textbf{3} & \textbf{11} & \textbf{101} & \textbf{8191} & \textbf{9765625} \\ \hline
Float16           & 1.02E+03                                     & 510        & 9           & -1           & -1            & -1               \\
Float32           & 8.39E+06                                     & 4.19E+06   & 8.56E+04    & 838          & -1            & -1               \\
Float64           & 4.50E+15                                     & 2.25E+15   & 4.60E+13    & 4.50E+11     & 6.71E+07      & 46               \\
Int16             & $2^{15}-1$                                     & 16382      & 333         & 2            & -1            & -1               \\
Int32             & $2^{31}-1$                                     & 1.07E+09   & 2.19E+07    & 2.15E+05     & 31            & -1               \\
Int64             & $2^{63}-1$                                     & 4.61E+18   & 9.41E+16    & 9.23E+14     & 1.38E+11      & 96713            \\
UInt16            & $2^16-1$                                     & 32766      & 667         & 5            & -1            & -1               \\
UInt32            & $2^{32}-1$                                     & 2.15E+09   & 4.38E+07    & 4.30E+05     & 63            & -1               \\
UInt64            & $2^{64}-1$                                     & 9.22E+18   & 1.88E+17    & 1.85E+15     & 2.75E+11      & 1.93E+05         \\ \hline
\end{tabular} \bigskip\\
\textbf{Table 1:} Maximum Number of Additions Before Overflow
\end{center}
%

Knowing $o$ helps us choose the smallest datatype such that matrix multiplication without overflow. In larger cases, we may need to reduce the current result before adding more elements. This common technique allows us to use considerably smaller datatypes, though it comes with cost of needing to reduce the result every so often.

\subsection{GPU and CUDA Considerations}

As described in [CITE], there are many additional considerations when doing matrix multiplication on the gpu. First, one should minimize memory access by storing commonly used items in cache or shared memory. Second, computations should be parallelized across groups of 32 threads, the smallest warp or unit call in CUDA. Third, conditionals and other control flow statements should be avoided in GPU kernels, as these are costly on the GPU. However, from our testing, we find it is worth checking whether our output needs to be reduced rather than reducing at every addition. Then, based on the input size, we end up with four regimes:
\begin{enumerate}
    \item Matmul on the CPU, if the time to move to the GPU and back is larger than computing on the CPU.
    \item CUBLAS's Matmul, if the matrix and modulus is small enough that we do not need to reduce partway.
    \item Reducing after every block, which allows us to get rid of the costly if statement.
    \item Reducing based on a conditional every addition, for extreme cases where $o < 32$.
\end{enumerate}
We implement this heuristic algorithm in the package.

% \subsection{Benchmarks and Comparisons}

% TODO: Compare to other implementations e.g. MAGMA

% We offer two ways to compare our implementation with other options. One is through raw speed in the amount of time it takes to complete the operation, and the other is by counring the number of floating point operations (flops) or Finite Field operations (Flops) completed per second. We remark that the number of flops is always larger than Flops, and that this number depends on the regime chosen.

\section{Heights of K3 surfaces}
\label{sec:heights:surfaces}

\subsection{Recollections on the moduli of K3 surfaces}

% We have the following:

\begin{thm}
    [Lang-Weil, \cite{lang-weil-1954-estimate} Theorem 1]
	Let \(X \ins \mathbb{P}^{n}\)
	be a projective variety of dimension \(r\). 
	Then 
	\[
		\#X(\mathbb{F}_{p}) = p^r + O(p^{r - \frac{1}{2}})
	\] 
\end{thm}

Say we have an ambient projective variety
\(Y\) with chosen
hypersurface \(D\).
The Lang-Weil estimate roughly
says that if we pick a random point \(x\),
we can expect \(x\) to lie in \(D\) 
with probability about \(1 / p\).
By \cite[Section~7]{artin-1974-k3-surfaces}, 
the locus \(M_{i}\) of height \(i\) such that \(h \leq i\)
is cut out by a single section in 
\(M_{i-1}\).
If we apply both of these facts,
with \(Y\) being the 
moduli space\footnote{
	say, the coarse moduli space of polarized K3 surfaces,
	though one should be able to make a similar statement
	for the moduli stack
} of quartic K3 surfaces,
we can conclude 
the probability of a random point 
in the moduli space being in the locus
\(M_{2}\) is about \(1 / p\). 
Inductively, we see:

\begin{heur}
The probability of a randomly
chosen surface having height \(h\) is about \(1 / p^{h}\).
\end{heur}

Thus, we may find a surface of height \(h\) by 
randomly choosing quartic K3 surfaces, and 
if we compute a few times \(p^{h}\) 
samples, we can be confident that we'll find
one with high probability.

In practice, our methodology is to choose
random quartic polynomials,
by sampling a point in the vector space
of homogeneous polynomials, which is isomorphic to
\(\mathbb{F}_{p}^{35}\).
To obtain the moduli space \(M\) of quartic K3 surfaces, 
we must projectivize and take the quotient by the action of 
\(\text{PGL}_{4}(\mathbb{F}_{p})\) which acts by changes of variables. 
Thus, there may be two sources of deviation from the expected
probability--the group quotient and the actual 
error term in the estimate.

\subsection{Computations}

Over \(\mathbb{F}_{5}\), we 
found that the probability of finding a K3 surface
of height \(h\) was about \(1 / 5^{h}\), to three digits
of precision, for all heights \(h \leq 6\).
For higher heights we had fewer samples
and more variance, although
the probabilities seem to be less than expected
for higher heights.
Similarly, over \(\mathbb{F}_{7}\) we found that
the probabilities very closely matched \(1 / 7^{h}\) 
for low heights, with more variance at higher heights.

For \(p=5\), the algorithm throughput is about
1400 surfaces per second on 
Nvidia 2080Ti GPUs provided by the 
UCSD research cluster, with 8 CPU threads running
examples on the same GPU.
Since much less time is taken for height \(1\) examples
(since the classical Fedder's criterion suffices), 
for the purposes of estimating the time to compute a
high height example we may ignore them.
Thus,
the expected compute time necessary to find a height = \(10\) 
example is about  \(5^{9} / 1400 \approx 1395\) seconds, or about 
compute minutes.
For \(p=7\), the throughput is about 185 surfaces
per second, also with 8 CPU cores calling the same GPU.
Thus, the expected compute time to compute
a height = \(10\) example is 
\(7^{9} / 185 \approx \) 218,127 seconds, or about 
60 compute hours.
The actual times to find the examples were much 
longer than this, because the authors were using a 
less-optimized NTT and had not yet discovered $\wicsalg$
when the computation started.
This is about three orders of magnitude better than the state of the art,
for calcualting newton polygons, which was \cite{fgmqt-2025-witt-vectors-macaulay2}.

For $p = 11$ and $p = 13$, 
the examples below could be computed in less than 12 compute hours
on the aforementioned 2080Tis (via random guessing).
Newton polygons of K3 surfaces may be computed by using the
library ToricControlledReduction
\cite{chk-2019-toric-controlled-reduction} 
(which computes the zeta
function of the surface, from which the Newton polygon and thus the
height can be deduced).
Our algorithm beats ToricControlledReduction by about 1.5 orders of magnitude:
on an Nvidia 3090 our method took 
around [TIME] seconds per surface (one CPU thread) in characteristic \(p = 11\), 
while ToricControlledReduction took 
about 43 seconds per surface.
For \(p = 13\), our 
code took [TIME] seconds per surface, while ToricControlledReduction
again took about 43 seconds per surface.
Since ToricControlledReduction has complexity \(O(p^{1 / 2})\) as \(p\) 
grows, we expect that if we implemented our algorithms for larger primes,
ToricControlledReduction would eventually beat it.


% p=11,
% us: 39s,TCR: error
% us: 39s,TCR: error
% us: 39s,TCR: error
% us: 39s,TCR: error

% us: 39s, TCR: 43s

% p=13, 
% us: 113 s, TCR: 43 s
% us: 102 s, TCR: 42 s
% us: 104 s, TCR: error
% us: 105 s, TCR: error


\begin{figure}[htbp]
	\begin{center}
        \def\arraystretch{1.5}
		\begin{tabular}{p{0.1\linewidth}p{0.8\linewidth}}
			 \toprule
             \textbf{Height} & \textbf{Equation} \\
			 \midrule
			 1 & \(x_{1}^{4} + x_{2}^{4} + x_{3}^{4} + x_{4}^{4}\) \\
			  
			 2 & \(4 x_1^4 + 2 x_1^3 x_2 + x_1^3 x_4 + 4 x_1^2 x_2^2 + 2 x_1^2 x_2 x_3 + 2 x_1^2 x_3^2 + x_1^2 x_3 x_4 + 3 x_1 x_2^3 + 4 x_1 x_2^2 x_3 + 4 x_1 x_2^2 x_4 + 2 x_1 x_2 x_3 x_4 + 3 x_1 x_2 x_4^2 + 3 x_1 x_3^3 + x_1 x_3^2 x_4 + x_1 x_3 x_4^2 + x_1 x_4^3 + 4 x_2^4 + 2 x_2^3 x_3 + 4 x_2^3 x_4 + 4 x_2^2 x_3^2 + x_2^2 x_3 x_4 + 2 x_2^2 x_4^2 + 3 x_2 x_3^3 + 4 x_2 x_3^2 x_4 + 4 x_2 x_3 x_4^2 + 2 x_2 x_4^3 + 2 x_3^4 + 2 x_3^3 x_4 + 2 x_3^2 x_4^2 + x_3 x_4^3 + 4 x_4^4\) \\
			  
			 3 & \(2 x_1^4 + x_1^3 x_2 + 3 x_1^3 x_3 + x_1^3 x_4 + x_1^2 x_2 x_3 + 4 x_1^2 x_2 x_4 + x_1^2 x_3^2 + 4 x_1^2 x_3 x_4 + 3 x_1^2 x_4^2 + 4 x_1 x_2^3 + 3 x_1 x_2^2 x_3 + x_1 x_2^2 x_4 + 2 x_1 x_2 x_3^2 + 3 x_1 x_2 x_3 x_4 + x_1 x_3^3 + 4 x_1 x_3 x_4^2 + 2 x_1 x_4^3 + x_2^3 x_3 + 3 x_2^3 x_4 + 4 x_2^2 x_3^2 + 4 x_2^2 x_3 x_4 + x_2^2 x_4^2 + 2 x_2 x_3^3 + 3 x_2 x_3^2 x_4 + 4 x_2 x_3 x_4^2 + 3 x_2 x_4^3 + 4 x_3^4 + 3 x_3^3 x_4 + 2 x_3 x_4^3 + 3 x_4^4\) \\
			  
			 4 & \(4 x_1^4 + 2 x_1^3 x_3 + 4 x_1^3 x_4 + 3 x_1^2 x_2^2 + 3 x_1^2 x_2 x_3 + 4 x_1^2 x_2 x_4 + 2 x_1^2 x_3 x_4 + x_1^2 x_4^2 + 3 x_1 x_2^3 + x_1 x_2^2 x_3 + x_1 x_2^2 x_4 + x_1 x_2 x_3^2 + x_1 x_2 x_3 x_4 + x_1 x_2 x_4^2 + 2 x_1 x_3^3 + 2 x_1 x_3^2 x_4 + x_1 x_3 x_4^2 + 2 x_1 x_4^3 + 4 x_2^4 + 3 x_2^3 x_3 + x_2^3 x_4 + 3 x_2^2 x_3^2 + 3 x_2^2 x_3 x_4 + x_2^2 x_4^2 + 2 x_2 x_3^3 + 3 x_2 x_3^2 x_4 + x_2 x_3 x_4^2 + 3 x_2 x_4^3 + 3 x_3^4 + 2 x_3^3 x_4 + 4 x_3^2 x_4^2 + x_3 x_4^3\) \\
			  
			 5 & \(2x_1^4 + 2x_1^3x_2 + 4x_1^3x_3 + 3x_1^3x_4 + 2x_1^2x_2^2 + 4x_1^2x_2x_3 + x_1^2x_2x_4 + 2x_1^2x_3^2 + 3x_1^2x_3x_4 + 4x_1x_2^2x_3 + 3x_1x_2^2x_4 + x_1x_2x_3^2 + x_1x_2x_3x_4 + 2x_1x_2x_4^2 + 3x_1x_3^3 + 3x_1x_3^2x_4 + x_1x_3x_4^2 + x_2^4 + x_2^3x_3 + 2x_2^2x_3^2 + 2x_2^2x_3x_4 + 3x_2^2x_4^2 + 2x_2x_3^3 + 2x_2x_3^2x_4 + 2x_2x_3x_4^2 + 4x_3^4 + x_3^3x_4 + 2x_3^2x_4^2 + 3x_4^4\) \\
			  
			 6 & \(x_1^3x_2 + x_1^3x_3 + 3x_1^3x_4 + 3x_1^2x_2^2 + 2x_1^2x_2x_4 + 4x_1^2x_3x_4 + 4x_1^2x_4^2 + x_1x_2^3 + 3x_1x_2^2x_3 + 4x_1x_2^2x_4 + 2x_1x_2x_3^2 + 2x_1x_2x_3x_4 + 2x_1x_2x_4^2 + 2x_1x_3^3 + 3x_1x_3^2x_4 + 3x_1x_3x_4^2 + x_1x_4^3 + 4x_2^4 + x_2^3x_3 + x_2^3x_4 + x_2^2x_3^2 + 4x_2^2x_3x_4 + x_2^2x_4^2 + 3x_2x_3^3 + 2x_2x_3^2x_4 + 3x_2x_4^3 + 4x_3^4 + x_3^3x_4 + 3x_3x_4^3 + x_4^4\) \\
			  
			 7 & \(4x_1^4 + x_1^3x_3 + 3x_1^3x_4 + 4x_1^2x_2^2 + 2x_1^2x_2x_3 + 2x_1^2x_2x_4 + 2x_1^2x_3^2 + 4x_1^2x_3x_4 + 4x_1x_2^2x_3 + 2x_1x_2x_3^2 + x_1x_2x_4^2 + 2x_1x_3^3 + 4x_1x_3^2x_4 + 2x_1x_3x_4^2 + x_1x_4^3 + 4x_2^4 + 3x_2^3x_4 + 3x_2^2x_3^2 + x_2^2x_3x_4 + 2x_2^2x_4^2 + 3x_2x_3^2x_4 + 4x_2x_3x_4^2 + 3x_2x_4^3 + 3x_3^3x_4 + x_3^2x_4^2 + 4x_4^4\) \\
			  
			 8 & \(x_1^4 + 2x_1^3x_2 + 4x_1^3x_3 + x_1^2x_2^2 + 4x_1^2x_2x_3 + x_1^2x_2x_4 + x_1^2x_3x_4 + 2x_1x_2^2x_3 + 2x_1x_2^2x_4 + 2x_1x_2x_3^2 + 4x_1x_2x_3x_4 + 3x_1x_2x_4^2 + 3x_1x_3^3 + 4x_1x_3^2x_4 + 3x_1x_3x_4^2 + x_1x_4^3 + 4x_2^4 + 4x_2^3x_3 + x_2^3x_4 + 4x_2^2x_3^2 + 2x_2^2x_3x_4 + x_2^2x_4^2 + 4x_2x_3^2x_4 + x_2x_4^3 + x_3^4 + 2x_3^3x_4 + x_3^2x_4^2 + 4x_4^4\) \\
			  
			 9 & \(3 x_1^4 + 3 x_1^3 x_2 + 3 x_1^3 x_3 + x_1^2 x_2^2 + 3 x_1^2 x_2 x_3 + 3 x_1^2 x_2 x_4 + 3 x_1^2 x_3^2 + 2 x_1^2 x_3 x_4 + 2 x_1^2 x_4^2 + 4 x_1 x_2^3 + 2 x_1 x_2^2 x_3 + 4 x_1 x_2 x_3^2 + 2 x_1 x_2 x_3 x_4 + 4 x_1 x_2 x_4^2 + x_1 x_3^3 + 3 x_1 x_3^2 x_4 + 3 x_1 x_3 x_4^2 + x_1 x_4^3 + 3 x_2^3 x_3 + 4 x_2^3 x_4 + 3 x_2^2 x_3 x_4 + x_2^2 x_4^2 + 4 x_2 x_3^2 x_4 + 4 x_2 x_3 x_4^2 + 4 x_2 x_4^3 + 3 x_3 x_4^3 + 4 x_4^4\) \\
			  
			 10 & \(2 x_1^4 + 4 x_1^3 x_2 + 3 x_1^3 x_3 + x_1^3 x_4 + x_1^2 x_2^2 + 2 x_1^2 x_2 x_3 + 2 x_1^2 x_2 x_4 + 4 x_1^2 x_3^2 + 4 x_1^2 x_3 x_4 + 2 x_1^2 x_4^2 + x_1 x_2^3 + 4 x_1 x_2^2 x_4 + 3 x_1 x_2 x_3^2 + 3 x_1 x_2 x_4^2 + 2 x_1 x_3^3 + 3 x_1 x_3^2 x_4 + 2 x_1 x_3 x_4^2 + x_1 x_4^3 + 3 x_2^4 + 2 x_2^3 x_3 + 2 x_2^3 x_4 + 4 x_2^2 x_3^2 + 3 x_2^2 x_3 x_4 + 3 x_2^2 x_4^2 + x_2 x_3^3 + 2 x_2 x_3 x_4^2 + 2 x_2 x_4^3 + 4 x_3^4 + x_3^3 x_4 + 3 x_3^2 x_4^2 + 4 x_3 x_4^3 + 3 x_4^4\)\\
			 
			 \(\infty\)& \(x_{1}^{4} + x_{2}^{4} + x_{3}^{4} + x_{4}^{4} + x y z w\) \\
             \bottomrule
			  
		\end{tabular}
	\end{center}
	\caption{Quartic K3 surfaces with specified Artin--Mazur height over \(\mathbb{F}_{5}\)}
\end{figure}

\begin{figure}[htbp]
	\begin{center}
		\def\arraystretch{1.5}
		\begin{tabular}{p{0.1\linewidth}p{0.8\linewidth}}
			 \toprule
             \textbf{Height} & \textbf{Equation} \\
			 \midrule
			 
			 1 & \(5x_1^4 + 5x_1^3x_3 + 2x_1^3x_4 + 3x_1^2x_2x_3 + x_1^2x_2x_4 + 6x_1^2x_3^2 + 3x_1^2x_3x_4 +
                 3x_1^2x_4^2 + 4x_1x_2^3 + 6x_1x_2^2x_3 + 2x_1x_2^2x_4 + 4x_1x_2x_3^2 + 5x_1x_2x_3x_4 + 4x_1x_2x_4^2
                 + 5x_1x_3^3 + 4x_1x_3^2x_4 + 5x_1x_4^3 + 5x_2^4 + x_2^3x_3 + 4x_2^3x_4 + 5x_2^2x_3^2 + x_2^2x_3x_4 +
                 x_2x_3^3 + 2x_2x_3^2x_4 + 2x_2x_3x_4^2 + x_2x_4^3 + 3x_3^4 + 5x_3^3x_4 + 3x_3^2x_4^2 + x_4^4 \) \\
			 
             2 & \(3x_1^4 + 4x_1^3x_2 + x_1^3x_3 + x_1^3x_4 + x_1^2x_2^2 + 5x_1^2x_2x_3 + 5x_1^2x_2x_4 + 
             3x_1^2x_3^2 + 5x_1^2x_3x_4 + 6x_1^2x_4^2 + 2x_1x_2^3 + x_1x_2^2x_3 + 5x_1x_2^2x_4 + 2x_1x_2x_3x_4 + 
             x_1x_2x_4^2 + 2x_1x_3^3 + 3x_1x_3^2x_4 + x_1x_3x_4^2 + x_1x_4^3 + 4x_2^4 + 4x_2^3x_3 + 4x_2^3x_4 + 
             6x_2^2x_3^2 + 3x_2^2x_3x_4 + 3x_2x_3^3 + 4x_2x_3x_4^2 + 2x_3^4 + 4x_3^3x_4 + 4x_3^2x_4^2 + 2x_3x_4^3 + 6x_4^4\) \\
			 
			 3 &  \(4x_1^4 + x_1^3x_2 + 2x_1^3x_3 + 6x_1^3x_4 + 6x_1^2x_2^2 + 3x_1^2x_2x_3 +
             3x_1^2x_2x_4 + 2x_1^2x_3x_4 + 4x_1^2x_4^2 + 2x_1x_2^3 + 5x_1x_2^2x_4 + 5x_1x_2x_3^2 +
             4x_1x_2x_3x_4 + 4x_1x_2x_4^2 + 6x_1x_3^3 + x_1x_3^2x_4 + 5x_1x_3x_4^2 + 2x_1x_4^3 +
             3x_2^4 + 2x_2^3x_3 + 5x_2^2x_3^2 + 5x_2^2x_3x_4 + 3x_2^2x_4^2 + 4x_2x_3^3 +
             6x_2x_3^2x_4 + 5x_2x_3x_4^2 + 3x_2x_4^3 + 4x_3^3x_4 + 4x_3^2x_4^2 + x_3x_4^3 + 5x_4^4\)
             \\
			 
			 4 & \(2x_1^4 + 6x_1^3x_2 + 3x_1^3x_3 + x_1^3x_4 + 4x_1^2x_2^2 + 3x_1^2x_2x_3 +
             3x_1^2x_2x_4 + 2x_1^2x_3^2 + x_1^2x_3x_4 + 2x_1^2x_4^2 + 3x_1x_2^3 + 6x_1x_2^2x_4 +
             x_1x_2x_3^2 + 6x_1x_2x_3x_4 + x_1x_2x_4^2 + 4x_1x_3^3 + 2x_1x_3^2x_4 + 5x_1x_3x_4^2 +
             2x_1x_4^3 + 6x_2^4 + 3x_2^3x_3 + 5x_2^2x_3^2 + x_2^2x_3x_4 + 5x_2^2x_4^2 + 4x_2x_3^3 +
             3x_2x_3^2x_4 + x_2x_4^3 + 6x_3^4 + 2x_3^3x_4 + x_3^2x_4^2 + 3x_3x_4^3 + 2x_4^4\) \\
			 
			 5 & \(5x_1^4 + 6x_1^3x_2 + 2x_1^3x_3 + 3x_1^3x_4 + 4x_1^2x_2^2 + 3x_1^2x_2x_4 +
             2x_1^2x_3^2 + 3x_1^2x_3x_4 + 6x_1^2x_4^2 + 4x_1x_2^2x_3 + 6x_1x_2^2x_4 + 2x_1x_2x_3^2 +
             3x_1x_2x_3x_4 + 5x_1x_2x_4^2 + 3x_1x_3^3 + x_1x_3^2x_4 + 5x_1x_3x_4^2 + 6x_2^4 +
             5x_2^3x_3 + 3x_2^2x_3^2 + 6x_2^2x_3x_4 + 3x_2x_3^3 + 3x_2x_3^2x_4 + 4x_2x_3x_4^2 +
             3x_2x_4^3 + 5x_3^4 + 6x_3^2x_4^2 + 6x_3x_4^3 + 3x_4^4\) \\
			 
			 6 & \(x_1^4 + x_1^3x_2 + 4x_1^3x_3 + 6x_1^3x_4 + 6x_1^2x_2^2 + 2x_1^2x_2x_4 +
             6x_1^2x_3x_4 + 6x_1^2x_4^2 + 4x_1x_2^3 + 3x_1x_2^2x_3 + 2x_1x_2^2x_4 + 2x_1x_2x_3^2 +
             5x_1x_2x_3x_4 + 6x_1x_2x_4^2 + 6x_1x_3^2x_4 + 3x_1x_3x_4^2 + 6x_2^4 + 2x_2^3x_3 +
             3x_2^3x_4 + 5x_2^2x_3^2 + 4x_2^2x_3x_4 + 6x_2^2x_4^2 + 5x_2x_3^2x_4 + x_2x_3x_4^2 +
             3x_2x_4^3 + 2x_3^4 + 2x_3^3x_4 + 5x_3^2x_4^2 + 2x_3x_4^3 + 4x_4^4 \) \\
			 
			 7 & \(2*x_1^4 + 3*x_1^3*x_2 + 4*x_1^3*x_3 + 4*x_1^3*x_4 + 6*x_1^2*x_2^2 + 2*x_1^2*x_2*x_3 + x_1^2*x_2*x_4 + 6*x_1^2*x_3^2 + 4*x_1^2*x_3*x_4 + x_1^2*x_4^2 + 5*x_1*x_2^3 + x_1*x_2^2*x_3 + 6*x_1*x_2^2*x_4 + 3*x_1*x_2*x_3^2 + 6*x_1*x_2*x_4^2 + 4*x_1*x_3^3 + 5*x_1*x_3^2*x_4 + 2*x_1*x_3*x_4^2 + x_1*x_4^3 + 5*x_2^4 + 4*x_2^3*x_3 + x_2^3*x_4 + 5*x_2^2*x_3^2 + x_2^2*x_3*x_4 + 4*x_2*x_3^3 + x_2*x_3*x_4^2 + 2*x_2*x_4^3 + 5*x_3^4 + x_3^3*x_4 + 2*x_3^2*x_4^2 + 4*x_3*x_4^3 + 4*x_4^4\) \\
			 
			 8 & \(2x_1^3x_2 + 2x_1^3x_4 + 4x_1^2x_2^2 + 6x_1^2x_2x_3 + 5x_1^2x_2x_4 + 4x_1^2x_3^2 +
             3x_1^2x_3x_4 + 3x_1^2x_4^2 + 4x_1x_2^3 + x_1x_2^2x_3 + x_1x_2^2x_4 + 4x_1x_2x_3^2 +
             5x_1x_2x_3x_4 + x_1x_2x_4^2 + 3x_1x_3^3 + x_1x_3^2x_4 + 3x_1x_3x_4^2 + x_1x_4^3 +
             5x_2^3x_3 + 5x_2^3x_4 + 6x_2^2x_3x_4 + 6x_2^2x_4^2 + 4x_2x_3^2x_4 + 3x_2x_3x_4^2 +
             2x_2x_4^3 + 6x_3^3x_4 + 6x_3^2x_4^2 + 4x_3x_4^3\) \\
			 
			 9 & \(2x_1^3x_2 + x_1^3x_3 + 6x_1^3x_4 + 6x_1^2x_2^2 + 4x_1^2x_2x_3 + 2x_1^2x_2x_4 +
             3x_1^2x_3x_4 + x_1^2x_4^2 + x_1x_2^3 + x_1x_2^2x_3 + 6x_1x_2^2x_4 + 6x_1x_2x_3^2 +
             6x_1x_2x_3x_4 + 6x_1x_2x_4^2 + 2x_1x_3^3 + 4x_1x_3x_4^2 + 6x_1x_4^3 + 6x_2^3x_3 +
             4x_2^3x_4 + 3x_2^2x_3^2 + 4x_2x_3^3 + 5x_2x_3^2x_4 + 4x_2x_3x_4^2 + 5x_2x_4^3 +
             3x_3^3x_4 + 4x_3^2x_4^2 + 2x_3x_4^3 + 3x_4^4\) \\
			 
			 10 & \( 3x_1^4 + 2x_1^3x_2 + x_1^3x_3 + x_1^3x_4 + 4x_1^2x_2x_3 + 2x_1^2x_2x_4 +
             5x_1^2x_3x_4 + 6x_1^2x_4^2 + x_1x_2^3 + 2x_1x_2^2x_4 + 5x_1x_2x_3^2 + 3x_1x_2x_3x_4 +
             4x_1x_2x_4^2 + 5x_1x_3^3 + x_1x_3^2x_4 + x_1x_3x_4^2 + x_1x_4^3 + 6x_2^4 + x_2^3x_4 +
             6x_2^2x_3^2 + x_2^2x_3x_4 + 4x_2^2x_4^2 + x_2x_3^3 + 5x_2x_4^3 + 2x_3^4 + 5x_3^3x_4 +
             5x_3^2x_4^2 + x_3x_4^3 + 6x_4^4\) \\
			 
             \(\infty\) & \( 3x_1^4 + 3x_1^3x_2 + 3x_1^3x_3 + 6x_1^2x_2^2 + 3x_1^2x_2x_4 + 2x_1^2x_3^2 + 2x_1^2x_3x_4 + 3x_1^2x_4^2 + 6x_1x_2^3 + 5x_1x_2^2x_3 + x_1x_2x_3x_4 + 5x_1x_2x_4^2 + 5x_1x_3^3 + 4x_1x_3^2x_4 + 3x_1x_3x_4^2 + 6x_1x_4^3 + x_2^4 + 4x_2^3x_4 + 3x_2^2x_3^2 + 5x_2^2x_3x_4 + 5x_2x_3^3 + x_2x_3^2x_4 + 6x_2x_3x_4^2 + x_3^3x_4 + x_3^2x_4^2 + 3x_3x_4^3 + 4x_4^4\) \\
             \bottomrule
		\end{tabular}
	\end{center}
	\caption{Quartic K3 surfaces with specified Artin--Mazur height over \(\mathbb{F}_{7}\)}
\end{figure}

\begin{figure}[htbp]
	\begin{center}
		\def\arraystretch{1.5}
		\begin{tabular}{p{0.1\linewidth}p{0.8\linewidth}}
			 \toprule
             \textbf{Height} & \textbf{Equation} \\
			 \midrule
			 1 & $4x_1^4 + 6x_1^3x_2 + x_1^3x_3 + 2x_1^3x_4 + 3x_1^2x_2^2 + x_1^2x_2x_3 + 3x_1^2x_2x_4 + 6x_1^2x_3^2 + 6x_1^2x_3x_4 + 8x_1^2x_4^2 + 7x_1x_2^3 + 2x_1x_2^2x_3 + 8x_1x_2^2x_4 + 8x_1x_2x_3x_4 + 10x_1x_2x_4^2 + 10x_1x_3^3 + 9x_1x_3^2x_4 + 6x_1x_3x_4^2 + 3x_1x_4^3 + 6x_2^4 + 7x_2^3x_3 + 4x_2^3x_4 + 10x_2^2x_3^2 + 3x_2^2x_3x_4 + 5x_2^2x_4^2 + 4x_2x_3^2x_4 + 6x_2x_4^3 + 3x_3^4 + 4x_3^3x_4 + 7x_3^2x_4^2 + 9x_3x_4^3 + 5x_4^4 $\\
			  
			 2 & $4x_1^4 + 5x_1^3x_2 + 9x_1^3x_3 + 2x_1^3x_4 + 8x_1^2x_2^2 + x_1^2x_2x_3 + 9x_1^2x_2x_4 + x_1^2x_3^2 + 8x_1^2x_3x_4 + 6x_1x_2^3 + 10x_1x_2^2x_3 + 2x_1x_2^2x_4 + 10x_1x_2x_3^2 + 9x_1x_2x_3x_4 + 6x_1x_2x_4^2 + 8x_1x_3^3 + 4x_1x_3^2x_4 + 7x_1x_3x_4^2 + 9x_1x_4^3 + 3x_2^4 + 7x_2^3x_3 + 6x_2^3x_4 + 10x_2^2x_3^2 + 8x_2^2x_3x_4 + x_2^2x_4^2 + 9x_2x_3^3 + 6x_2x_3^2x_4 + x_2x_3x_4^2 + 9x_3^4 + 10x_3^3x_4 + x_3^2x_4^2 + x_3x_4^3 + 4x_4^4$\\
			  
			 3 & $10x_1^4 + 9x_1^3x_2 + 5x_1^3x_3 + 4x_1^3x_4 + 3x_1^2x_2^2 + 9x_1^2x_2x_3 + 4x_1^2x_2x_4 + 10x_1^2x_3^2 + 4x_1^2x_3x_4 + 8x_1^2x_4^2 + 8x_1x_2^3 + 9x_1x_2^2x_3 + 3x_1x_2^2x_4 + 7x_1x_2x_3^2 + 3x_1x_2x_4^2 + 8x_1x_3^3 + 2x_1x_3^2x_4 + x_1x_3x_4^2 + 7x_1x_4^3 + 2x_2^4 + 3x_2^3x_4 + x_2^2x_3^2 + x_2^2x_3x_4 + x_2^2x_4^2 + 5x_2x_3^3 + 9x_2x_3^2x_4 + 9x_2x_3x_4^2 + 4x_2x_4^3 + 5x_3^4 + 10x_3^3x_4 + 10x_3x_4^3 + 10x_4^4$\\
             
			 4 & $2x_1^4 + 4x_1^3x_2 + 9x_1^3x_3 + 10x_1^3x_4 + 2x_1^2x_2^2 + 4x_1^2x_2x_3 + 4x_1^2x_2x_4 + 4x_1^2x_3^2 + 10x_1^2x_3x_4 + 9x_1^2x_4^2 + 5x_1x_2^3 + 5x_1x_2^2x_3 + x_1x_2^2x_4 + 8x_1x_2x_3^2 + 2x_1x_2x_3x_4 + 10x_1x_2x_4^2 + 8x_1x_3^3 + 7x_1x_3^2x_4 + 5x_1x_3x_4^2 + 4x_1x_4^3 + 3x_2^4 + 6x_2^3x_3 + 4x_2^3x_4 + 10x_2^2x_3^2 + 5x_2^2x_3x_4 + 5x_2^2x_4^2 + x_2x_3^3 + 5x_2x_4^3 + 5x_3^4 + 7x_3^2x_4^2 + 5x_3x_4^3 + 9x_4^4$\\
			 
			 5 & \(10x_1^4 + x_1^3x_2 + 6x_1^3x_3 + 3x_1^3x_4 + x_1^2x_2^2 + 9x_1^2x_2x_3 + 6x_1^2x_2x_4 + 6x_1^2x_3^2 + 8x_1^2x_3x_4 + 4x_1^2x_4^2 + 3x_1x_2^3 + 7x_1x_2^2x_3 + 3x_1x_2^2x_4 + 7x_1x_2x_3^2 + 9x_1x_2x_3x_4 + 8x_1x_2x_4^2 + 7x_1x_3^3 + x_1x_3x_4^2 + 7x_1x_4^3 + x_2^4 + 3x_2^3x_3 + 7x_2^3x_4 + 5x_2^2x_3^2 + 7x_2^2x_3x_4 + 8x_2^2x_4^2 + 8x_2x_3^3 + 5x_2x_3^2x_4 + x_2x_3x_4^2 + 9x_2x_4^3 + 7x_3^4 + 4x_3^3x_4 + 4x_3^2x_4^2 + 3x_3x_4^3\)\\
             \bottomrule
		\end{tabular}
	\end{center}
	\caption{Quartic K3 surfaces with specified Artin--Mazur height over \(\mathbb{F}_{11}\)}
\end{figure}

\begin{figure}[htbp]
	\begin{center}
		\def\arraystretch{1.5}
		\begin{tabular}{p{0.1\linewidth}p{0.8\linewidth}}
			 \toprule
             \textbf{Height} & \textbf{Equation} \\
			 \midrule
			 1 & \(6x_1^4 + 7x_1^3x_3 + 4x_1^3x_4 + 6x_1^2x_2^2 + 7x_1^2x_2x_3 + 9x_1^2x_2x_4 + 2x_1^2x_3^2 + 3x_1^2x_3x_4 + 12x_1^2x_4^2 + 8x_1x_2^3 + 4x_1x_2^2x_3 + x_1x_2^2x_4 + 9x_1x_2x_3^2 + 8x_1x_2x_3x_4 + 10x_1x_2x_4^2 + 8x_1x_3^3 + 2x_1x_3^2x_4 + 9x_1x_3x_4^2 + 4x_1x_4^3 + 5x_2^4 + 4x_2^3x_3 + 2x_2^2x_3^2 + x_2^2x_3x_4 + 2x_2^2x_4^2 + 10x_2x_3^3 + 2x_2x_3^2x_4 + 2x_2x_3x_4^2 + 5x_2x_4^3 + 4x_3^4 + 3x_3^2x_4^2 + 2x_4^4\)\\
			  
			 2 & \(12x_1^4 + 8x_1^3x_2 + 8x_1^3x_3 + 10x_1^3x_4 + 8x_1^2x_2^2 + 11x_1^2x_2x_4 + 8x_1^2x_3^2 + 12x_1^2x_3x_4 + x_1^2x_4^2 + 7x_1x_2^3 + 9x_1x_2^2x_3 + 11x_1x_2^2x_4 + 10x_1x_2x_3^2 + 7x_1x_2x_4^2 + 8x_1x_3^3 + 3x_1x_3^2x_4 + 11x_1x_3x_4^2 + x_1x_4^3 + 4x_2^4 + 7x_2^3x_3 + 4x_2^3x_4 + 8x_2^2x_3^2 + 12x_2^2x_3x_4 + 6x_2^2x_4^2 + 7x_2x_3^3 + 12x_2x_3^2x_4 + 4x_2x_3x_4^2 + 10x_2x_4^3 + 4x_3^4 + 8x_3^3x_4 + 5x_3^2x_4^2 + 4x_3x_4^3\)\\
			  
			 3 & \(8x_1^4 + 2x_1^3x_2 + 3x_1^3x_3 + x_1^3x_4 + 6x_1^2x_2^2 + 7x_1^2x_2x_3 + 5x_1^2x_2x_4 + 2x_1^2x_3^2 + x_1^2x_4^2 + 11x_1x_2^3 + 10x_1x_2^2x_3 + 3x_1x_2^2x_4 + 5x_1x_2x_3^2 + 10x_1x_2x_3x_4 + 7x_1x_2x_4^2 + 12x_1x_3^3 + 12x_1x_3^2x_4 + 5x_1x_3x_4^2 + 7x_1x_4^3 + 7x_2^4 + 6x_2^3x_3 + 3x_2^3x_4 + 10x_2^2x_3^2 + 5x_2^2x_3x_4 + 12x_2^2x_4^2 + x_2x_3^3 + 3x_2x_3^2x_4 + 12x_2x_3x_4^2 + 8x_2x_4^3 + 10x_3^4 + 7x_3^3x_4 + 4x_3^2x_4^2 + 8x_3x_4^3 + 2x_4^4\)\\
             
			 4 & \(4x_1^4 + 4x_1^3x_2 + 2x_1^3x_3 + 3x_1^3x_4 + 9x_1^2x_2^2 + 6x_1^2x_2x_3 + 7x_1^2x_2x_4 + 10x_1^2x_3^2 + x_1^2x_3x_4 + 4x_1x_2^3 + 4x_1x_2^2x_3 + 6x_1x_2^2x_4 + 12x_1x_2x_3^2 + 7x_1x_2x_3x_4 + 3x_1x_2x_4^2 + 11x_1x_3^3 + 9x_1x_3^2x_4 + 10x_1x_3x_4^2 + 11x_1x_4^3 + 3x_2^4 + 5x_2^3x_3 + 8x_2^3x_4 + 5x_2^2x_3x_4 + 5x_2^2x_4^2 + 5x_2x_3^3 + 10x_2x_3^2x_4 + 2x_2x_3x_4^2 + 10x_2x_4^3 + 4x_3^4 + 5x_3^2x_4^2 + 4x_3x_4^3 + 6x_4^4\)\\
			  
			 5 & \(11x_1^4 + 4x_1^3x_2 + 12x_1^3x_3 + 4x_1^3x_4 + 6x_1^2x_2^2 + 10x_1^2x_2x_3 + 4x_1^2x_2x_4 + x_1^2x_3^2 + 7x_1^2x_3x_4 + 4x_1^2x_4^2 + 6x_1x_2^3 + 11x_1x_2^2x_3 + 7x_1x_2^2x_4 + 8x_1x_2x_3^2 + 10x_1x_2x_4^2 + x_1x_3^3 + 9x_1x_3^2x_4 + 8x_1x_3x_4^2 + 11x_1x_4^3 + 4x_2^4 + 8x_2^3x_3 + 5x_2^2x_3x_4 + 7x_2^2x_4^2 + 8x_2x_3^3 + 6x_2x_3^2x_4 + 5x_2x_4^3 + 2x_3^4 + 10x_3^3x_4 + 8x_3^2x_4^2 + 10x_3x_4^3 + 6x_4^4\) \\
            \bottomrule
		\end{tabular}
	\end{center}
	\caption{Quartic K3 surfaces with specified Artin--Mazur height over \(\mathbb{F}_{13}\)}
\end{figure}


\clearpage

\bibliographystyle{plain}
\bibliography{main}

\end{document}
