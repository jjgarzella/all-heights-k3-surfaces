
The purpose of the following section is the following statement:

\begin{thm}
	Given a random quartic K3 surface over \(\mathbb{F}_{p}\),
	the probability of guessing a surface of height \(h\) 
	is \(1 / p^{h}\).
\end{thm}

\begin{proof}
	Let \(M\)

	By \cite[Section~7]{artin-1974-k3-surfaces}, 
	the locus \(M_{i}\) of height \(i\) such that \(h \leq i\)
	is cut out by a single section in 
	\(M_{i-1}\).

	INSERT ZARISKI TOLOPOGY ARGUMENT.
\end{proof}

In practice, when we guess quartic polynomials with random coefficients,
we are choosing a point in the vector space
of homogeneous polynomials, which is isomorphic to
\(\mathbb{F}_{p}^{56}\).
To obtain the moduli space \(M\) of quartic K3 surfaces, 
we must projectivize and take the quotient by the action of 
\(PGL_{4}\) which acts by changes of variables. 
In other words, \(M = \left[  \mathbb{P}^{55} / PGL_{4} \right] \).
Picking points in this moduli space should give us precisely
a  \(1 / p^{h}\) chance. 

However, the probability of picking a random
polynomial \(f\)
(equivalently, an element of \(\mathbb{F}_{p}^{56}\))
whose associated hypersurface
\(Z(f)\) has height \(h\) is still nonzero,
and seems to be roughly the same.
That is, projectivization
and the quotient by changes of variables 
don't seem to affect the probabilities
too much. 
 
For \(p = 7\) and above, when we restrict the coefficients
that may appear in the random polynomial (to speed up the FFT),
the probabilities of getting higher heights
seem to be less.
