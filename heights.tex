\subsection{Recollections on the moduli of K3 surfaces}

% We have the following:

\begin{thm}
    [Lang-Weil, \cite{lang-weil-1954-estimate} Theorem 1]
	Let \(X \ins \mathbb{P}^{n}\)
	be a projective variety of dimension \(r\). 
	Then 
	\[
		\#X(\mathbb{F}_{p}) = p^r + O(p^{r - \frac{1}{2}})
	\] 
\end{thm}

Say we have an ambient projective variety
\(Y\) with chosen
hypersurface \(D\).
The Lang-Weil estimate roughly
says that if we pick a random point \(x\),
we can expect \(x\) to lie in \(D\) 
with probability about \(1 / p\).
By \cite[Section~7]{artin-1974-k3-surfaces}, 
the locus \(M_{i}\) of height \(i\) such that \(h \leq i\)
is cut out by a single section in 
\(M_{i-1}\).
If we apply both of these facts,
with \(Y\) being the 
moduli space\footnote{
	say, the coarse moduli space of polarized K3 surfaces,
	though one should be able to make a similar statement
	for the moduli stack
} of quartic K3 surfaces,
we can conclude 
the probability of a random point 
in the moduli space being in the locus
\(M_{2}\) is about \(1 / p\). 
Inductively, we see:

\begin{heur}
The probability of a randomly
chosen surface having height \(h\) is about \(1 / p^{h}\).
\end{heur}

Thus, we may find a surface of height \(h\) by 
randomly choosing quartic K3 surfaces, and 
if we compute a few times \(p^{h}\) 
samples, we can be confident that we'll find
one with high probability.

In practice, our methodology is to choose
random quartic polynomials,
by sampling a point in the vector space
of homogeneous polynomials, which is isomorphic to
\(\mathbb{F}_{p}^{35}\).
To obtain the moduli space \(M\) of quartic K3 surfaces, 
we must projectivize and take the quotient by the action of 
\(\text{PGL}_{4}(\mathbb{F}_{p})\) which acts by changes of variables. 
Thus, there may be two sources of deviation from the expected
probability--the group quotient and the actual 
error term in the estimate.

\subsection{Computations}

Over \(\mathbb{F}_{5}\), we 
found that the probability of finding a K3 surface
of height \(h\) was about \(1 / 5^{h}\), to three digits
of precision, for all heights \(h \leq 6\).
For higher heights we had fewer samples
and more variance, although
the probabilities seem to be less than expected
for higher heights.
Similarly, over \(\mathbb{F}_{7}\) we found that
the probabilities very closely matched \(1 / 7^{h}\) 
for low heights, with more variance at higher heights.

For \(p=5\), the algorithm throughput is about
1400 surfaces per second on 
Nvidia 2080Ti GPUs provided by the 
UCSD research cluster, with 8 CPU threads running
examples on the same GPU.
Since much less time is taken for height \(1\) examples
(since the classical Fedder's criterion suffices), 
for the purposes of estimating the time to compute a
high height example we may ignore them.
Thus,
the expected compute time necessary to find a height = \(10\) 
example is about  \(5^{9} / 1400 \approx 1395\) seconds, or about 
compute minutes.
For \(p=7\), the throughput is about 185 surfaces
per second, also with 8 CPU cores calling the same GPU.
Thus, the expected compute time to compute
a height = \(10\) example is 
\(7^{9} / 185 \approx \) 218,127 seconds, or about 
60 compute hours.
The actual times to find the examples were much 
longer than this, because the authors were using a 
less-optimized NTT and had not yet discovered $\wicsalg$
when the computation started.
This is about three orders of magnitude better than the state of the art,
for calcualting newton polygons, which was \cite{fgmqt-2025-witt-vectors-macaulay2}.

For $p = 11$ and $p = 13$, 
the examples below could be computed in less than 12 compute hours
on the aforementioned 2080Tis (via random guessing).
Newton polygons of K3 surfaces may be computed by using the
library ToricControlledReduction
\cite{chk-2019-toric-controlled-reduction} 
(which computes the zeta
function of the surface, from which the Newton polygon and thus the
height can be deduced).
Our algorithm beats ToricControlledReduction by about 1.5 orders of magnitude:
on an Nvidia 3090 our method took 
around [TIME] seconds per surface (one CPU thread) in characteristic \(p = 11\), 
while ToricControlledReduction took 
about 43 seconds per surface.
For \(p = 13\), our 
code took [TIME] seconds per surface, while ToricControlledReduction
again took about 43 seconds per surface.
Since ToricControlledReduction has complexity \(O(p^{1 / 2})\) as \(p\) 
grows, we expect that if we implemented our algorithms for larger primes,
ToricControlledReduction would eventually beat it.


% p=11,
% us: 39s,TCR: error
% us: 39s,TCR: error
% us: 39s,TCR: error
% us: 39s,TCR: error

% us: 39s, TCR: 43s

% p=13, 
% us: 113 s, TCR: 43 s
% us: 102 s, TCR: 42 s
% us: 104 s, TCR: error
% us: 105 s, TCR: error


\begin{figure}[htbp]
	\begin{center}
        \def\arraystretch{1.5}
		\begin{tabular}{p{0.1\linewidth}p{0.8\linewidth}}
			 \toprule
             \textbf{Height} & \textbf{Equation} \\
			 \midrule
			 1 & \(x_{1}^{4} + x_{2}^{4} + x_{3}^{4} + x_{4}^{4}\) \\
			  
			 2 & \(4 x_1^4 + 2 x_1^3 x_2 + x_1^3 x_4 + 4 x_1^2 x_2^2 + 2 x_1^2 x_2 x_3 + 2 x_1^2 x_3^2 + x_1^2 x_3 x_4 + 3 x_1 x_2^3 + 4 x_1 x_2^2 x_3 + 4 x_1 x_2^2 x_4 + 2 x_1 x_2 x_3 x_4 + 3 x_1 x_2 x_4^2 + 3 x_1 x_3^3 + x_1 x_3^2 x_4 + x_1 x_3 x_4^2 + x_1 x_4^3 + 4 x_2^4 + 2 x_2^3 x_3 + 4 x_2^3 x_4 + 4 x_2^2 x_3^2 + x_2^2 x_3 x_4 + 2 x_2^2 x_4^2 + 3 x_2 x_3^3 + 4 x_2 x_3^2 x_4 + 4 x_2 x_3 x_4^2 + 2 x_2 x_4^3 + 2 x_3^4 + 2 x_3^3 x_4 + 2 x_3^2 x_4^2 + x_3 x_4^3 + 4 x_4^4\) \\
			  
			 3 & \(2 x_1^4 + x_1^3 x_2 + 3 x_1^3 x_3 + x_1^3 x_4 + x_1^2 x_2 x_3 + 4 x_1^2 x_2 x_4 + x_1^2 x_3^2 + 4 x_1^2 x_3 x_4 + 3 x_1^2 x_4^2 + 4 x_1 x_2^3 + 3 x_1 x_2^2 x_3 + x_1 x_2^2 x_4 + 2 x_1 x_2 x_3^2 + 3 x_1 x_2 x_3 x_4 + x_1 x_3^3 + 4 x_1 x_3 x_4^2 + 2 x_1 x_4^3 + x_2^3 x_3 + 3 x_2^3 x_4 + 4 x_2^2 x_3^2 + 4 x_2^2 x_3 x_4 + x_2^2 x_4^2 + 2 x_2 x_3^3 + 3 x_2 x_3^2 x_4 + 4 x_2 x_3 x_4^2 + 3 x_2 x_4^3 + 4 x_3^4 + 3 x_3^3 x_4 + 2 x_3 x_4^3 + 3 x_4^4\) \\
			  
			 4 & \(4 x_1^4 + 2 x_1^3 x_3 + 4 x_1^3 x_4 + 3 x_1^2 x_2^2 + 3 x_1^2 x_2 x_3 + 4 x_1^2 x_2 x_4 + 2 x_1^2 x_3 x_4 + x_1^2 x_4^2 + 3 x_1 x_2^3 + x_1 x_2^2 x_3 + x_1 x_2^2 x_4 + x_1 x_2 x_3^2 + x_1 x_2 x_3 x_4 + x_1 x_2 x_4^2 + 2 x_1 x_3^3 + 2 x_1 x_3^2 x_4 + x_1 x_3 x_4^2 + 2 x_1 x_4^3 + 4 x_2^4 + 3 x_2^3 x_3 + x_2^3 x_4 + 3 x_2^2 x_3^2 + 3 x_2^2 x_3 x_4 + x_2^2 x_4^2 + 2 x_2 x_3^3 + 3 x_2 x_3^2 x_4 + x_2 x_3 x_4^2 + 3 x_2 x_4^3 + 3 x_3^4 + 2 x_3^3 x_4 + 4 x_3^2 x_4^2 + x_3 x_4^3\) \\
			  
			 5 & \(2x_1^4 + 2x_1^3x_2 + 4x_1^3x_3 + 3x_1^3x_4 + 2x_1^2x_2^2 + 4x_1^2x_2x_3 + x_1^2x_2x_4 + 2x_1^2x_3^2 + 3x_1^2x_3x_4 + 4x_1x_2^2x_3 + 3x_1x_2^2x_4 + x_1x_2x_3^2 + x_1x_2x_3x_4 + 2x_1x_2x_4^2 + 3x_1x_3^3 + 3x_1x_3^2x_4 + x_1x_3x_4^2 + x_2^4 + x_2^3x_3 + 2x_2^2x_3^2 + 2x_2^2x_3x_4 + 3x_2^2x_4^2 + 2x_2x_3^3 + 2x_2x_3^2x_4 + 2x_2x_3x_4^2 + 4x_3^4 + x_3^3x_4 + 2x_3^2x_4^2 + 3x_4^4\) \\
			  
			 6 & \(x_1^3x_2 + x_1^3x_3 + 3x_1^3x_4 + 3x_1^2x_2^2 + 2x_1^2x_2x_4 + 4x_1^2x_3x_4 + 4x_1^2x_4^2 + x_1x_2^3 + 3x_1x_2^2x_3 + 4x_1x_2^2x_4 + 2x_1x_2x_3^2 + 2x_1x_2x_3x_4 + 2x_1x_2x_4^2 + 2x_1x_3^3 + 3x_1x_3^2x_4 + 3x_1x_3x_4^2 + x_1x_4^3 + 4x_2^4 + x_2^3x_3 + x_2^3x_4 + x_2^2x_3^2 + 4x_2^2x_3x_4 + x_2^2x_4^2 + 3x_2x_3^3 + 2x_2x_3^2x_4 + 3x_2x_4^3 + 4x_3^4 + x_3^3x_4 + 3x_3x_4^3 + x_4^4\) \\
			  
			 7 & \(4x_1^4 + x_1^3x_3 + 3x_1^3x_4 + 4x_1^2x_2^2 + 2x_1^2x_2x_3 + 2x_1^2x_2x_4 + 2x_1^2x_3^2 + 4x_1^2x_3x_4 + 4x_1x_2^2x_3 + 2x_1x_2x_3^2 + x_1x_2x_4^2 + 2x_1x_3^3 + 4x_1x_3^2x_4 + 2x_1x_3x_4^2 + x_1x_4^3 + 4x_2^4 + 3x_2^3x_4 + 3x_2^2x_3^2 + x_2^2x_3x_4 + 2x_2^2x_4^2 + 3x_2x_3^2x_4 + 4x_2x_3x_4^2 + 3x_2x_4^3 + 3x_3^3x_4 + x_3^2x_4^2 + 4x_4^4\) \\
			  
			 8 & \(x_1^4 + 2x_1^3x_2 + 4x_1^3x_3 + x_1^2x_2^2 + 4x_1^2x_2x_3 + x_1^2x_2x_4 + x_1^2x_3x_4 + 2x_1x_2^2x_3 + 2x_1x_2^2x_4 + 2x_1x_2x_3^2 + 4x_1x_2x_3x_4 + 3x_1x_2x_4^2 + 3x_1x_3^3 + 4x_1x_3^2x_4 + 3x_1x_3x_4^2 + x_1x_4^3 + 4x_2^4 + 4x_2^3x_3 + x_2^3x_4 + 4x_2^2x_3^2 + 2x_2^2x_3x_4 + x_2^2x_4^2 + 4x_2x_3^2x_4 + x_2x_4^3 + x_3^4 + 2x_3^3x_4 + x_3^2x_4^2 + 4x_4^4\) \\
			  
			 9 & \(3 x_1^4 + 3 x_1^3 x_2 + 3 x_1^3 x_3 + x_1^2 x_2^2 + 3 x_1^2 x_2 x_3 + 3 x_1^2 x_2 x_4 + 3 x_1^2 x_3^2 + 2 x_1^2 x_3 x_4 + 2 x_1^2 x_4^2 + 4 x_1 x_2^3 + 2 x_1 x_2^2 x_3 + 4 x_1 x_2 x_3^2 + 2 x_1 x_2 x_3 x_4 + 4 x_1 x_2 x_4^2 + x_1 x_3^3 + 3 x_1 x_3^2 x_4 + 3 x_1 x_3 x_4^2 + x_1 x_4^3 + 3 x_2^3 x_3 + 4 x_2^3 x_4 + 3 x_2^2 x_3 x_4 + x_2^2 x_4^2 + 4 x_2 x_3^2 x_4 + 4 x_2 x_3 x_4^2 + 4 x_2 x_4^3 + 3 x_3 x_4^3 + 4 x_4^4\) \\
			  
			 10 & \(2 x_1^4 + 4 x_1^3 x_2 + 3 x_1^3 x_3 + x_1^3 x_4 + x_1^2 x_2^2 + 2 x_1^2 x_2 x_3 + 2 x_1^2 x_2 x_4 + 4 x_1^2 x_3^2 + 4 x_1^2 x_3 x_4 + 2 x_1^2 x_4^2 + x_1 x_2^3 + 4 x_1 x_2^2 x_4 + 3 x_1 x_2 x_3^2 + 3 x_1 x_2 x_4^2 + 2 x_1 x_3^3 + 3 x_1 x_3^2 x_4 + 2 x_1 x_3 x_4^2 + x_1 x_4^3 + 3 x_2^4 + 2 x_2^3 x_3 + 2 x_2^3 x_4 + 4 x_2^2 x_3^2 + 3 x_2^2 x_3 x_4 + 3 x_2^2 x_4^2 + x_2 x_3^3 + 2 x_2 x_3 x_4^2 + 2 x_2 x_4^3 + 4 x_3^4 + x_3^3 x_4 + 3 x_3^2 x_4^2 + 4 x_3 x_4^3 + 3 x_4^4\)\\
			 
			 \(\infty\)& \(x_{1}^{4} + x_{2}^{4} + x_{3}^{4} + x_{4}^{4} + x y z w\) \\
             \bottomrule
			  
		\end{tabular}
	\end{center}
	\caption{Quartic K3 surfaces with specified Artin--Mazur height over \(\mathbb{F}_{5}\)}
\end{figure}

\begin{figure}[htbp]
	\begin{center}
		\def\arraystretch{1.5}
		\begin{tabular}{p{0.1\linewidth}p{0.8\linewidth}}
			 \toprule
             \textbf{Height} & \textbf{Equation} \\
			 \midrule
			 
			 1 & \(5x_1^4 + 5x_1^3x_3 + 2x_1^3x_4 + 3x_1^2x_2x_3 + x_1^2x_2x_4 + 6x_1^2x_3^2 + 3x_1^2x_3x_4 +
                 3x_1^2x_4^2 + 4x_1x_2^3 + 6x_1x_2^2x_3 + 2x_1x_2^2x_4 + 4x_1x_2x_3^2 + 5x_1x_2x_3x_4 + 4x_1x_2x_4^2
                 + 5x_1x_3^3 + 4x_1x_3^2x_4 + 5x_1x_4^3 + 5x_2^4 + x_2^3x_3 + 4x_2^3x_4 + 5x_2^2x_3^2 + x_2^2x_3x_4 +
                 x_2x_3^3 + 2x_2x_3^2x_4 + 2x_2x_3x_4^2 + x_2x_4^3 + 3x_3^4 + 5x_3^3x_4 + 3x_3^2x_4^2 + x_4^4 \) \\
			 
             2 & \(3x_1^4 + 4x_1^3x_2 + x_1^3x_3 + x_1^3x_4 + x_1^2x_2^2 + 5x_1^2x_2x_3 + 5x_1^2x_2x_4 + 
             3x_1^2x_3^2 + 5x_1^2x_3x_4 + 6x_1^2x_4^2 + 2x_1x_2^3 + x_1x_2^2x_3 + 5x_1x_2^2x_4 + 2x_1x_2x_3x_4 + 
             x_1x_2x_4^2 + 2x_1x_3^3 + 3x_1x_3^2x_4 + x_1x_3x_4^2 + x_1x_4^3 + 4x_2^4 + 4x_2^3x_3 + 4x_2^3x_4 + 
             6x_2^2x_3^2 + 3x_2^2x_3x_4 + 3x_2x_3^3 + 4x_2x_3x_4^2 + 2x_3^4 + 4x_3^3x_4 + 4x_3^2x_4^2 + 2x_3x_4^3 + 6x_4^4\) \\
			 
			 3 &  \(4x_1^4 + x_1^3x_2 + 2x_1^3x_3 + 6x_1^3x_4 + 6x_1^2x_2^2 + 3x_1^2x_2x_3 +
             3x_1^2x_2x_4 + 2x_1^2x_3x_4 + 4x_1^2x_4^2 + 2x_1x_2^3 + 5x_1x_2^2x_4 + 5x_1x_2x_3^2 +
             4x_1x_2x_3x_4 + 4x_1x_2x_4^2 + 6x_1x_3^3 + x_1x_3^2x_4 + 5x_1x_3x_4^2 + 2x_1x_4^3 +
             3x_2^4 + 2x_2^3x_3 + 5x_2^2x_3^2 + 5x_2^2x_3x_4 + 3x_2^2x_4^2 + 4x_2x_3^3 +
             6x_2x_3^2x_4 + 5x_2x_3x_4^2 + 3x_2x_4^3 + 4x_3^3x_4 + 4x_3^2x_4^2 + x_3x_4^3 + 5x_4^4\)
             \\
			 
			 4 & \(2x_1^4 + 6x_1^3x_2 + 3x_1^3x_3 + x_1^3x_4 + 4x_1^2x_2^2 + 3x_1^2x_2x_3 +
             3x_1^2x_2x_4 + 2x_1^2x_3^2 + x_1^2x_3x_4 + 2x_1^2x_4^2 + 3x_1x_2^3 + 6x_1x_2^2x_4 +
             x_1x_2x_3^2 + 6x_1x_2x_3x_4 + x_1x_2x_4^2 + 4x_1x_3^3 + 2x_1x_3^2x_4 + 5x_1x_3x_4^2 +
             2x_1x_4^3 + 6x_2^4 + 3x_2^3x_3 + 5x_2^2x_3^2 + x_2^2x_3x_4 + 5x_2^2x_4^2 + 4x_2x_3^3 +
             3x_2x_3^2x_4 + x_2x_4^3 + 6x_3^4 + 2x_3^3x_4 + x_3^2x_4^2 + 3x_3x_4^3 + 2x_4^4\) \\
			 
			 5 & \(5x_1^4 + 6x_1^3x_2 + 2x_1^3x_3 + 3x_1^3x_4 + 4x_1^2x_2^2 + 3x_1^2x_2x_4 +
             2x_1^2x_3^2 + 3x_1^2x_3x_4 + 6x_1^2x_4^2 + 4x_1x_2^2x_3 + 6x_1x_2^2x_4 + 2x_1x_2x_3^2 +
             3x_1x_2x_3x_4 + 5x_1x_2x_4^2 + 3x_1x_3^3 + x_1x_3^2x_4 + 5x_1x_3x_4^2 + 6x_2^4 +
             5x_2^3x_3 + 3x_2^2x_3^2 + 6x_2^2x_3x_4 + 3x_2x_3^3 + 3x_2x_3^2x_4 + 4x_2x_3x_4^2 +
             3x_2x_4^3 + 5x_3^4 + 6x_3^2x_4^2 + 6x_3x_4^3 + 3x_4^4\) \\
			 
			 6 & \(x_1^4 + x_1^3x_2 + 4x_1^3x_3 + 6x_1^3x_4 + 6x_1^2x_2^2 + 2x_1^2x_2x_4 +
             6x_1^2x_3x_4 + 6x_1^2x_4^2 + 4x_1x_2^3 + 3x_1x_2^2x_3 + 2x_1x_2^2x_4 + 2x_1x_2x_3^2 +
             5x_1x_2x_3x_4 + 6x_1x_2x_4^2 + 6x_1x_3^2x_4 + 3x_1x_3x_4^2 + 6x_2^4 + 2x_2^3x_3 +
             3x_2^3x_4 + 5x_2^2x_3^2 + 4x_2^2x_3x_4 + 6x_2^2x_4^2 + 5x_2x_3^2x_4 + x_2x_3x_4^2 +
             3x_2x_4^3 + 2x_3^4 + 2x_3^3x_4 + 5x_3^2x_4^2 + 2x_3x_4^3 + 4x_4^4 \) \\
			 
			 7 & \(2*x_1^4 + 3*x_1^3*x_2 + 4*x_1^3*x_3 + 4*x_1^3*x_4 + 6*x_1^2*x_2^2 + 2*x_1^2*x_2*x_3 + x_1^2*x_2*x_4 + 6*x_1^2*x_3^2 + 4*x_1^2*x_3*x_4 + x_1^2*x_4^2 + 5*x_1*x_2^3 + x_1*x_2^2*x_3 + 6*x_1*x_2^2*x_4 + 3*x_1*x_2*x_3^2 + 6*x_1*x_2*x_4^2 + 4*x_1*x_3^3 + 5*x_1*x_3^2*x_4 + 2*x_1*x_3*x_4^2 + x_1*x_4^3 + 5*x_2^4 + 4*x_2^3*x_3 + x_2^3*x_4 + 5*x_2^2*x_3^2 + x_2^2*x_3*x_4 + 4*x_2*x_3^3 + x_2*x_3*x_4^2 + 2*x_2*x_4^3 + 5*x_3^4 + x_3^3*x_4 + 2*x_3^2*x_4^2 + 4*x_3*x_4^3 + 4*x_4^4\) \\
			 
			 8 & \(2x_1^3x_2 + 2x_1^3x_4 + 4x_1^2x_2^2 + 6x_1^2x_2x_3 + 5x_1^2x_2x_4 + 4x_1^2x_3^2 +
             3x_1^2x_3x_4 + 3x_1^2x_4^2 + 4x_1x_2^3 + x_1x_2^2x_3 + x_1x_2^2x_4 + 4x_1x_2x_3^2 +
             5x_1x_2x_3x_4 + x_1x_2x_4^2 + 3x_1x_3^3 + x_1x_3^2x_4 + 3x_1x_3x_4^2 + x_1x_4^3 +
             5x_2^3x_3 + 5x_2^3x_4 + 6x_2^2x_3x_4 + 6x_2^2x_4^2 + 4x_2x_3^2x_4 + 3x_2x_3x_4^2 +
             2x_2x_4^3 + 6x_3^3x_4 + 6x_3^2x_4^2 + 4x_3x_4^3\) \\
			 
			 9 & \(2x_1^3x_2 + x_1^3x_3 + 6x_1^3x_4 + 6x_1^2x_2^2 + 4x_1^2x_2x_3 + 2x_1^2x_2x_4 +
             3x_1^2x_3x_4 + x_1^2x_4^2 + x_1x_2^3 + x_1x_2^2x_3 + 6x_1x_2^2x_4 + 6x_1x_2x_3^2 +
             6x_1x_2x_3x_4 + 6x_1x_2x_4^2 + 2x_1x_3^3 + 4x_1x_3x_4^2 + 6x_1x_4^3 + 6x_2^3x_3 +
             4x_2^3x_4 + 3x_2^2x_3^2 + 4x_2x_3^3 + 5x_2x_3^2x_4 + 4x_2x_3x_4^2 + 5x_2x_4^3 +
             3x_3^3x_4 + 4x_3^2x_4^2 + 2x_3x_4^3 + 3x_4^4\) \\
			 
			 10 & \( 3x_1^4 + 2x_1^3x_2 + x_1^3x_3 + x_1^3x_4 + 4x_1^2x_2x_3 + 2x_1^2x_2x_4 +
             5x_1^2x_3x_4 + 6x_1^2x_4^2 + x_1x_2^3 + 2x_1x_2^2x_4 + 5x_1x_2x_3^2 + 3x_1x_2x_3x_4 +
             4x_1x_2x_4^2 + 5x_1x_3^3 + x_1x_3^2x_4 + x_1x_3x_4^2 + x_1x_4^3 + 6x_2^4 + x_2^3x_4 +
             6x_2^2x_3^2 + x_2^2x_3x_4 + 4x_2^2x_4^2 + x_2x_3^3 + 5x_2x_4^3 + 2x_3^4 + 5x_3^3x_4 +
             5x_3^2x_4^2 + x_3x_4^3 + 6x_4^4\) \\
			 
             \(\infty\) & \( 3x_1^4 + 3x_1^3x_2 + 3x_1^3x_3 + 6x_1^2x_2^2 + 3x_1^2x_2x_4 + 2x_1^2x_3^2 + 2x_1^2x_3x_4 + 3x_1^2x_4^2 + 6x_1x_2^3 + 5x_1x_2^2x_3 + x_1x_2x_3x_4 + 5x_1x_2x_4^2 + 5x_1x_3^3 + 4x_1x_3^2x_4 + 3x_1x_3x_4^2 + 6x_1x_4^3 + x_2^4 + 4x_2^3x_4 + 3x_2^2x_3^2 + 5x_2^2x_3x_4 + 5x_2x_3^3 + x_2x_3^2x_4 + 6x_2x_3x_4^2 + x_3^3x_4 + x_3^2x_4^2 + 3x_3x_4^3 + 4x_4^4\) \\
             \bottomrule
		\end{tabular}
	\end{center}
	\caption{Quartic K3 surfaces with specified Artin--Mazur height over \(\mathbb{F}_{7}\)}
\end{figure}

\begin{figure}[htbp]
	\begin{center}
		\def\arraystretch{1.5}
		\begin{tabular}{p{0.1\linewidth}p{0.8\linewidth}}
			 \toprule
             \textbf{Height} & \textbf{Equation} \\
			 \midrule
			 1 & $4x_1^4 + 6x_1^3x_2 + x_1^3x_3 + 2x_1^3x_4 + 3x_1^2x_2^2 + x_1^2x_2x_3 + 3x_1^2x_2x_4 + 6x_1^2x_3^2 + 6x_1^2x_3x_4 + 8x_1^2x_4^2 + 7x_1x_2^3 + 2x_1x_2^2x_3 + 8x_1x_2^2x_4 + 8x_1x_2x_3x_4 + 10x_1x_2x_4^2 + 10x_1x_3^3 + 9x_1x_3^2x_4 + 6x_1x_3x_4^2 + 3x_1x_4^3 + 6x_2^4 + 7x_2^3x_3 + 4x_2^3x_4 + 10x_2^2x_3^2 + 3x_2^2x_3x_4 + 5x_2^2x_4^2 + 4x_2x_3^2x_4 + 6x_2x_4^3 + 3x_3^4 + 4x_3^3x_4 + 7x_3^2x_4^2 + 9x_3x_4^3 + 5x_4^4 $\\
			  
			 2 & $4x_1^4 + 5x_1^3x_2 + 9x_1^3x_3 + 2x_1^3x_4 + 8x_1^2x_2^2 + x_1^2x_2x_3 + 9x_1^2x_2x_4 + x_1^2x_3^2 + 8x_1^2x_3x_4 + 6x_1x_2^3 + 10x_1x_2^2x_3 + 2x_1x_2^2x_4 + 10x_1x_2x_3^2 + 9x_1x_2x_3x_4 + 6x_1x_2x_4^2 + 8x_1x_3^3 + 4x_1x_3^2x_4 + 7x_1x_3x_4^2 + 9x_1x_4^3 + 3x_2^4 + 7x_2^3x_3 + 6x_2^3x_4 + 10x_2^2x_3^2 + 8x_2^2x_3x_4 + x_2^2x_4^2 + 9x_2x_3^3 + 6x_2x_3^2x_4 + x_2x_3x_4^2 + 9x_3^4 + 10x_3^3x_4 + x_3^2x_4^2 + x_3x_4^3 + 4x_4^4$\\
			  
			 3 & $10x_1^4 + 9x_1^3x_2 + 5x_1^3x_3 + 4x_1^3x_4 + 3x_1^2x_2^2 + 9x_1^2x_2x_3 + 4x_1^2x_2x_4 + 10x_1^2x_3^2 + 4x_1^2x_3x_4 + 8x_1^2x_4^2 + 8x_1x_2^3 + 9x_1x_2^2x_3 + 3x_1x_2^2x_4 + 7x_1x_2x_3^2 + 3x_1x_2x_4^2 + 8x_1x_3^3 + 2x_1x_3^2x_4 + x_1x_3x_4^2 + 7x_1x_4^3 + 2x_2^4 + 3x_2^3x_4 + x_2^2x_3^2 + x_2^2x_3x_4 + x_2^2x_4^2 + 5x_2x_3^3 + 9x_2x_3^2x_4 + 9x_2x_3x_4^2 + 4x_2x_4^3 + 5x_3^4 + 10x_3^3x_4 + 10x_3x_4^3 + 10x_4^4$\\
             
			 4 & $2x_1^4 + 4x_1^3x_2 + 9x_1^3x_3 + 10x_1^3x_4 + 2x_1^2x_2^2 + 4x_1^2x_2x_3 + 4x_1^2x_2x_4 + 4x_1^2x_3^2 + 10x_1^2x_3x_4 + 9x_1^2x_4^2 + 5x_1x_2^3 + 5x_1x_2^2x_3 + x_1x_2^2x_4 + 8x_1x_2x_3^2 + 2x_1x_2x_3x_4 + 10x_1x_2x_4^2 + 8x_1x_3^3 + 7x_1x_3^2x_4 + 5x_1x_3x_4^2 + 4x_1x_4^3 + 3x_2^4 + 6x_2^3x_3 + 4x_2^3x_4 + 10x_2^2x_3^2 + 5x_2^2x_3x_4 + 5x_2^2x_4^2 + x_2x_3^3 + 5x_2x_4^3 + 5x_3^4 + 7x_3^2x_4^2 + 5x_3x_4^3 + 9x_4^4$\\
			 
			 5 & \(10x_1^4 + x_1^3x_2 + 6x_1^3x_3 + 3x_1^3x_4 + x_1^2x_2^2 + 9x_1^2x_2x_3 + 6x_1^2x_2x_4 + 6x_1^2x_3^2 + 8x_1^2x_3x_4 + 4x_1^2x_4^2 + 3x_1x_2^3 + 7x_1x_2^2x_3 + 3x_1x_2^2x_4 + 7x_1x_2x_3^2 + 9x_1x_2x_3x_4 + 8x_1x_2x_4^2 + 7x_1x_3^3 + x_1x_3x_4^2 + 7x_1x_4^3 + x_2^4 + 3x_2^3x_3 + 7x_2^3x_4 + 5x_2^2x_3^2 + 7x_2^2x_3x_4 + 8x_2^2x_4^2 + 8x_2x_3^3 + 5x_2x_3^2x_4 + x_2x_3x_4^2 + 9x_2x_4^3 + 7x_3^4 + 4x_3^3x_4 + 4x_3^2x_4^2 + 3x_3x_4^3\)\\
             \bottomrule
		\end{tabular}
	\end{center}
	\caption{Quartic K3 surfaces with specified Artin--Mazur height over \(\mathbb{F}_{11}\)}
\end{figure}

\begin{figure}[htbp]
	\begin{center}
		\def\arraystretch{1.5}
		\begin{tabular}{p{0.1\linewidth}p{0.8\linewidth}}
			 \toprule
             \textbf{Height} & \textbf{Equation} \\
			 \midrule
			 1 & \(6x_1^4 + 7x_1^3x_3 + 4x_1^3x_4 + 6x_1^2x_2^2 + 7x_1^2x_2x_3 + 9x_1^2x_2x_4 + 2x_1^2x_3^2 + 3x_1^2x_3x_4 + 12x_1^2x_4^2 + 8x_1x_2^3 + 4x_1x_2^2x_3 + x_1x_2^2x_4 + 9x_1x_2x_3^2 + 8x_1x_2x_3x_4 + 10x_1x_2x_4^2 + 8x_1x_3^3 + 2x_1x_3^2x_4 + 9x_1x_3x_4^2 + 4x_1x_4^3 + 5x_2^4 + 4x_2^3x_3 + 2x_2^2x_3^2 + x_2^2x_3x_4 + 2x_2^2x_4^2 + 10x_2x_3^3 + 2x_2x_3^2x_4 + 2x_2x_3x_4^2 + 5x_2x_4^3 + 4x_3^4 + 3x_3^2x_4^2 + 2x_4^4\)\\
			  
			 2 & \(12x_1^4 + 8x_1^3x_2 + 8x_1^3x_3 + 10x_1^3x_4 + 8x_1^2x_2^2 + 11x_1^2x_2x_4 + 8x_1^2x_3^2 + 12x_1^2x_3x_4 + x_1^2x_4^2 + 7x_1x_2^3 + 9x_1x_2^2x_3 + 11x_1x_2^2x_4 + 10x_1x_2x_3^2 + 7x_1x_2x_4^2 + 8x_1x_3^3 + 3x_1x_3^2x_4 + 11x_1x_3x_4^2 + x_1x_4^3 + 4x_2^4 + 7x_2^3x_3 + 4x_2^3x_4 + 8x_2^2x_3^2 + 12x_2^2x_3x_4 + 6x_2^2x_4^2 + 7x_2x_3^3 + 12x_2x_3^2x_4 + 4x_2x_3x_4^2 + 10x_2x_4^3 + 4x_3^4 + 8x_3^3x_4 + 5x_3^2x_4^2 + 4x_3x_4^3\)\\
			  
			 3 & \(8x_1^4 + 2x_1^3x_2 + 3x_1^3x_3 + x_1^3x_4 + 6x_1^2x_2^2 + 7x_1^2x_2x_3 + 5x_1^2x_2x_4 + 2x_1^2x_3^2 + x_1^2x_4^2 + 11x_1x_2^3 + 10x_1x_2^2x_3 + 3x_1x_2^2x_4 + 5x_1x_2x_3^2 + 10x_1x_2x_3x_4 + 7x_1x_2x_4^2 + 12x_1x_3^3 + 12x_1x_3^2x_4 + 5x_1x_3x_4^2 + 7x_1x_4^3 + 7x_2^4 + 6x_2^3x_3 + 3x_2^3x_4 + 10x_2^2x_3^2 + 5x_2^2x_3x_4 + 12x_2^2x_4^2 + x_2x_3^3 + 3x_2x_3^2x_4 + 12x_2x_3x_4^2 + 8x_2x_4^3 + 10x_3^4 + 7x_3^3x_4 + 4x_3^2x_4^2 + 8x_3x_4^3 + 2x_4^4\)\\
             
			 4 & \(4x_1^4 + 4x_1^3x_2 + 2x_1^3x_3 + 3x_1^3x_4 + 9x_1^2x_2^2 + 6x_1^2x_2x_3 + 7x_1^2x_2x_4 + 10x_1^2x_3^2 + x_1^2x_3x_4 + 4x_1x_2^3 + 4x_1x_2^2x_3 + 6x_1x_2^2x_4 + 12x_1x_2x_3^2 + 7x_1x_2x_3x_4 + 3x_1x_2x_4^2 + 11x_1x_3^3 + 9x_1x_3^2x_4 + 10x_1x_3x_4^2 + 11x_1x_4^3 + 3x_2^4 + 5x_2^3x_3 + 8x_2^3x_4 + 5x_2^2x_3x_4 + 5x_2^2x_4^2 + 5x_2x_3^3 + 10x_2x_3^2x_4 + 2x_2x_3x_4^2 + 10x_2x_4^3 + 4x_3^4 + 5x_3^2x_4^2 + 4x_3x_4^3 + 6x_4^4\)\\
			  
			 5 & \(11x_1^4 + 4x_1^3x_2 + 12x_1^3x_3 + 4x_1^3x_4 + 6x_1^2x_2^2 + 10x_1^2x_2x_3 + 4x_1^2x_2x_4 + x_1^2x_3^2 + 7x_1^2x_3x_4 + 4x_1^2x_4^2 + 6x_1x_2^3 + 11x_1x_2^2x_3 + 7x_1x_2^2x_4 + 8x_1x_2x_3^2 + 10x_1x_2x_4^2 + x_1x_3^3 + 9x_1x_3^2x_4 + 8x_1x_3x_4^2 + 11x_1x_4^3 + 4x_2^4 + 8x_2^3x_3 + 5x_2^2x_3x_4 + 7x_2^2x_4^2 + 8x_2x_3^3 + 6x_2x_3^2x_4 + 5x_2x_4^3 + 2x_3^4 + 10x_3^3x_4 + 8x_3^2x_4^2 + 10x_3x_4^3 + 6x_4^4\) \\
            \bottomrule
		\end{tabular}
	\end{center}
	\caption{Quartic K3 surfaces with specified Artin--Mazur height over \(\mathbb{F}_{13}\)}
\end{figure}
