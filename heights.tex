\subsection{Recollections on the moduli of K3 surfaces}

We have the following:

\begin{thm}
	[Lang-Weil, Theorem 1]
	Let \(X \ins \mathbb{P}^{n}\)
	be a projective variety of dimension \(r\). 
	Then 
	\[
		\#X(\mathbb{F}_{p}) = p^r + O(p^{r - \frac{1}{2}})
	\] 
\end{thm}

Say we have an ambient projective variety
\(Y\) with chosen
hypersurface \(D\).
The Lang-Weil estimate roughly
says that if we pick a random point \(x\),
we can expect \(x\) to lie in \(D\) 
with probably about \(1 / p\).
By \cite[Section~7]{artin-1974-k3-surfaces}, 
the locus \(M_{i}\) of height \(i\) such that \(h \leq i\)
is cut out by a single section in 
\(M_{i-1}\).
If we apply both of these facts,
with \(Y\) the 
moduli space\footnote{
	say, the course moduli space of polarized K3 surfaces,
	though one should be able to make a similar statement
	for the moduli stack
} of quartic K3 surfaces,
we can conclude that the locus 
the probability of a random point 
in the moduli space being in the locus
\(M_{2}\) is about \(1 / p\). 
Inductively, we see 

\begin{heur}
The probability of a randomly
chosen surface having height \(h\) is about \(1 / p^{h}\).
\end{heur}

Thus, we may find a surface of height \(h\) by 
randomly choosing quartic K3 surfaces, and 
if we compute a few times \(p^{h}\) 
samples, we can be confident that we'll find
one with high probability.

In practice, our methodology is to choose
random quartic polynomials,
by sampling a point in the vector space
of homogeneous polynomials, which is isomorphic to
\(\mathbb{F}_{p}^{35}\).
To obtain the moduli space \(M\) of quartic K3 surfaces, 
we must projectivize and take the quotient by the action of 
\(PGL_{4}\) which acts by changes of variables. 
Thus, there may be two sources of deviation from the expected
probability--the group quoteint and the actual 
error term in the estimate.
 
For \(p = 7\) and above, we restrict 
that may appear in the random polynomial (to speed up the FFT),
so this likely further effect the probability of getting
a surface of height \(h\).



\subsection{Computations}

Over \(\mathbb{F}_{5}\), we 
found that the probability of finding a K3 surface
of height \(h\) was about \(1 / 5^{h}\), to three digits
of precision, for all heights \(h \leq 6\).
For higher heights we had less samples
and more variance, although
the probabilities seem to be less than expected
for higher heights.
Similarly, over \(\mathbb{F}_{7}\) we found that
the probabilities very closely matched \(1 / 7^{h}\) 
for low heights, with a more variance at higher heights.

For \(p=5\), the algorithm throughput was about
40 surfaces per second on 
Nvidia 1080Ti and 2080Ti GPUs provided by the 
UCSD research cluster.
Since much less time is taken for height \(1\) examples
(since the classical Fedder's criterion suffices), 
for the purposes of estimating the time to compute a
high height example we may ignore them.
Thus,
the expected compute time necessary to find a height = \(10\) 
example is about  \(5^{9} / 40 \approx 48000\) seconds, or about 
\(12\)-\(13\) compute hours.
For \(p=7\), the throughput was about 10 surfaces
per second.
Thus, the expected compute time to compute
a height = \(10\) example is 
\(7^{9} / 10 \approx \) 4 million seconds, or a little
more than 45 compute days.
The actual times to find the examples roughly
matched these numbers, though the \(p=5\) 
computation was carried out using the \(\merge\) 
algorithm for ``multiply then split''
because the authors had not yet discovered \(\wics\),
so that computation took a few days.

\begin{figure}[h]
	\begin{center}
		\begin{tabular}{|p{0.1\linewidth}|p{0.8\linewidth}|}
			 \hline
             Height & Equation \\
			 \hline
			 1 & \(x_{1}^{4} + x_{2}^{4} + x_{3}^{4} + x_{4}^{4}\) \\
			 \hline 
			 2 & \(4 x_1^4 + 2 x_1^3 x_2 + x_1^3 x_4 + 4 x_1^2 x_2^2 + 2 x_1^2 x_2 x_3 + 2 x_1^2 x_3^2 + x_1^2 x_3 x_4 + 3 x_1 x_2^3 + 4 x_1 x_2^2 x_3 + 4 x_1 x_2^2 x_4 + 2 x_1 x_2 x_3 x_4 + 3 x_1 x_2 x_4^2 + 3 x_1 x_3^3 + x_1 x_3^2 x_4 + x_1 x_3 x_4^2 + x_1 x_4^3 + 4 x_2^4 + 2 x_2^3 x_3 + 4 x_2^3 x_4 + 4 x_2^2 x_3^2 + x_2^2 x_3 x_4 + 2 x_2^2 x_4^2 + 3 x_2 x_3^3 + 4 x_2 x_3^2 x_4 + 4 x_2 x_3 x_4^2 + 2 x_2 x_4^3 + 2 x_3^4 + 2 x_3^3 x_4 + 2 x_3^2 x_4^2 + x_3 x_4^3 + 4 x_4^4\) \\
			 \hline 
			 3 & \(2 x_1^4 + x_1^3 x_2 + 3 x_1^3 x_3 + x_1^3 x_4 + x_1^2 x_2 x_3 + 4 x_1^2 x_2 x_4 + x_1^2 x_3^2 + 4 x_1^2 x_3 x_4 + 3 x_1^2 x_4^2 + 4 x_1 x_2^3 + 3 x_1 x_2^2 x_3 + x_1 x_2^2 x_4 + 2 x_1 x_2 x_3^2 + 3 x_1 x_2 x_3 x_4 + x_1 x_3^3 + 4 x_1 x_3 x_4^2 + 2 x_1 x_4^3 + x_2^3 x_3 + 3 x_2^3 x_4 + 4 x_2^2 x_3^2 + 4 x_2^2 x_3 x_4 + x_2^2 x_4^2 + 2 x_2 x_3^3 + 3 x_2 x_3^2 x_4 + 4 x_2 x_3 x_4^2 + 3 x_2 x_4^3 + 4 x_3^4 + 3 x_3^3 x_4 + 2 x_3 x_4^3 + 3 x_4^4\) \\
			 \hline 
			 4 & \(4 x_1^4 + 2 x_1^3 x_3 + 4 x_1^3 x_4 + 3 x_1^2 x_2^2 + 3 x_1^2 x_2 x_3 + 4 x_1^2 x_2 x_4 + 2 x_1^2 x_3 x_4 + x_1^2 x_4^2 + 3 x_1 x_2^3 + x_1 x_2^2 x_3 + x_1 x_2^2 x_4 + x_1 x_2 x_3^2 + x_1 x_2 x_3 x_4 + x_1 x_2 x_4^2 + 2 x_1 x_3^3 + 2 x_1 x_3^2 x_4 + x_1 x_3 x_4^2 + 2 x_1 x_4^3 + 4 x_2^4 + 3 x_2^3 x_3 + x_2^3 x_4 + 3 x_2^2 x_3^2 + 3 x_2^2 x_3 x_4 + x_2^2 x_4^2 + 2 x_2 x_3^3 + 3 x_2 x_3^2 x_4 + x_2 x_3 x_4^2 + 3 x_2 x_4^3 + 3 x_3^4 + 2 x_3^3 x_4 + 4 x_3^2 x_4^2 + x_3 x_4^3\) \\
			 \hline 
			 5 & \(4 x_1^2 x_2^2 + 4 x_1^2 x_2 x_3 + 2 x_1^2 x_2 x_4 + 3 x_1^2 x_3 x_4 + 4 x_1 x_2^2 x_3 + 3 x_1 x_2 x_3 x_4 + x_1 x_3^3 + x_1 x_3^2 x_4 + 3 x_1 x_3 x_4^2 + x_1 x_4^3 + x_2^4 + 2 x_2 x_3^2 x_4 + 4 x_2 x_3 x_4^2\) \\
			 \hline 
			 6 & \(4 x_1^3 x_2 + 4 x_1^3 x_4 + 4 x_2^4 + 2 x_2 x_3^2 x_4 + x_2 x_3 x_4^2 + x_3^3 x_4\) \\
			 \hline 
			 7 & \(4 x_1^3 x_4 + x_1 x_2^3 + 2 x_1 x_3^3 + 2 x_2^3 x_4 + x_2^2 x_4^2 + x_3^3 x_4 + 3 x_3^2 x_4^2 + 2 x_3 x_4^3\) \\
			 \hline 
			 8 & \(2 x_1^3 x_3 + x_1^2 x_2^2 + x_1^2 x_4^2 + x_1 x_2^2 x_4 + 3 x_1 x_3^2 x_4 + x_2^4 + 2 x_2^3 x_3 + 3 x_2^2 x_4^2 + 4 x_2 x_3^2 x_4 + 3 x_3 x_4^3\) \\
			 \hline 
			 9 & \(3 x_1^4 + 3 x_1^3 x_2 + 3 x_1^3 x_3 + x_1^2 x_2^2 + 3 x_1^2 x_2 x_3 + 3 x_1^2 x_2 x_4 + 3 x_1^2 x_3^2 + 2 x_1^2 x_3 x_4 + 2 x_1^2 x_4^2 + 4 x_1 x_2^3 + 2 x_1 x_2^2 x_3 + 4 x_1 x_2 x_3^2 + 2 x_1 x_2 x_3 x_4 + 4 x_1 x_2 x_4^2 + x_1 x_3^3 + 3 x_1 x_3^2 x_4 + 3 x_1 x_3 x_4^2 + x_1 x_4^3 + 3 x_2^3 x_3 + 4 x_2^3 x_4 + 3 x_2^2 x_3 x_4 + x_2^2 x_4^2 + 4 x_2 x_3^2 x_4 + 4 x_2 x_3 x_4^2 + 4 x_2 x_4^3 + 3 x_3 x_4^3 + 4 x_4^4\) \\
			 \hline 
			 10 & \(2 x_1^4 + 4 x_1^3 x_2 + 3 x_1^3 x_3 + x_1^3 x_4 + x_1^2 x_2^2 + 2 x_1^2 x_2 x_3 + 2 x_1^2 x_2 x_4 + 4 x_1^2 x_3^2 + 4 x_1^2 x_3 x_4 + 2 x_1^2 x_4^2 + x_1 x_2^3 + 4 x_1 x_2^2 x_4 + 3 x_1 x_2 x_3^2 + 3 x_1 x_2 x_4^2 + 2 x_1 x_3^3 + 3 x_1 x_3^2 x_4 + 2 x_1 x_3 x_4^2 + x_1 x_4^3 + 3 x_2^4 + 2 x_2^3 x_3 + 2 x_2^3 x_4 + 4 x_2^2 x_3^2 + 3 x_2^2 x_3 x_4 + 3 x_2^2 x_4^2 + x_2 x_3^3 + 2 x_2 x_3 x_4^2 + 2 x_2 x_4^3 + 4 x_3^4 + x_3^3 x_4 + 3 x_3^2 x_4^2 + 4 x_3 x_4^3 + 3 x_4^4\)\\
			 \hline
			 \(\infty\)& \(x_{1}^{4} + x_{2}^{4} + x_{3}^{4} + x_{4}^{4} + x y z w\) \\
			 \hline 
		\end{tabular}
	\end{center}
	\caption{Quartic K3 surfaces with specified Artin-Mazur height over \(\mathbb{F}_{5}\)}
\end{figure}

\begin{figure}[h]
	\begin{center}
		\begin{tabular}{|c|c|}
			 \hline
			 Height & Equation \\
			 \hline
			 1 & \\
			 \hline 
			 2 & \\
			 \hline 
			 3 & \\
			 \hline 
			 4 & \\
			 \hline 
			 5 & \\
			 \hline 
			 6 & \\
			 \hline 
			 7 & \\
			 \hline 
			 8 & \\
			 \hline 
			 9 & \\
			 \hline 
			 10 & \\
			 \hline
			 \(\infty\)& \\
			 \hline 
		\end{tabular}
	\end{center}
	\caption{Quartic K3 surfaces with specified Artin-Mazur height over \(\mathbb{F}_{7}\)}
\end{figure}

\begin{figure}[h]
	\begin{center}
		\begin{tabular}{|c|c|}
			 \hline
			 Height & Equation \\
			 \hline
			 1 & \\
			 \hline 
			 2 & \\
			 \hline 
			 3 & \\
             \hline
			 \(\infty\)& \\
			 \hline 
		\end{tabular}
	\end{center}
	\caption{Quartic K3 surfaces with specified Artin-Mazur height over \(\mathbb{F}_{11}\)}
\end{figure}

\begin{figure}[h]
	\begin{center}
		\begin{tabular}{|c|c|}
			 \hline
			 Height & Equation \\
			 \hline
			 1 & \\
			 \hline 
			 2 & \\
			 \hline 
			 3 & \\
             \hline
			 \(\infty\)& \\
			 \hline 
		\end{tabular}
	\end{center}
	\caption{Quartic K3 surfaces with specified Artin-Mazur height over \(\mathbb{F}_{13}\)}
\end{figure}

%\subsection{Raising polynomials to powers}
%\subsection{Matrix of Multiply then Split}
%
%
%
%\subsection{Matrix Multiplication}
%
%Magma benchmarks were done by taking the average of 3 runs,
%Julia benchmarks were taken using @benchmark
% size & magma CPU & magma GPU & Oscar/Flint 
%       & Naive Julia CPU & Naive Julia GPU & Our fallback implementation
% 
%\subsubsection{A note on Benchmarking on GPUs}
%
%Both CUDA.jl and MAGMA 
%claim to
%use the CUBLAS library for their
%matrix multiplication, so one might expect them to
%give similar performance.
%However, we see above that the naive Julia matmul implementation
%beats MAGMA by about one order of magnitude. 
%There are two likely causes of this discrepancy. 
%Firstly, as mentioned in the CUDA.jl documentation (citation), 
%naive benchmarking on the GPU may not be robust. 
%We use the 
%\texttt{BenchmarkTools}
%library for robust measurments. 
%If we use a less robust method like the \texttt{CUDA.@time} macro,
%sometimes we get a much larger time.
%MAGMA benchmarks are taken using the \texttt{Realtime()}
%function, which may be subject to such inconsistencies.
%However, this cannot account for all of the discrepancy, 
%because even using inconsistent benchmarks in Julia, 
%the times are still often the same order of magnitude
%as \texttt{BenchmarkTools}.
%
%Another possible source of discrepancy is copying data between
%the CPU and the GPU. 
%The authors are not familiar with the implementation of MAGMA's 
%GPU multiplication, but based on some simple GPU profiling,
%MAGMA seems to be copying memory between the CPU's and GPU's memory.
%Perhaps the input matrices are copied to the GPU, or the result
%is copied back to the CPU, or both.
%When running Julia benchmarks that include copying the data back and forth
%between the CPU and GPU, one gets times that are on the
%order of magnitude of the MAGMA times. 
%In such Julia benchmarks, the time spent copying dominates the time spent 
%multiplying.
%
%Also note: Julia, Oscar/FLINT, and (probably) MAGMA use the same
%BLAS library on the CPU in this case, just Julia has multithreading enabled
%by default
%
%* MAGMA seems to be slower
