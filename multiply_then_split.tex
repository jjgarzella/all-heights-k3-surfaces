
To find the matrix of \(\theta\), mathematically
speaking we must first multiply by \(\Delta_{1}(f)\) 
and then apply the map \(u\).
Our first observation is that naively 
multiplying and then applying \(u\) 
as in algorithm \ref{alg:naive:u}
does a lot of unnecessary work.
In particular, any term in the 
product \(\Delta_{1}(f)a\) which 
is not congruent to 
\((p-1, \ldots, p-1) \mod p\)
is not needed. 
Even if \(a\) is a monomial, giving a linear
algoritm for multiplication, using naive
multiplication stores a lot of unnecessary terms
in memory.
Here, we give an algorithm that calculates
the matrix of \(\theta\) by only
using such terms.
We start with an elementary fact.

\begin{lem}
	\label{lem:tuples:modp}
	Let \(\underbar{a} = (a_{1}, \ldots, a_{n}),
	\underbar{b} = (b_{1}, \ldots, b_{n})\)
	be in \(\{0, \ldots, p-1\}^{n} \ins \mathbb{Z}^{n}\).

	Then \(\underbar{a} + \underbar{b} = 
	(p-1, \ldots, p-1) \mod p\)
	if and only if 
	\(\underbar{a} + \underbar{b} = 
	(p-1, \ldots, p-1)\).
\end{lem}

\begin{proof}
	Each coordinate has \(a_{i} + b_{i} \leq 2(p-1) = 2p-2\).
\end{proof}

\begin{defn}
	Let \(M\) be a monomial in \(S\). 
	Then we say that \(M\) \textit{matches}
	with another monomial \(M^{\prime}\) 
	if the monomial \(MM^{\prime}\) is
	not killed by \(u\).
\end{defn}

\begin{cor}
	Let \(\underbar{a}\) be as above.
	Let \(\underbar{x}^{\underbar{a}}\)
	denote \(x_{1}^{a_{1}} \ldots x_{n}^{a_{n}}\).
	Then there exists a unique
	match \(m(\underbar{a}) \in \{0, \ldots, p-1\}^{n}\) 
\end{cor}

\begin{proof}
	\(m(\underbar{a}) = 
	(p-1, \ldots, p-1) - \underbar{a}\).
	The claim follows from 
	\ref{lem:tuples:modp}.
\end{proof}

Essentially, this corollary says that
when we consider arbitrary monomials \(M\),
when we consider its exponent vector 
mod \(p\), it has a unique
match mod \(p\).
However, we have even more:

\begin{cor}
	\label{cor:match:order}
	Let \(\leq_{lex}\) denote the 
	lexographical ordering.
	If \(\underbar{a} \leq_{lex} \underbar{b}\),
	then 
	\(m(\underbar{b}) \leq_{lex} m(\underbar{a})\)
\end{cor}

\begin{proof}
	Follows from the definition of lexographical
	ordering.
\end{proof}

The previous two corollaries justify
the following algorithm.

\begin{cxt}
	Let \(f \in S\) be a homogeneous
	polynomial of degree \(n\).
	Let \(\Delta\) be an array containing the 
	exponent vectors of \(\Delta_{1}(f^{p-1})\).
	Let \(M\) be an array containing the exponent
	vectors of the monomial basis of
	homogeneous polynomials of degree
	\(n*(p-1)\).
\end{cxt}

In what follows, when we put a dot (\(.\))
before an operation, that denotes
componentwise application
of that operation, similar to Julia syntax.

\begin{alg}
	\(answer \colonequals []\)
	\(L \colonequals \Delta .\% p\) 
	and \(R \colonequals M .\% p\).

	Sort both \(L\) and \(R\) mod \(p\).
	Sort \(\Delta\) and \(M\) according to the 
	same permutation.

	Let \(\ell \colonequals 1, r \colonequals \text{length}(R)\)

	while \(1 \leq r \&\& l \leq \text{length}(L)\)

	--- \(cmp \colonequals L[\ell] + R[r] - (p-1, \ldots, p-1)\)

	--- switch \(cmp\)

	--- --- case  \(cmp < 0\) 

	--- --- --- \(l += 1\) 

	--- --- case \(0 < cmp\) 

	--- --- --- \(r -= 1\) 

	--- --- case \(cmp == 0\) 

	--- --- --- \(matchr \colonequals R[r]\) 

    --- --- --- \(numLmatches \colonequals \) the number of
	adjacent entrices in \(L\) that are equal to \(L[\ell]\)

	--- --- --- while \(R[r] == matchr\) 

	--- --- --- --- for \(\ell\ell\) in \(\ell\) to \(\ell\) + numLmatches

	--- --- --- --- --- Extract \(\Delta[\ell\ell]\) 
	and its corresponding coeffient,
	as well as \(M[r]\)

	--- --- --- --- --- Apply \(u\) to the product
	of the resulting monomials

	--- --- --- --- --- add this term to \(answer\)

	--- --- --- ---  end for

	--- --- --- --- \(r += 1\) 

	--- --- --- end while

	--- --- --- \(l += numLmatches\) 
	
	--- end switch

	end while

\end{alg}

Note that the inner while loops account for the fact that
after modding by \(p\), one does not expect 
either array to have unique keys.
The algorithm is justified by the previous corollaries.

In practice, instead of sorting the arrays in place, 
we use a method like julia's ``sortperm''
%TODO: code font
to get the sort permutation.

We also may use the kronecker substitution to make
the comparisons and additions use machine integers
rather than vectors.
In practice, this speeds the algorithm up a lot.

* complexity analysis
n = length L, m = length R
we need n + m + n log n + m log m operations.
Thus, the algorithm is big o of max(n log n, m log m),
and in practice the \(\Delta_{1}\) length
usually dominates.

* picture









